\chapter{Intro al Análisis Funcional}

\section{¿Qué estudia el Análisis Funcional?}

Estudia los espacios vectoriales de dimensión infinita y las transformaciones lineales entre ellos.

\begin{fdefinition}
    Un espacio vectorial $V$ sobre $\mathbb{K}$ campo de escalares tiene dimensión infinita si $\forall n\in\N$ hay $n$ elementos de $V$ que son linealmente independientes sobre $\mathbb{K}$
\end{fdefinition}

\begin{fexample}
    $V=C([0,1],\R)=$ funciones reales continuas en $[0,1]$.

    $\{1,x,\ldots,x^{n-1}\}\subseteq V$ es linealmente independiente sobre $\R$.
\end{fexample}

\begin{proof}
    $\displaystyle\sum_{k=0}^{n-1}a_kx^k\equiv 0$, $a_k\in\R$.

    Reconocemos que existe la operación $\frac{d}{dx}$ definida en $C^{\infty}([0,1],\R)$, funciones suaves, y la operación evaluar en $x=0$.

    Evaluando en $x=0\to a_0=0$. Derivamos a los lados.

    \[\sum_{k=1}^{n-1}a_k k x^{k-1}\equiv 0\]

    y ahora evaluamos en $x=0$:
    
    $$a_1=0$$...
\end{proof}

\begin{proof}[Demostración alternativa]
    Reconocemos que hay un producto interno en $V=C([0,1],\R)$

    \[\langle f,g\rangle=\int_0^1 f(x)g(x)\,dx\]

    $\{f_k=\sin (\pi kx)\}_{k=1}^n\subseteq V$

    \[\langle \sin (\pi kx),\sin (\pi lx)\rangle=\begin{cases} 0&k\neq l\\
    \frac{1}{2}&k=l\end{cases}\]

    \[S=\sum_{k=1}^n a_kf_k\equiv 0\]

    \[0=\langle S,f_k\rangle=\left\langle \sum a_kf_k,f_l\right\rangle=a_l\langle f_0,f_l\rangle =\frac{1}{2}a_l\]

    \[\implies a_l=0, \forall l=1,\ldots,n\]

\end{proof}

\section{Motivación}

\begin{fexample}[Ecuación de Poisson]
    \[\begin{cases}
        \Delta u=f&\text{en }\Omega\subseteq \R^n\\
        u=0&\text{en } \partial \Omega
    \end{cases}\]

    \seba{Aañdir dibujo}

    El problema se reformula así:

    \[\begin{cases}
        D=\Delta:x\to Y\ni f\\
        Du=f
    \end{cases}\]
    
    tiene una solución $u\in X$ para ciertos espacios $X,Y$ apropiados.
\end{fexample}

El Análaisis Funcional busca construir teoría más general que aplica para todos los problemas que \color{red} comparten \color{black}las \color{red}mismas características \color{black}topológicas/algebraicas/métricas.

\section{Objeto central: espacio de Banach}

\begin{fdefinition}[Espacio de Banach]
    $(V,||\cdot ||)$ es un espacio de Banach si es un espacio normado \color{red}completo \color{black}(clave para sacar límites).
\end{fdefinition}

\[\{\text{Espacios de Hilbert}, (V,\langle \cdot,\cdot\rangle) completos\}\subseteq\{\text{Espacios de Banach}, (V,||\cdot||)\}\subseteq \{\text{Espacios métricos}, (V,d) completos\}\]

\seba{Arreglar}

\paragraph{Lógica de inclusiones}

\begin{enumerate}
    \item $\langle\cdot,\cdot\rangle$ induce una norma $||\cdot||$

    \[||v||=\langle v,v\rangle^{1/2}\]

    \item $||\cdot||$ induce una métrica $d(\cdot,\cdot)$

    \[d(v,w)=||v-w||\]
\end{enumerate}

\section{Resultados que vamos a ver}

\begin{enumerate}
    \item Resultados que se parecen a los teoremas que conocemos en la situación de dimensión finita.

    \begin{fexample}
        Cada funcional lineal en $\R$ ($l:\R^{n}\to \R$) se puede representar como $l(v)=v\cdot w$ para algún vector (único) $w\in\R^n$.

    En la situación de dimensión $\infty$, se tiene el Teorema de Representación de Riesz:

    \begin{ftheorem}[Representación de Riesz]
        Sea $(V,\langle,\rangle)$ un espacio de Hilbert y $l:V\to \R$ un funcional lineal \color{red}continuo \color{black}. Entonces existe un único $w\in V$, tal que

        \[l(v)=\langle v,w\rangle\]
    \end{ftheorem}
    \end{fexample}

    \item Resultados son muy diferentes de la situación en dimensión finita. \color{red} contraintuitivos \color{black}.

    \begin{fexample}
        $\overline{B_1(0)}\subseteq \R^n$ es compacta (Heine-Borel). 

        En $\dimn V=\infty$, este teorema es falso.
    \end{fexample}
    
    \begin{fproposition}
        Sea $V$ un espacio de Banach y sea $B=\{v\in V:||v||\leq 1\}$. $B$ es compacto en $V$ $\iff \dimn V<\infty$
    \end{fproposition}

    \begin{fexample}
        En particular, la bola unitaria cerrada en

        \[B\subseteq L^p([0,1]),\quad p\in (1,\infty)\]

        no es compacta.

        \color{blue}$\implies$ motiva la definición de \color{red} topologías débiles.
    \end{fexample}
    
\end{enumerate}
