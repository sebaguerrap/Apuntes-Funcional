\chapter{Espacios de Hilbert}

\section{Conceptos Básicos}

\begin{fdefinition}
    Sea $H$ un espacio vectorial sobre $\K=\R$ o $\C$. Un producto interno $\langle \cdot,\cdot\rangle$ es una función $H\times H\to \K$ que satisface

    \begin{enumerate}
        \item Linealidad en $\langle\cdot,y\rangle,\quad \forall y\in H$:
        
        \[\langle x_1+x_2,y\rangle=\langle x_1,y\rangle+\langle x_2,y\rangle\]
        \[\langle \lambda x,y\rangle = \lambda \langle x,y\rangle\]

        \item (Hermiticidad)
        
        \[\langle y,x\rangle=\overline{\langle x,y\rangle}\]

        (En $\K=\R$, esto es simetría)

        \item (Definidad) $\langle x,x\rangle\geq 0$ y $\langle x,x\rangle=\implies x=0$
    \end{enumerate}
\end{fdefinition}

\begin{fnote}
    1. y 2., implican que $\langle x,\cdot\rangle$ es lineal conjugada en la segunda entrada.

    \[\langle x,\lambda y+z\rangle=\overline{\lambda}\langle x,y\rangle + \langle x,z\rangle\]
\end{fnote}

\paragraph*{Terminología} Tal función se llama \textbf{forma sesquilineal}

\begin{fnote}
    $\K=\R$, $\inn{\cdot,\cdot}$ es una \textbf{forma simétrica definida positiva}
\end{fnote}

Decimos que $(H,\inn{\cdot,\cdot})$ es un \textbf{espacio pre-Hilbertiano}

De 1. y 2., $\inn{0,y}=0$, $\inn{x,0}=0$

Definimos $||x||:=\inn{x,x}^{1/2}$

\begin{fproposition}[Desigualdad de Cauchy-Schwarz]
    Sea $H$ un espacio pre-Hilbertiano

    \[|\inn{x,y}|\leq ||x||\cdot ||y||\quad \forall x,y\in H\]
\end{fproposition}

\begin{proof}
    Si $y=0$, la desigualdad es verdadera. Podemos asumir que $y\neq 0$.

    \begin{align*}
        0&\leq \inn{x+\lambda y,x+\lambda y}\\
        &=\inn{x,x}+\lambda\inn{y,x}+\overline{\lambda}\inn{x,y}+\lambda \overline{\lambda}\inn{y,y}\\
        &=||x||^2 +\underbrace{\lambda \overline{\inn{x,y}}+\overline{\lambda}\inn{x,y}}_{2\Re (\inn{x,y}\overline{\lambda})}+|\lambda|^2|\cdot |y||^2
    \end{align*}

    Evaluando en $\lambda=-\dfrac{\inn{x,y}}{||y||^2}$

    \begin{align*}
        0&\leq ||x||^2+2\Re (\inn{x,y}\frac{-\overline{\inn{x,y}}}{||y||^2})\\
        0&\leq ||x||^2-2\frac{|\inn{x,y}|^2}{||y||^2}+\frac{|\inn{x,y}|^2}{||y||^2}\\
        &\implies ||x||^2\geq \frac{|\inn{x,y}|^2}{||y||^2}
    \end{align*}
\end{proof}

\begin{fproposition}
    $||\cdot||$ define una norma $H$.
\end{fproposition}
\begin{proof}
    \begin{enumerate}
        \item Definidad $\checkmark$
        \item $||\lambda x||=\inn{\lambda x,\lambda x}^{1/2}=(\lambda \overline{\lambda}||x||^2)^{1/2}=|\lambda|\cdot ||x||$
        \item (Desigualdad triangular)
        
        \begin{align*}
            ||x+y||^2=||x||^2+2\Re (\inn{x,y})+||y||^2&\leq ||x||^2+2||x||\cdot ||y||+||y||^2\\
            &=(||x||+||y||)^2
        \end{align*}
    \end{enumerate}
\end{proof}

\begin{fproposition}
    $\inn{\cdot,\cdot}$ es continuo en $H\times H$
\end{fproposition}

\begin{proof}
    $x_n\to x$ en $||\cdot$ e $y_n\to y$ en $||\cdot||$

    \begin{align*}
        |\inn{x_n,y_n}-\inn{x,y}|&=|\inn{x_n-x,y_n}+\inn{x,y_n-y}|\\
        &\leq |\inn{x_n-x,y_n}|+|\inn{x,y_n-y}|\\
        &\leq ||x_n-x||\cdot ||y_n||+||x||\cdot ||y_n-y||\\
        &\xrightarrow[n\to\infty]{} 0
    \end{align*}
\end{proof}

\begin{fdefinition}
    Decimos que $x\perp y$ en el espacio pre-Hilbertiano $H$ si $\inn{x,y}=0$. Si $E\subseteq H$ subconjunto, definimos el \textbf{espacio ortogonal}

    \[E^\perp :=\{x\in H:x\perp y\quad \forall y\in E\}\]
\end{fdefinition}

$E^\perp$ es un \textbf{subespacio} de $H$ y es cerrado:

$x_n\in E^\perp$ y $x_n\to x$ en $H$ entonces 

\[\inn{x,y}=\lim_{n\to\infty} \inn{x_n,y}=0\quad \forall y\in E\]

\begin{ftheorem}[Pitagoras]
    Si $x_1,\ldots,x_n\in H$ (pre-Hilbertiano) son mutuamente ortogonales, entonces 

    \[||x_1+\cdots+x_n||^2=\sum_{k=1}^n ||x_k||^2\]
\end{ftheorem}

\begin{fproposition}[Ley del paralelogramo]
    \[||x+y||^2+||x-y||^2=2||x||^2+2||y||^2\]
\end{fproposition}

\begin{proof}
    \begin{align*}
        ||x\pm y||^2=||x||^2\pm 2\Re \inn{x,y}+||y||^2
    \end{align*}

    Sumando los 2 términos (diagonales), estamos listos.
\end{proof}

\begin{fdefinition}
    Decimos que un espacio $(H,\inn{\cdot,\cdot})$ pre-Hilbertiano es un espacio de \textbf{Hilbert} si es \textbf{completo} respecto $||\cdot||$ inducida por $\inn{\cdot,\cdot}$
\end{fdefinition}

\begin{fexample}
    $(\C^n,\inn{\cdot,\cdot})$. $\inn{x,y}=\sum_{k=1}^n x_k\overline{y_k}$ es un espacio de Hilbert.
\end{fexample}

\begin{fexample}
    $(\ell^2,\inn{\cdot,\cdot})$. $\inn{\{x_k\},\{y_k\}}=\sum_{k=1}^\infty x_k\overline{y_k}$
\end{fexample}

¿$\ell^p$ tiene una estructura de espacio de Hilbert? $\iff p=2$

\begin{fexample}
    $(X,\mathcal{M},\mu)$ es un espacio de medida, definimos 

    \[L^2(X,\mathcal{M},\mu)=\{f:X\to\C\text{ medibles}:\int_X|f|^2\,d\mu<\infty\}/_\sim\]

    $f_1\sim f_2$ si $\{f_1\neq f_2\}$ es despreciable.
\end{fexample}

\section{Teorema de la Proyección}

Sea $H$ un espacio de Hilbert. $C\subseteq R^n$ cerrado y convexo. Existe único $y\in C$ tal que $y$ minimiza la distancia entre $x$ y $C$.

\begin{fdefinition}
    Sea $C$ un subconjunto de un espacio vectorial $V$. Decimos que $C$ es \textbf{convexo} en $V$ si

    \[\forall x,y\in C\quad (1-t)x+ty\in C\quad \forall t\in [0,1]\]
\end{fdefinition}

\begin{ftheorem}
    Sea $C\subseteq H$ un subconjunto cerado y convexo del espacio de Hilbert $H$. Entonces $\forall x\in H,\exists! y=P_C x\in C$ que satisface:

    \[||x-P_C x||=d(x,C)=\inf_{c\in C} ||x-c||\]

    Además, $y=P_C x\iff \Re \inn{c-y,x-y}\leq 0,\quad \forall c\in C$
\end{ftheorem}

\begin{proof}
    Tome $\{y_n\}\subseteq C$, tal que 

    \[d_n:=||x-y_n||\xrightarrow{n\to\infty} d:=d(x_n,c)\]

    $\{y_n\}$ será convergente si es Cauchy, ya que $y_n\to y\in H$. Ya que $C$ es cerrado, de hecho $y\in C$.

    Por la ley del paralelogramo, con $v=x-y_n,w=x-y_m$

    \begin{align*}
        2 d_n^2+2 d_m^2&=||v-w||^2+||v+w||^2\\
        &=||y_n-y_m||^2+||2x-(y_n+y_m)||^2\\
        &=||y_n-y_m||^2+4\left|\left|x-\underbrace{\frac{y_n+y_m}{2}}_{\in C}\right|\right|^2\\
        &\geq ||y_n-y_m||^2+4d^2
    \end{align*}

    Luego,

    \begin{align*}
        ||y_n-y_m||^2&\leq 2d_n^2+d_m^2-4d^2\\
        &\xrightarrow{n,m\to\infty} 0
    \end{align*}

    por lo que $\{y_n\}$ es Cauchy.

    $y=\displaystyle\lim_{n\to\infty} y_n$,

    \[||x-y||=\lim_{n\to\infty} \overbrace{||x-y_n||}^{d_n}=d\]

    Este minimizador es el único!. Si hubiera otro $z\neq y$, aplicamos el mismo argumento a $\{y,z,y,z,\ldots\}$ que no converge por construcción, pero es Cauchy, lo que es una contradicción.

    $\implies:$ Sea $c\in C$ y considere $(1-t)y+tc$, $t\in [0,1]$.

    \begin{align*}
        ||x-(1-t)y-tc||^2&=||x-y-t(c-y)||^2\\
        &=||x-y||^2-2t\Re \inn{x-y,c-y}+t^2||c-y||^2\\
        &\geq ||x-y||^2
    \end{align*}

    \[\implies 2t\Re \inn{x-y,c-y}\leq t^2||c-y||^2\]
    \[\implies 2\Re \inn{x-y,c-y}\leq 0\]

    $\impliedby:$ Evalúe $||x-(1-t)y+tc||^2$ en $t=1$.

    \begin{align*}
        ||x-c||^2&=||x-y||^2-2\Re \inn{x-y,c-y}+||c-y||^2\\
        \implies & ||x-c||^2-||x-y||^2=||c-y||^2-2\Re \inn{x-y,c-y}\\
        \implies &||x-c||^2\geq ||x-y||^2\quad \forall c\in C
    \end{align*}

    Tenemos igualdad $\iff c=y$. 
\end{proof}

\begin{fexample}
    $W\subseteq H$ es un subespacio $\implies W$ es convexo.
\end{fexample}

\begin{ftheorem}
    Sea $F\subseteq H$ un subespacio cerrado. Entonces $H=F\oplus F^\perp$, es decir, que todo $x\in H$ se puede escribir de manera única como $x=y+z$ con $y\in F$ y $z\in F^\perp$. Además $y=P_Fx, z=P_{F^\perp} x$. y 

    \[P_F:H\to H\]

    es lineal, acotado y satisface:

    \begin{itemize}
        \item $||P_F||\leq 1$ ($=1$ cuando $F=\{0\}$)
        \item $P_F^2=P_F$
        \item $\I P_F=F$, $\ker P_F=F^\perp$
        \item $\inn{P_Fx_1,x_2}=\inn{x_1,P_Fx_2}$
    \end{itemize}
\end{ftheorem}

\begin{fdefinition}
    $P_F$ se llama la \textbf{proyección ortogonal}
\end{fdefinition}

\begin{proof}
    Ya que $F\cap F^\perp=\{0\}$, la unicidad se cumple.

    \[y+z=y'+z'\implies y-y'=z'-z=0\]

    Tome $x\in H$. Define $y=P_Fx$. Queremos demostrar que $x:x-y\in F^\perp$. Del teorema \ref{theo:2.5.1} sabemos que 
    
    $$\Re\inn{c-y,x-y}\leq 0\quad \forall c\in F$$.

    \[\implies \Re \inn{v,z}\leq 0\quad\forall v\in F\]
    \[\implies \Re \inn{\lambda v,z}\leq 0\quad \forall \lambda\in \K\]
    \[\implies \Re \lambda\inn{v,z}\leq 0\]

    \seba{añadir align}

    tome $\lambda=\overline{\inn{v,z}}$

    \[\implies\Re |\inn{v,z}|^2\leq 0\]
    \[\implies |\inn{v,z}|=0\implies z\in F^\perp\]

    \large{Propiedades de $P_F$:} $x_1=y_1+z,\quad x_2=y_2+z_2$

    \begin{align*}
        \inn{P_Fx_1,x_2}&=\inn{y_1,x_2}\\
        &=\inn{y_1,y_2+z_2}\\
    \end{align*}

    \begin{align*}
        \inn{x_1,P_F x_2}&=\inn{y_1+z_1,y_2}\\
        &=\inn{y_1,y_2}
    \end{align*}

    Por lo que $P_F$ es lineal

    \begin{align*}
        \inn{P_F(x_1+x_2),x_3}&=\inn{x_1+x_2,P_F x_3}\\
        &=\inn{x_1,P_Fx_3}+\inn{x_2,P_Fx_3}\\
        &=\inn{P_Fx_1,x_3}+\inn{P_Fx_2,x_3}\\
        &=\inn{(P_Fx_1+P_Fx_2),x_3}
    \end{align*}

    \[\iff P_F(x_1+x_2)=P_Fx_1+P_Fx_2\]

    $P_F(\lambda x)=\lambda P_F x$ de la misma manera.

    $P_F/_F=\Id/_F$ 

    \begin{align*}
        &\implies P_F^2 x=P_F(P_Fx)=P_F x\quad \forall x\in H\\
        &\implies P_F^2=P_F
    \end{align*}

    $||P_F x||^2=||y||^2\leq ||x||^2$ mientras

    \[||x||^2\leq ||y||^2+||z||^2\]
    \[\implies ||P_F||\leq 1\]

\end{proof}

\section{Teorema de Representación de Riesz}

\begin{ftheorem}
    Sea $H$ un espacio de Hilbert y sea $f\in H^*$ un funcional lineal acotado. Entonces existe único $u\in H$ tal que 

    \[f(x)=\inn{x,u}\quad \forall x\in H\]
\end{ftheorem}

\paragraph*{Observaciones}

\begin{enumerate}
    \item $||f||_*=||u||$ por Cauchy-Schwarz
    \item \begin{align*}
        H^*&\to H\\
        f&\to u_f
    \end{align*}
    es una isometría biyectiva, lineal-conjugada. Para todo $v\in H$ define $f_v(x):\inn{x,v}$
    \item $f_1+f_2\to u_{f_1+f_2}=u_{f_1}+u_{f_2}$, ya que
    
    \begin{align*}
        (f_1+f_2)(x)&=f_1(x)+f_2(x)=\inn{x,u_{f_1}}+\inn{x,u_{f_2}}\\
        &=\inn{x,u_{f_1}+u_{f_2}}\implies u_{f_1+f_2}=u_{f_1}+u_{f_2}
    \end{align*}

    \item ¿$\lambda f\to u_{\lambda f}=\lambda u_f$?
    
    \begin{align*}
        [\lambda f](x)=\lambda (f(x))=\lambda\inn{x,u_f}=\inn{x,\overline{\lambda}u_f}
    \end{align*}
\end{enumerate}

\begin{fnote}
    Teorema falso. Cuando $H$ es solo espacio pre-Hilbertiano, por ejemplo,

    \[H=C([-1,1])\]

    con producto interno usual.

    \[f(x)=\int_0^1 x(t)\,dt\in H^*\]

\end{fnote}

\begin{proof}
    Si $f=0\implies u=0$. Asumimos que $f\neq 0$ y consideramos $F:=\ker f=\{x\in H:f(x)=0\}$. $F$ es un subespacio de $H$ cerrado. Si $f\neq 0\implies F\neq H$. Por el teorema de la proyección (\ref{theo:3.2.2})

    \[H=F\oplus F^\perp\]

    Elije $z\in F^\perp\setminus\{0\}$. Afirmamos que $u=\overline{f(z)}z|z|^2\neq 0$ satisface $f=\inn{\cdot,u}$. Ya que 

    \begin{align*}
        f(z)x-f(x)z\in F\\
        \implies f(z)x-f(x)z\perp z\\
        \inn{f(z)x,z}-\inn{f(x)z,z}=0\\
        \implies \inn{x,\overline{f(z)}z}=f(x)||z||^2\\
        \implies f(x)=\inn{x,\frac{\overline{f(x)}z}{||z||^2}}
    \end{align*}

    Entonces $u\in H$ que satisface $f=\inn{\cdot,u}$. Es único: si tenemos $u,u'\in H$ 

    \begin{align*}
        f(x)&=\inn{x,u}=\inn{x,u'}\\\
        &\implies \inn{x,u-u'}=0\quad \forall x\in H\\
        &\implies u-u'\in H^\perp=\{0\}
    \end{align*}
\end{proof}

\section{Bases Ortonormales}

Sea $V$ un espacio vectorial sobre $\K$. Un subconjunto $\{v_\alpha\}_{\alpha\in A}$ es LI si $\forall I\overset{\text{finito}}{\subseteq} A$, 

\[\sum_{i\in I}c_iv_i=0\implies c_i=0\quad \forall i\in I\]

\[\gen (\{u_\alpha\}_{\alpha\in A})=\left\{\sum_{i\in I}c_iu_i:I\overset{\text{finito}}{\subseteq} A, c_i\in \K\right\}\]

\begin{fdefinition}
    Sea $H$ un espacio de Hilbert, $\{e_\alpha\}_{\alpha\in A}$ es ortonormal (o.n.) si 

    \[\inn{e_\alpha,e_\beta}=\delta_{\alpha\beta}\quad \text{$\delta$ de Kronecker}\]
\end{fdefinition}

Suponga que $\{e_1,\ldots,e_n\}$ es o.n.

\[F:=\gen(\{e_i\}_i^n)\subseteq H\]

es un subespacio cerrado. Podemos definir $P_F$

\[P_F x=\underbrace{\sum_{i=1}^n \inn{x,e_i}e_i}_{y}\]

Es suficiente demostrar que $x-y\perp F$.

\begin{align*}
    \inn{x-\sum_{x,e_i}e_i,e_k}=0\quad \forall k= 1,\ldots,n
\end{align*}

\begin{align*}
    ||P_F x||^2&\leq ||x||^2
\end{align*}

Por Pitagoras

\begin{align*}
    =\sum_{i=1}^n ||\inn{x,e_i}e_i||^2\leq ||x||^2\\
    \implies \sum_{i=1}^n |\inn{x,e_i}|^2\leq ||x||^2
\end{align*}

\begin{fproposition}[Desigualdad de Bessel]
    Sea $S=\{e_\alpha\}_\alpha$ un conjunto o.n. Entonces,

    \[\sum_{\alpha} |\inn{x,e_\alpha}|^2\leq ||x||^2 \]
\end{fproposition}

\[\sum_\alpha r_\alpha:=\sup\left\{\sum_{i\in I}r_i:I\subseteq A\right\}\]

\begin{proof}
    Utilizando $\sum_{i=1}^n |\inn{x,e_i}|^2\leq ||x||^2$, y tomando supremo.
\end{proof}

\paragraph{Consecuencias} $\{\alpha:\inn{x,e_\alpha}\neq 0\}=\bigcup_{n=1}^\infty \{\alpha \in A:|\inn{x,e_\alpha}|\geq \frac{1}{n}\}$ es \textbf{contable}: Si es infinito: $|\inn{x,e_{\alpha_k}}|^2>\frac{1}{n^2}, k=1,\ldots$. Sumando suficientes términos superaríamos $||x||^2$, que no es posible por Bessel.

\begin{fdefinition}
    \[\hat x(\alpha)=\inn{x,e_\alpha}\]

    \textbf{coeficientes de Fourier} respecto a $\{e_\alpha\}$
\end{fdefinition}

\[\sum_\alpha |\hat x(\alpha)|^2\leq ||x||^2\]

¿Cuando tenemos igualdad?

\begin{ftheorem}
    Sea $\mathcal{B}=\{e_\alpha\}_{\alpha\in A}$ un subconjunto o.n. del espacio de Hilbert $H$. Los siguientes enunciados son equivalentes:

    \begin{enumerate}
        \item \[\sum_\alpha |\hat x(\alpha)|^2=||x||^2\]
        \item $\mathcal{B}$ es \textbf{maximal} en el sentido de:
        
        Si $x\in H$, tal que $x\perp e_\alpha,\forall \alpha\in A\implies x=0$

        \item $\forall x\in H$, 
        
        \[x=\sum_{\alpha}\inn{x,e_\alpha}e_\alpha\]

        donde la suma en el lado derecho tiene solo un número contable de términos no ceros y la suma de estos converge a $x$ en $||\cdot||$ independiente de su orden.

        \item $\gen(\mathcal{B})$ es denso en $H$
    \end{enumerate}
\end{ftheorem}

\begin{fdefinition}
    Decimos que un conjunto $\{e_\alpha\}_{\alpha\in A}$ o.n. es una \textbf{base ortonormal} si satisface cualquiera de $\emph{1.-4.}$
\end{fdefinition}

\begin{proof}
    $\emph{2.}\implies \emph{3.}$ Sea $e_{\alpha_1},\ldots,e_{\alpha_n},\ldots$ una enumeración de los $\{e_\alpha\}_{\alpha\in \mathcal{J}}$ para los cuales $\hat x(\alpha)\neq 0$. Por Bessel:

    \[\sum_{k=1}^\infty |\hat x(\alpha_k)|^2\leq ||x||^2<\infty\]

    \[\implies \sum_{k=n}^m |\hat x(\alpha_k)|^2\xrightarrow{m,n\to\infty} 0\]

    Por Pitagoras,

    \[||\sum_{k=n^m}\inn{x,e_{\alpha_k}}e_{\alpha_k}||\xrightarrow
    {m,n\to\infty} 0\]

    Sea $S_n=\sum_{k=1}^n \hat x(\alpha_k)e_{\alpha_k}$. $\{S_n\}$ es Cauchy en $H$

    \[\implies S_n\xrightarrow{n\to\infty} S\quad\text{en $H$}\]

    Además 

    \begin{align*}
        \inn{x-S,e_\alpha}&=\inn{x,e_\alpha}-\inn{S,e_\alpha}\\
        &=\inn{x,e_\alpha}-\lim_{n\to\infty} \inn{S_n,e_\alpha}\\
        &=\begin{cases}
            0 &\text{cuando $\alpha\in \mathcal{J}$}\\
            0 &\text{cuando $\alpha\notin \mathcal{J}$}
        \end{cases}
        &\implies x-S=0\implies x=S
    \end{align*}

    $\emph{3.}\implies \emph{1.}$: Por continuidad de la norma 

    \begin{align*}||x||^2&=||\lim_{n\to\infty} S_n||^2\\
        &=\lim_{n\to\infty} ||S_n||^2\\
        &=\lim_{n\to\infty} \sum_{k=1}^n |\hat x(\alpha_k)|^2\\
        &=\sum_\alpha |\hat x(\alpha)|^2
    \end{align*}

    $\emph{1.}\implies \emph{2.}$: obvio 

    \[||x||^2=\sum_\alpha |\inn{x,e_\alpha}|^2=0\implies x=0\]

    $\emph{3.}\implies \emph{4.}$: Si $x\perp e_\alpha,\quad\forall \alpha$, 

    \begin{align*}
        &\implies x\perp \gen(\{e_\alpha\})\\
        &\overset{\text{continuidad}}{\implies} x\perp \overline{\gen(\{e_\alpha\})}=H\\
        &\implies x=0
    \end{align*}

\end{proof}

\begin{fexample}
    $\ell^2$, $e_k=\{(0,\ldots,\underbrace{1}_k,0,\ldots)\},k\in\N$.

    \[||x||^2=\sum |x_i|^2=\sum |\inn{x,e_i}|^2\]
\end{fexample}

\begin{ftheorem}
    Todo espacio de Hilbert tiene una \textbf{base ortonormal}.
\end{ftheorem}

\begin{proof}
    Utiliza el Lema de Zorn
\end{proof}

\begin{fdefinition}
    $X$ espacio métrico es \textbf{separable} si existe un subconjunto $C\subseteq X$ contable y denso en $X$.
\end{fdefinition}

\begin{fexample}
    $\ell^p,p\in [1,\infty)$ es separable. 

    $L^2([0,1])$ es separable. $\text{Polinomios con coeficientes $\in\K$}\overset{\text{denso}}{\subseteq}C([0,1])\overset{\text{denso}}{\subseteq} L^2([0,1])$

    \seba{Faltan los polinomios con coefs $\in\Q$} cuando $\K=\R \text{ o } \C$.
\end{fexample}

\begin{ftheorem}
    $H$ es separable si y solo si existe una \textbf{base ortonormal} para $H$ que es \textbf{contable}. En este caso, toda base o.n. es contable.
\end{ftheorem}

\begin{proof}
    $\implies:$ $\{x_n\}\subseteq H$ es denso. $x_1,\ldots,x_n,\ldots$ Descartando posiblemente términos, podemos asumir que $x_1,\ldots,x_n$ son LI $\forall n\in\N$ y todos los descartados pertenecen a $\gen(\{x_k\})$. De esta manera, $\gen(\{x_k\})$ es denso en $H$.

    Por Gram-Schmidt producimos una sucesión $\{y_k\}_{k=1}^\infty$ tal que, $\gen(\{y_k\}_{k=1}^n)=\gen(\{x_k\}_{k=1}^n)\forall n\in\N$ y $\mathcal{B}=\{y_k\}$ es un conjunto o.n.

    $\mathcal{B}$ es o.n. y $\gen(\mathcal{B})=\gen(\{x_k\})$ es denso en $H$. Entonces $\mathcal{B}$ es una base ortonormal contable.

    $\impliedby:$ Sea $\{e_k\}_k$ una base o.n. contable.

    \[G_n:=\gen(\{e_k\}_{k=1}^n)=\left\{\sum_{k=1}^n \lambda_ke_k,\lambda_k\in\K\right\}\]

    $\implies\gen(\{e_k\}_k)=\bigcup_{n=1}^\infty G_n$ es denso en $H$.

    \[\bigcup_{n=1}^\infty \hat{G_n}\overset{\text{denso}}{\subseteq} \bigcup_{n=1}^\infty G_n\]

    donde $\hat{G_n}=\{\sum_{i=1}^n \lambda_i e_i,\lambda_k\in\Q \text{ si }\K=\R, \lambda_k\in \Q+i\Q \text{ si } \K=\C\}$

    \seba{añadir cases en vola}

    Sea $\{u_\alpha\}_{\alpha\in\mathcal{A}}$ otra base o.n. 

    \[A_n=\left\{\alpha\in\mathcal{A}:\inn{\overbrace{x}^{e_n},u_\alpha}\neq 0\right\}\text{ es contable}\]

    Además, para cada $\alpha\in\mathcal{A}$, 

    \[\inn{u_\alpha,e_k}\neq 0\text{ para algún }k\]

    por la maximalidad de la base $\{e_n\}_n$ (que es contable). Entonces, $\mathcal{A}=\bigcup_{k=1}^\infty A_k$ es contable.
\end{proof}

Vamos a demostrar que todo espacio de Hilbert separable es $\ell^2=\{\{x_k\}\in \K^n:\sum ||x_k|^2|<\infty\}$

\begin{fdefinition}
    Sean $H_1,H_2$ dos espacios de Hilbert. Un \textbf{isomorfismo} $T:H_1\to H_2$ se llama \textbf{unitario} si 

    \[\inn{Tx_1,Tx_2}_{H_2}=\inn{x_1,x_2}_{H_1}\quad\forall x_1,x_2\in H_1\]
\end{fdefinition}

$T$ unitario $\implies T$ es una \textbf{isometría}:

\[||Tx||_{H_2}^2=\inn{Tx,Tx}_{H_2}=\inn{x,x}_{H_1}=||x||_{H_1}^2\]

\begin{ftheorem}
    Todo espacio de Hilbert separable es unitariamente isomorfo a $\ell^2$.
\end{ftheorem}

\begin{proof}
    Sea $\{e_n\}$ una base o.n. contable para $H$.

    \begin{align*}
        H&\to \ell^2\\
        x&\to \hat x=(\hat x(1),\hat x(2),\ldots)
    \end{align*}

    donde $\hat x(k)=\inn{x,e_k}$.

    Por Parseval,

    \[||\hat x||_{\ell^2}^2=\sum_{k}|\hat x(k)|^2=||x||^2<\infty\]

    \[\implies \hat x\in \ell^2\implies T\text{ es bien definido}\]

    es lineal, inyectivo (por maximalidad), sobreyectivo: si $c\in\ell^2, \sum_{k=1}^n c_k e_k\xrightarrow{H} x_c$, donde 

    \[\hat x_c(k)=\inn{x_c,e_k}=c_k\quad\forall k\in\N\]

    Es una isometría: \textbf{Identidad de Parseval}.

    \[||Tx||_{\ell^2}^2=||x||_H^2\]

    Identidad de Polarización:\\*
    $\K=\R:\inn{x,y}=\frac{1}{4}(||x+y||^2-||x-y||^2)$\\*
    $\K=\C:\inn{x,y}=\frac{1}{4}(||x+y||^2-||x-y||^2+i||x+iy||^1-i||x-iy||^2)$

    Por lo tanto, $T$ preserva el producto interno:

    \[\inn{Tx_1,Tx_2}_{\ell^2}=\inn{x_1,x_2}_H\]
\end{proof}

\section{Series de Fourier}

\subsection{Series de Fourier y convergencia}

$f:\R\to\C$ periódica de período $2\pi$.

$F:\T\to\C$, $\T$ es el círculo unitario.

\[F(e^{i\theta})=f(\theta)\]

\[\hookrightarrow \tilde f:[-\pi,\pi]\to\C\]

con

\[\tilde f(-\pi)=\tilde f(\pi)\]

Vamos a asumir que $\inn{f,g}_{L^2}:=\int_{-\pi}^\pi f(x)\overline{g(x)}\,dx$

\[f\in L^2(\T)=\left\{f:\R\to\C\text{ medibles, periódicas-$2\pi$ t.q.} \int_{-\pi}^\pi |f(x)|^2\,dx<\infty\right\}=L^2([-\pi,\pi])\]

Definimos 

\[e_n=\frac{1}{\sqrt{2\pi}}e^{inx}\quad n=0,\pm 1,\pm 2,\ldots\]

\begin{fproposition}
    $\{e_n\}$ es un conjunto ortonormal de $L^2(\T)$.
\end{fproposition}

\begin{proof}
    \begin{align*}
        \inn{e_n,e_m}&=\int_{-\pi}^\pi e_n(x)\overline{e_m(x)}\,dx\\
        &=\int_{-\pi}^\pi \frac{2}{\pi} e^{inx}e^{-imx}\,dx\\
        &=\frac{1}{2\pi}\int_{-\pi}^\pi e^{i(n-m)x}\,dx\\
        &=\begin{cases}
            \frac{2\pi}{2\pi}=1&n=m\\
            \left.\frac{e^{i(n-m)x}}{i(n-m)}\right|_{x=-\pi}^{x=\pi}&n\neq m
        \end{cases}
    \end{align*}
\end{proof}

\begin{fdefinition}
    Sea $f\in L^2(\T)$. Defina 
    
    \begin{align*}
        \hat f (n)&=\inn{f,e_n}_{L^2}\\
        &=\frac{1}{\sqrt{2\pi}}\int_{-\pi}^\pi f(x) e^{-inx}\,dx
    \end{align*}
    coeficiente de Fourier.

    \[f\to \sum_{n\in\Z}\hat f(n)e_n\]

    serie de Fourier.
\end{fdefinition}

\[S_N f(x)=\sum_{|n|\leq N} \hat f(n)\frac{1}{\sqrt{2\pi}} e^{inx}\]

suma de Fourier parcial.

Preguntas:

\begin{enumerate}
    \item ¿Converge $S_n f$ a $f$ en $L^2$?
    \item ¿Converge $S_N f(x)$ a $f(x)$ puntualmente?
    
    Si falla para algún $x$, ¿es este comportamiento raro o genérico?

    \item ¿Converge $S_N f$ a $f$ en otras normas (e.g. $L^p$,p>1)?
\end{enumerate}

\begin{ftheorem}
    $f\in L^2(\T)$, $S_N f\xrightarrow{L^2}f$ cuando $N\to\infty$.
\end{ftheorem}

\begin{fnote}
    El enunciado $\iff$,  $\mathcal{B}=\{e_n(x)\}_{n\in\Z}$ es una base o.n. para $L^2(\T)$
\end{fnote}

    Entonces será suficiente demostrar que $\mathcal{B}$ es maximal:
    \[\hat f(n)=0\quad\forall n\in\Z\implies f=0\]


\begin{ftheorem}
    $f\in L^2(\T)$. Entonces, 
    \[S_N f(x)=\int_{-\pi}^\pi D_N(x-t)f(t)\,dt\]
    donde 
    \[D_N(x)=\begin{cases}
        \frac{2N+1}{2\pi}& x=0\\
        \frac{\sin(N+\frac{1}{2})x}{2\pi\sin \frac{x}{2}}&x\neq 0
    \end{cases}\]
\end{ftheorem}

\begin{proof}
    \begin{align*}
        S_n f&=\sum_{|n|\leq N}\inn{f,e_n} e_n(x)\\
        &=\sum_{|n|\leq N} \frac{1}{2\pi}\left( \int_{\T} f(t)e^{-int}\,dt\right) e^{inx}\\
        &=\int_{\T} \underbrace{\left(\sum_{|n|\leq N} \frac{1}{2\pi} e^{in(x-t)}\right)}_{D_N(x-t)}f(t)\,dt
    \end{align*}

    donde 
    \[D_N(x)=\sum_{|n|\leq N} \frac{1}{2\pi} e^{inx}\]
    Kernel de Dirichlet.
    \[D_N(0)=\frac{2N+1}{2\pi}\]
    Para $x\neq 0$,

    \begin{align*}
        D_N(x)&=\frac{1}{2\pi} e^{-iNx} \sum_{n=0}^{2N} e^{inx}\\
        &=\frac{1}{2\pi} e^{-iNx} \frac{e^{i(2N+1)x}-1}{e^{ix}-1}\\
        &=\frac{1}{2\pi} \frac{e^{i(N+1)x}-e^{-iNx}}{e^{ix}-1}\\
        &=\frac{1}{2\pi}\frac{e^{i(N+\frac{1}{2})x}-e^{-i(N+\frac{1}{2})x}}{e^{ix/2}-e^{-ix/2}}\\
        &=\frac{1}{2\pi}\frac{2i \sin(N+\frac{1}{2})x}{2i \sin \frac{x}{2}}
    \end{align*}
\end{proof}

\begin{fnote}
    $D_N(x)$ es $2\pi$-periodico, par, suave y 

    \[\int_\T D_N(x)\,dx=1\]

    \seba{añadir foto del kernel de Dirichlet}

    Es difícil demostrar directamente que $S_N f(x)\to f(x)$ ($D_N(x)$ cambia de signo y oscila muy rápidamente). 
\end{fnote}

\paragraph{Desvío} En lugar de demostrar que $S_N f\xrightarrow{L^2}f$ directamente, vamos a considerar la sucesión \textbf{media de Cesàro}

\[\sigma_N f=\frac{S_0 f+S_1 f+\cdots + S_{N-1}f}{N}\]

\begin{fnote}
    $S_N f$ converge a $f$, $\sigma_N f$ converge a $f$
\end{fnote}

\begin{ftheorem}[Fejér]
    \[\sigma_N f\xrightarrow{L^2} f\]

    Cuando $f\in C(\T)$, 

    \[\sigma_N f\xrightarrow{\text{unif.}} f\text{ en }\T\]
\end{ftheorem}

Si $\hat f(n)=0\quad\forall n\in\Z$

\[\implies S_n f \equiv 0\quad \forall n\in\Z\implies \sigma_N f\equiv 0\]
\[\overset{\text{Fejer}}{\implies} f=\lim_{N\to\infty}\sigma_N f=0\implies \text{Maximalidad de }\mathcal{B}\]

\begin{fproposition}
    Sea $f\in L^2(\T)$. Entonces 

    \[\sigma_N f(x)=\int_{-\pi}^\pi F_N(x-t)f(t)\,dt\]

    donde 

    \[F_N(x)=\begin{cases}
        \frac{1}{2\pi}N & x=0\\
        \frac{1}{2\pi N} \frac{\sin^2(Nx/2)}{\sin^2 \frac{x}{2}}& x\neq 0
    \end{cases}\]

    es el Kernel de Fejér.
\end{fproposition}

\begin{proof}
    \begin{align*}
        \sigma_N f&=\frac{1}{N} \sum_{n=0}^{N-1} S_n f\\
        &\downarrow\\
        F_N(x)&=\frac{1}{N}\sum_{n=0}^{N-1} D_n(x)
    \end{align*}
    $x=0$
    \begin{align*}
        F_N(0)&=\dfrac{1}{N}\displaystyle\sum_{n=0}^{N-1} \overbrace{\frac{1}{2\pi} (2n+1)}^{D_n(0)}\\
        &=\frac{1}{2\pi} N
    \end{align*}
    $x\neq 0$,
    \begin{align*}
        F_N(x)&=\frac{1}{N}\cdot \frac{1}{2\pi} \sum_{n=0}^{N-1} \frac{\sin ((n+\frac{1}{2})x)}{\sin\frac{x}{2}}\\
        &=\frac{1}{2\pi N}\cdot \frac{1}{\sin^2 \frac{x}{2}}\sum_{n=0}^{N-1} \underbrace{\sin(n+\frac{1}{2})x\sin \frac{x}{2}}_{\frac{1}{2}\left(\cos(nx)-\cos((n+1)x)\right)}\\
        &=\frac{2}{\pi N}\frac{1}{\sin^2 \frac{x}{2}}\underbrace{\frac{1}{2}(\cos(0x)-\cos(Nx))}_{\sin^2 \frac{Nx}{2}}\\
        &=\frac{1}{2\pi N}\frac{\sin^2 \frac{Nx}{2}}{\sin^2\frac{x}{2}}
    \end{align*}
\end{proof}

Propiedades de $F_N(x)$

\begin{enumerate}
    \item $F_N(x)\geq 0$, suave, periódico-$2\pi$, par 
    \item \[\int_\T F_N(x)\,dx=1\]
    
    (como promedio de $D_N(x)$)

    \item \[|F_N(x)|\leq \frac{1}{2\pi N\sin^2 \frac{\delta}{2}}\quad \delta\leq |x|\leq \pi\]
\end{enumerate}

\seba{añadir foto pero borrarla pa zapit}

\paragraph{Notación}

\begin{align*}
    S_N f(x)&=\int_\T D_N(x-t)f(t)\,dt=D_N*f\\
    \sigma_N f(x)&=\int_\T F_N(x-t)f(t)\,dt=F_N*f
\end{align*}

Convolución: $f\in C(\T)$, $g\in L^1(\T)$

\[f*g(x)=\int_{-\pi}^\pi f(x-t)g(t)\,dt\]

tomando $\tau=x-t$

\[f*g(x)=\int_{x-\pi}^{x+\pi} f(\tau)g(x-\tau)\,d\tau=\int_{-\pi}^\pi f(\tau) g(x-\tau)=g*f(x)\]

\begin{fdefinition}
    $\{K_n\}_{n\in\N}$ es una familia de buenos kernels en $L^1(\T)$ si

    \begin{enumerate}
        \item \[\int_\T K_n(x)\,dx=1\]
        \item \[\sup_n \int_\T |K_n(x)|\,dx<\infty\]
        \item \[\int_{\delta\leq |x|\leq \pi} |K_n(x)|\,dx \xrightarrow{n\to\infty} 0\quad \forall \delta >0\]
    \end{enumerate}
\end{fdefinition}

\begin{fnote}
    $\{F_N(x)\}_{N\in\N}$ es una familia de buenos kernels pero $\{D_N\}$ \textbf{no} lo es. Veremos que 2. falla para el kernel de Dirichlet.
\end{fnote}

\begin{ftheorem}
    Si $\{K_N\}_{N\in\N}$ es una familia de buenos kernels en $L^1(\T)$ y $f\in C(\T)$, entonces 

    \[K_N*f=f*K_N\to f\]

    uniformemente en $\T$
\end{ftheorem}

\begin{fcorollary}
    \[\sigma_N f\xrightarrow[N\to\infty]{\text{unif}} f\text{ para }f\in C(\T)\]

\end{fcorollary}

\begin{proof}[Demsotración del teorema \ref{theo:3.5.6}]
    \begin{align*}
        K_n*f(x)-f(x)&=f*K_n(x)-f(x)\\
        &=\int f(x-y)K_n(y)\,dy-f(x)\\
        &=\int (f(x-y)-f(x))K_n(y)\,dy
    \end{align*}

    \begin{align*}
        \implies |K_n*f(x)-f(x)|&\leq \int_\T |f(x-y)-f(x)||K_n(y)|\,dy\\
        &=\int_{|y|<\delta} |f(x-y)-f(x)||K_n(y)|\,dy+\int_{|y|>\delta} |f(x-y)-f(x)||K_n(y)|\,dy\\
        &\leq \varepsilon\int_\mathbb{T} |K_n(y)|\,dy+2\max_\T |f|\int_{|y|>\delta}|K_n(y)|\,dy\\
        &\leq C\varepsilon
    \end{align*}
    cuando $n$ es suficientemente grande.
\end{proof}

\begin{fcorollary}
    Si $f\in C(\T)$ y $\hat f(n)=0\ \forall n\in\Z\implies f\equiv 0$.
\end{fcorollary}

\begin{proof}
    \begin{align*}
        &\sigma_N f\equiv 0\\
        &\downarrow \text{unif}\\
        &f\equiv 0
    \end{align*}
\end{proof}

\begin{fcorollary}
    Suponga que $f\in C(\T)$ y su serie de Fourier converge absoluta y uniformemente, es decir:

    \[\sum_n |\hat f(n) e_n(x)|=\sum_n |\hat f(n)|\frac{1}{\sqrt{2\pi}}<\infty\]

    Entonces, 

    \[S_N f\to f\text{ unif}\]
\end{fcorollary}

\begin{proof}
    Defina 

    \[g(x):=\sum_{n\in\Z}\hat f(n)e_n(x)\in C(\T)\]

    por convergencia absoluta uniforme.

    \[h(x):=g(x)-f(x)\]
    \begin{align*}
        \hat h(n)=\hat g(n)-\hat f(n)&=\inn{\sum_{k} \hat f(k)e_k(x),e_n(x)}-\hat f(n)\\
        &=\hat f(n)-\hat f(n)=0
    \end{align*}

    Se puede intercambiar la suma con la integral por convergencia uniforme y el corolario anterior, se concluye que $h\equiv 0$.
\end{proof}

Tenemos la convergencia $\sigma_N f\xrightarrow{\text{unif}} f$ para $f\in C(\T)$. Queremos pasar a convergencia en $L^2$. Vamos a utilizar la \textbf{densidad} de $C(\T)\subseteq L^2(\T)$. Vamos a necesitar la estimación adicional:

\begin{fproposition}
    \[||\sigma_N f||_{L^2}\leq ||f||_{L^2}\]
\end{fproposition}

\begin{proof}
    $\sigma_N f=\frac{1}{N}(S_0 f+\cdots+S_{N-1} f)$

    \begin{align*}
        ||\sigma_N f||_{L^2} \leq \frac{1}{N} \sum_{k=0}^{N-1} ||S_k f||_{L^2}
    \end{align*}

    Tenemos,

    \[||S_k f||_{L^2}\leq ||f||_{L^2}\text{ (Bessel)}\]

    $S_k f=$ proyección de $f$ en $\gen(\{e_l\}_{|l|\leq k})$

    \[||\sigma_N f||_{L^2}\leq \frac{1}{N}N||f||_{L^2}\]
\end{proof}

De hecho, tenemos 

\begin{fproposition}
    Si $f\in L^p(\T)$, $1\leq p<\infty$, entonces 

    \[||\sigma_N f||_{L^p}\leq ||f||_{L^p}\]
\end{fproposition}

\begin{ftheorem}
    Sea $f\in L^p(\T)$, $1\leq p<\infty$. Entonces,

    \[\sigma_N f\xrightarrow{L^p} f\]
\end{ftheorem}

\begin{proof}
    Fije $\varepsilon>0$. Aproxime $f\in L^p(\T)$ con $g\in C(\T)$:

    \[||f-g||_{L^p}\leq \varepsilon\]
    \begin{align*}
        \sigma_N f-f&=\sigma_N g-g +\sigma_N (f-g)-(f-g)\\
        ||\sigma_N f-f||_{L^p}&\leq ||\sigma_N g-g||_{L^p}+||\sigma_N (f-g)||_{L^p}+||f-g||_{L^p}\\
        &\leq C\varepsilon
    \end{align*}

    Podemos elegir $N$ suficientemente gtande, tal que 

    \[||\underbrace{\sigma_N g-g}_{h}||_{\infty}\leq \varepsilon\]

    por convergencia uniforme.

    \begin{align*}
        ||h||_{L^p}&=\left(\int_\T |h|^p\,dx\right)^{1/p}\\
        &\leq \left(\int_\T \varepsilon^p\,dx\right)^{1/p}=(2\pi)^{1/p}\varepsilon
    \end{align*}
\end{proof}

\begin{fcorollary}
    \[S_N f\xrightarrow{L^2} f\]
\end{fcorollary}

\begin{proof}
    \[\sigma_N f\xrightarrow{L^2}f\]
\end{proof}

\begin{flemma}[Riemann-Lebesgue]
    $f\in L^1(\T)$, $\hat f(n)=\frac{1}{2\pi}\int_\T f(x)e^{-inx}\,dx\xrightarrow{n\to\infty} 0$
\end{flemma}

\begin{proof}
    Fije $\varepsilon>0$. Utilizaremos que 

    \[\sigma_N f\xrightarrow{L^1}f\]

    Podemos encontrar $N$ suficientemente grande, tal que

    \[||\underbrace{f-\sigma_N f}_{g}||_{L^1}\leq \varepsilon\]

    $n>N$,

    \begin{align*}
        \hat g(n)&=\hat f(n)-\cancelto{0}{\widehat{\sigma_N f}(n)}\\
        \implies |\hat f(n)|&=|\hat g(n)|\leq \frac{1}{\sqrt{2\pi}}\int |g(x)e^{-inx}|\,dx=\frac{1}{\sqrt{2\pi}}\int |g|\,dx\leq \varepsilon/\sqrt{\pi}
    \end{align*}
\end{proof}

\begin{align*}
    L^2(\T)&\to \ell_\Z^2=\{(\ldots,a_{-1},a_0,a_1,\ldots):\sum_{k\in\Z}|a_k|^2<\infty\}\\
    f&\to \hat f=(\ldots,\hat f_{(-1)},\hat f_{(0)},\hat f_{(1)},\ldots)
\end{align*}

es un isomorfismo unitario.

\begin{align*}
    L^1(\T)&\xrightarrow{\mathcal{F}} \hat c_0=\{(\ldots,a_{-1},a_0,a_1,\ldots):\lim_{|n|\to\infty} a_n=0\}\\
    f&\to \hat f
\end{align*}

\begin{ftheorem}
    $L^1(\T)\xrightarrow{\mathcal{F}} \hat c_0$ es lineal, acotado e inyectivo.
\end{ftheorem}

\begin{proof}
    lineal \checkmark

    \begin{align*}
        ||\hat f||_{\ell^\infty}&\leq ?\\
        |\hat f(n)|&\leq \frac{1}{\sqrt{2\pi}}\int |f(x)e^{-inx}|\,dx\\
        &\leq \frac{1}{\sqrt{2\pi}}||f||_{L^1}
    \end{align*}

    por lo que $||\hat f||_{\ell^\infty}\leq \frac{1}{\sqrt{2\pi}} ||f||_{L^1}$

    $\rightarrow$ inyectivo? Suponga que $\hat f=0\iff \hat f(n)=0\quad\forall n\in\Z$

    \begin{align*}
        &\sigma_N f\equiv 0\\
        &\downarrow L^1\\
        &f\equiv 0
    \end{align*}

    pero $\mathcal{F}$ no es sobreyectiva. Si $\F$ fuera inyectivo, sería un isomorfismo continuo. Por teorema de aplicación abiert, tenemos que $\F^{-1}$ es acotada:

    \begin{align*}
        ||\F^{-1}\hat f||_{L^1}\leq c||\hat f||_{\infty}\\
        ||f||_{L^1}\leq c||\hat f||_{\infty}
    \end{align*}

    Tomamos $f(x)=D_N(x)=\frac{1}{\sqrt{2\pi}}\sum_{|n|\leq N}\frac{1}{\sqrt{2\pi}}e^{inx}=\frac{1}{\sqrt{2\pi}}\sum_{|n|\leq N}e_n(x)$.

    \begin{align*}
        \hat f(n)&=\inn{f,e_n}\\
        &=\frac{1}{\sqrt{2\pi}} \inn{e_n,e_n}\quad |n|\leq N\\
        &=0\quad |n|>N
    \end{align*}

    $||\hat f||_\infty=\frac{1}{\sqrt{2\pi}}$

    \begin{fproposition}
        \[||D_N||_{L^1}\geq C\log N\]
    \end{fproposition}

    \begin{fcorollary}
        $f_N:=D_N$ contradice $||f||_{L^1}\leq c||\hat f||_\infty$
    \end{fcorollary}

    \[D_N(x)=\frac{1}{2\pi}\frac{\sin(N+\frac{1}{2})x}{\sin \frac{x}{2}}\]
    \[||D_N||=\int_{-\infty}^\infty |D_N(x)|\,dx=\frac{1}{2\pi}\int_0^\infty \frac{|\sin(N+\frac{1}{2})|}{\sin \frac{x}{2}}\]
    \[||D_N||\geq \frac{2}{\pi}\int_0^\pi \frac{\sin(N+\frac{1}{2})x}{x}\,dx\]

    $u=(N+\frac{1}{2})x$

    \[=\frac{2}{3\pi}\int_0^{(N+\frac{1}{2})\pi} \frac{|\sin u|}{u}\,du\geq \frac{2}{\pi}\int_0^{N\pi}\frac{|\sin u|}{u}\,du\]
    \[=\frac{2}{\pi}\sum_{k=1}^N \int_{(k-1)\pi}^{k\pi} \frac{|\sin u|}{u}\,du\]
    \[\geq \frac{2}{\pi}\sum_{k=1}^N \frac{1}{k}\int_{(k-1)\pi}^{k\pi} |\sin u|\,du\]
    \[=\frac{2}{\pi}\sum_{k=1}^N \frac{1}{k}\underbrace{\int_0^\pi |\sin u|\,dy}_{c'}\]
    \[=\frac{2c'}{\pi}\sum_{k=1}^N \frac{1}{k}\geq c\log N\]
    \seba{añadir align}
\end{proof}

Vimos que $\forall f\in L^2(\T)$, $S_Nf\xrightarrow{L^2}f$. 

Q. ¿Converge $S_nf\to f$ puntualmente?\\
A. ¡Generalmente no!

Q. ¿Converge $S_N f\to f$ c.t.p? Es fácil ver (si conocemos teoría de integración) que existe una subsucesión 

\[S_{N_k}f\to f\quad \text{c.t.p}\]

(dada la convergencia $S_N f\xrightarrow{L^2}f$)\\
A. (Teorema de Carleson) Sí, $S_nf\to f\quad \text{c.t.p}$ (Difícil).

\subsection{Convergencia puntual de la serie de Fourier}

Vieron en ayudantía un ejemplo de función $f\in C(\T)$ tal que 

\[S_N f(0)\not\to f(0)\]

De hecho, este ejemplo es \textbf{genérico}

\begin{ftheorem}
    Para todo $x\in \T$, existe un conjunto genérico $A_x\subseteq C(\T)$ tal que 

    \[\sup_N |S_N f(x)|=\infty\]
\end{ftheorem}

La demostración utiliza el marco del \textbf{principio de acotación uniforme}/Teorema de Banach-Steinhaus

\begin{ftheorem}[Banach-Steinhaus]
    Sea $X$ Banach, $Y$ un espacio normado. Sean $T_k\in \mathcal{B}(X,Y)$, $k\in I$, no necesariamente contable. Entonces 

    \begin{enumerate}
        \item o $\sup_k ||T_k||<\infty$
        \item o $\sup_k ||T_k x||=\infty$ para todo $x\in A$, donde $A\subseteq X$ es un subconjunto genérico $G_\delta$.
    \end{enumerate}

    \seba{cambiar enumerate a letras a., b.}
\end{ftheorem}

\begin{fnote}
    Si $\sup_k ||T_k x||<\infty$ $\forall x\in X$, entonces $||T_k||$ son uniformemente acotadas.
\end{fnote}

\begin{fcorollary}
    Sean $X$ Banach, $Y$ normado. Sean $T_k\in\mathcal{B}(X,Y)$. Suponga que $\forall x\in X$

    \[\lim_{k\to\infty} T_k x\eqqcolon Tx\quad \text{existe}\]

    Entonces, $T\in \mathcal{B}(X,Y)$ y 

    \[||T||\leq \liminf_{k\to\infty} ||T_k||<\infty\]
\end{fcorollary}

\begin{proof}
    $\lim_{k\to\infty} T_k x=Tx$.

    \[\implies \forall x\in X\quad \sup_k ||T_k x||<\infty\]

    (sucesión que converge es acotada)

    \[\implies \sup_k ||T_k||<\infty\]

    Que $T$ es lineal, fácil $\checkmark$

    \[||Tx||=||\lim_{k\to\infty}T_k x||=\ im_{k\to\infty} ||T_k x||\]
    \[=\sup_n\inf_{k\geq n}||T_k x||\leq (\sup_n\inf_{k\geq n} ||T_k||)x=(\liminf_{k\to\infty} ||T_k||)||x||\]

    \seba{añadir align}
\end{proof}

\begin{proof}[Demostración del teorema de Banach-Steinhaus (\ref{theo:3.5.14})]
    Defina $\psi(x):=\sup_k ||T_k x||$.
    \[U_n=\{x\in X:\psi(x)>n\}=\bigcup_{k} \underbrace{\{||T_k x|| > n\}}_{\text{abierto pues $T_k$ es continuo}}\]

    Tenemos 2 posibilidades:

    \begin{enumerate}
        \item Si todos los $U_n$'s son densos en $X$,
        \[\implies A\coloneqq \bigcap_{n=1}^\infty U_n \quad \text{es genérico},G_\delta\]

        $\forall x\in A$, $\psi(x)>n\quad \forall n\in\N$
        \[\implies \psi(x)=\infty\quad \text{(caso \emph{b.})}\]

        \item Si unos de los $U_n$'s \textbf{no} es denso, entonces $U_m^c$ contiene una bola $B=B_r(a)$.
        
        \begin{align*}
            \psi(x)\leq m\quad \forall x\in B_r(a)\\
            \implies ||T_k x||\leq m\quad \forall x\in B_r(a),\forall k\\
            \implies ||T_k(a+y)||\leq m\quad \forall y\in B_r(0),\forall k\\
        \end{align*}

        $\forall y\in B_r(0)$

        \begin{align*}
            ||T_k y||&\leq ||T_k a||+||T_k(y-a)||\\
            &=||T_k a||+||T_k(a-y)||\\
            &\leq m + m = 2m
        \end{align*}

        \[\implies ||T_k y||\leq \frac{2m}{r}||y||\quad \forall y\in X,\forall k\]
    \end{enumerate}
\end{proof}

\begin{proof}[Demostración del teorema \ref{theo:3.5.13}]
    Será suficiente demostrar el teorema para $x=0$. Aplicaremos el principio de acotación uniforme (Banach-Steinhaus) a 

    \begin{align*}
        S_N^0:C(\T)&\to\C\\
        f&\to S_n f(0)
    \end{align*}

    Estaremos listos cuando probemos que 

    \[\sup_N ||S_N^0||=\infty\]
    $\iff$ estamos en la alternativa \emph{b.}
    \[\implies \sup_N |S_N f(0)|=\infty\forall f\in A, A\overset{gen.}{\subseteq} C(\T)\]

    Recordando que $(S_N f(x)=D_N*f(x))$

    \begin{align*}
        S_N^{(0)}&=S_N f(0)\\
        &=\int_{-\pi}^\pi D_N(0-y)f(y)\,dy\\
        &=\int_{-\pi}^\pi D_N(y)f(y)\,dy\\
        \implies |S_N f(0)|\leq \int_{-\pi}^\pi |D_n(y)|\cdot|f(y)|\,dy&\leq ||D_N||_{L^1}||f||_{\infty}\\
        \implies ||S_N^0||\leq ||D_N||_{L^1}
    \end{align*}

    Pero, de hecho, afirmamos que 

    \[||S_N^0||=||D_N||_{L^1}\]

    Noten que cuando ponemos $f(y)=\sgn D_n(y)$

    \[S_N f(0)=\int_{-\pi}^\pi D_N(y)\sgn D_N(y)\,dy=||D_N||_{L^1}\]

    $f=\sgn D_N\in L^1(\T)$, $\implies$ podemos encontrar $f_k\in C(\T)$:

    \[||f_k-f||_{L^1}\xrightarrow{k\to\infty}0\]

    \begin{align*}
        S_N f_k(0)&=\int D_N(y)(f_k-f)(y)\,dy+\underbrace{\int D_N(y)f(y)\,dy}_{||D_N||_{L^1}}\xrightarrow{k\to\infty} ||D_N||_{L^1}
    \end{align*}

    mientras 

    \[\left|\int D_N(y)(f_k-f)(y)\,dy\right|\leq \max_\T |D_N|||f_k-f||_{L^1}\xrightarrow{k\to\infty}0\]
\end{proof}

\section{Repaso/Crash course en teoría de la medida}

$\Omega = \{M_1,\ldots,M_{100}\}$ pila de monedas. $V:\Omega\to [0,\infty)$. En el caso de Chile tenemos $Im V=\{1,5,10,50,100,500\}$. Queremos calcular el valor total de la pila de monedas. Hay 2 métodos:

\begin{enumerate}
    \item Dividimos la pila en grupos de digamos 10 monedas: $M_1,\ldots,M_{10}$ y así sucesivamente. Luego, sumamos los valores de cada grupo y sumamos los resultados. Esto corresponde con la integral de Riemann
    \item Dividimos las monedas en grupos de acuerdo al valor 
    \[E_1=\{M\in\Omega:V(M)=\alpha_1\}\]
    \[E_2=\{M\in\Omega: V(m)=\alpha_2\}\]
    \[\vdots\]
    \[E_N\]

    Luego, $S=\displaystyle\sum_{k=1}^N \alpha_k (\# E_k)$. Esto corresponde con la integral de Lebesgue.
\end{enumerate}

\subsection{Espacios de medida y funciones medibles}

\begin{fdefinition}[Espacio de medida]
    un espacio de medida $(\Omega, \mathcal{M}, \mu)$.
\end{fdefinition}

\begin{fdefinition}[$\sigma$-álgebra]
    Una colección $\mathcal{M}$ de subconjuntos de $\Omega$ es una $\sigma$-álgebra si

    \begin{enumerate}
        \item $\Omega\in \mathcal{M}$
        \item $E\in \mathcal{M}\implies E^c\coloneqq \Omega\setminus E\in \mathcal{M}$
        \item $\{E_k\}_{k=1}^\infty\subseteq \mathcal{M}\implies \bigcup_{k=1}^\infty E_k\in\mathcal{M}$
    \end{enumerate}
\end{fdefinition}

Podemos ver que $\varnothing\in \mathcal{M}$, $\bigcap_{k} E_k\in\mathcal{M}$ si $\forall E_k\in\mathcal{M}$ y $E\setminus F\in\mathcal{M}$ si $E,F\in\mathcal{M}$.

\begin{fexample}
    $\Omega=\{a,b\}$,

    \[\mathcal{M}_1=\{\varnothing,\Omega\}\]
    \[\mathcal{M}_2=\{\varnothing, \{a\},\{b\},\{a,b\}\}\]

    son $\sigma$-álgebras.
\end{fexample}

\begin{fexample}
    Si $\Omega$ es un espacio métrico (topológico más general).

    \[\mathcal{B}_\Omega\to \sigma\text{-álgebra de Borel}\]

    definida como la menor $\sigma$-álgebra que contiene todos los abiertos de $\Omega$.
\end{fexample}

\begin{fdefinition}
    Una medida $\mu:\mathcal{M}\to[0,\infty]$ que satisface:

    \begin{enumerate}
        \item $\mu(\varnothing)=0$
        \item $\{E_k\}_{k=1}^\infty$ de conjuntos en $\mathcal{M}$ mutuamente disjuntos, 
        \[\mu\left(\bigcup_{k=1}^\infty E_k\right)=\sum_{k=1}^\infty \mu(E_m)\]

        Esto se llama $\sigma$-aditividad
    \end{enumerate}
\end{fdefinition}

Las siguientes propiedades son consecuencias fáciles de la definición:

\begin{enumerate}
    \item (Aditividad finita) 
    \[\mu\left(\bigcup_{k=1}^N E_k\right)=\sum_{k=1}^N \mu(E_k)\]

    \item Si $A,B\in\mathcal{M}$ y $A\subseteq B$ 
    \[\implies \mu(B)=\mu(A\cup B\setminus A)=\mu(A)+\mu(B\setminus A)\geq \mu(A)\]

    \item (subaditividad) Si $\{E_k\}\subseteq \mathcal{M}$, no necesariamente disjuntos,
    
    \[\mu\left(\bigcup_{k=1}^\infty E_k\right)\leq \sum_{k=1}^\infty \mu(E_k)\]

    \item $E_1\subseteq E_2\subseteq \cdots\subseteq E_{k+1}\subseteq \cdots$, sucesión creciente de medibles, 
    
    \[E=\bigcup_k E_k, E_k\uparrow E\]
    \[\mu(E)=\lim_{k\to\infty} \mu(E_k)\]

    \item $E_1\supseteq E_2\supseteq \cdots$
    \[E=\bigcap_{k=1}^\infty E_k, E_k\downarrow E\]
    \[\mu(E)=\lim_{n\to\infty} \mu(E_k)\]

    si $\mu(E_1)<\infty$
\end{enumerate}

\begin{fexample}
    $(\N,\mathcal{P}(\N),\mu)$,

    \[\mu(E)=\sum_{n\in E}\mu_n \leftarrow \text{pesos}\in [0,\infty)\]

    Cuando todos los $\mu_n\equiv 1$, $\mu$ es la medida de contar.
\end{fexample}

\begin{ftheorem}
    Existe una $\sigma$-álgebra $\mathcal{M}$ de subconjuntos de $\R^n$ y 

    \[|\cdot|:\mathcal{M}\to[0,\infty]\]

    con las siguientes propiedades:

    \begin{enumerate}
        \item $\mathcal{M}$ contiene todos los abiertos ($\supseteq B_{\R^n}$)
        \item $|B|=\text{Vol} (B)$ para toda la bola abierta $B\subseteq \R^n$
        \item (completitud) Si $A\subseteq B$, donde $B\in\mathcal{M}$ y $|B|=0$, entonces $A\in\mathcal{M}$ y $|A|=0$.
    \end{enumerate}
\end{ftheorem}

Conjuntos de medida 0 $=$ conjuntos \textbf{despreciables}.

\paragraph{Notación:} Una propiedad se cumple para x c.t.p (en casi todas partes) si se cumple para todo $x\in E^c$ donde $E$ es despreciable.

\paragraph{La medida producto}: $(\Omega_1,\mathcal{M}_1,\mu_1)$ y $\Omega_2,\mathcal{M}_2,\mu_2$. Existe una única medida $\mu:\mathcal{M}_1\times \mathcal{M}_2\to[0,\infty]$ donde $\mathcal{M}_1\times \mathcal{M}_2\coloneqq$ la menor $\sigma$-álgebra que contiene todos los $E_1\times E_2$ con $E_1\in\mathcal{M},E_2\in\mathcal{M}_2$ tal que 
\[\mu(E_1\times E_2)=\mu_1(E_1)\mu_2(E_2)\]

\begin{fexample}
    $(\R^n,\mathcal{B}_{\R^n},|\cdot|=\lambda_n)$ y $(\R^m, \mathcal{B}_{\R^m},|\cdot|=\lambda_m)$

    \[\mathcal{B}_{\R^n}\times \mathcal{B}_{\R^m}=\mathcal{B}_{\R^n\times \R^m}\]
    \[\lambda\coloneqq \lambda_n\times \lambda_m=\lambda_{m+n}\]

    que es la medida de Lebesgue en $\R^{m+n}$.
\end{fexample}

\begin{fdefinition}
    $f:\Omega\to [-\infty,\infty]$ es \textbf{medible} si $\{x\in \Omega:f(x)>r\}\in\mathcal{M}\quad \forall r\in[-\infty,\infty]$
    \[\iff f^{-1}(I)\in \mathcal{M}\quad\forall I\subseteq [-\infty,\infty]\]
    \[\iff f^{-1}(O)\in\mathcal{M}\quad\forall O\overset{ab}{\subseteq}[-\infty,\infty]\]
\end{fdefinition}

Esta clase de funciones con valores reales es cerrada bajo las operaciones usuales: $+,\times$ y tomar $\sup_k$,$\inf_k$,$\limsup_k$,$\liminf_k$. Si $\{f_k\}_{k=1}^\infty$ sucesión de funciones medibles, entonces $\sup_k f_k$, $\inf_k f_k$, $\liminf_k f_k$, $\limsup_k f_k$ son medibles.

\begin{fexample}
    La funciones simples 

    \[s(x)=\sum_{i=1}^n\alpha_i\chi_{E_i}\]

    De hecho toda función medible es límite de funciones simples $s_n(x)$ tal que 

    \[|s_n(x)|\nearrow|f(x)|\]

    Descomponemos $f=f^+-f^-$,

    \[0\leq f^+=\max\{f,0\}\]
    \[0\leq f^-=\max\{-f,0\}\]

    $|f|=f^+ + f^-$.

    Cuando $f\geq 0$ es medible, podemos aproximarla con 

    \[s_n(x)=n\chi_{\{f>n\}}+\sum_{k=1}^{n2^n}\frac{k-1}{2^n}\chi_{\{\frac{k-1}{2^n}\leq f<\frac{k}{2^n}\}}\]

    $s_n(x)\nearrow f(x),\quad n\to\infty$
\end{fexample}

\subsection{La integral de Lebesgue}

\paragraph{Funciones simples}

\[\int_\Omega s\,d\mu=\sum_{i=1}^n \alpha_i \mu(E_i)\]

donde $s(x)=\sum_{i=1}^n \alpha_i\chi_{E_i}$

\paragraph{Funciones medibles}

\[\int f\,d\mu\coloneqq \sup \{\int s\,d\mu:0\leq s\leq f\}\]

\subsubsection*{Propiedades}

\begin{enumerate}
    \item \[\int (f+g)\,d\mu=\int f\,d\mu+\int g\,d\mu\]
    \item \[\int cf\,d\mu=c\int f\,d\mu,\quad c\geq 0\]
    \item \[\int f\,d\mu=0\iff f\equiv 0\quad \text{c.t.p.}\]
    \item \[\int f\,d\mu<\infty\implies f<\infty \quad\text{c.t.p.}\]
\end{enumerate}

\subsubsection*{Propiedades de convergencia}

\begin{enumerate}
    \item Teorema de Convergencia Monotona:
    
    \[0\leq f_n\nearrow f \quad\text{c.t.p.} \implies \int f\,d\mu=\lim_{n\to\infty}\int f_n\,d\mu\]

    \item Lema de Fatou:
    
    \[f_n\geq 0\qquad \int \liminf_{n} f_n\,d\mu\leq \liminf_n \int f_n\]
\end{enumerate}

\paragraph{Funciones reales}

$f:\Omega\to [-\infty,\infty]$, $f=f^+ - f^-$

\[\int f\,d\mu\coloneqq \int f^+\,d\mu-\int f^-\,d\mu\]

si uno de estos dos términos $<\infty$.

Cuando ambos son finitos,

\[\iff \int |f|\,d\mu=\int f^+\,d\mu+\int f^-\,d\mu<\infty\]

decimos que $f\in \mathcal{L}_\R^1(\mu)$ es integrable.

En $\mathcal{L}^1(\mu)$, la integral es un funcional lineal (POS) que es $\geq 0$ cuando $f\geq 0$. Como consecuencia, para $f,g\in\mathcal{L}^1$:

\begin{enumerate}
    \item \[\int f\,d\mu\leq \int g\,d\mu\]

    cuando $f\geq g$ c.t.p.

    \item \[\left|\int f\,d\mu\right|\leq \int |f|\,d\mu\]
    \item \[|f|<\infty\quad \text{c.t.p.}\]
\end{enumerate}

\paragraph{Funciones complejas}\hfill

\begin{fdefinition}
    $f:\Omega\to\C$ es medible si $u\coloneqq \Re f$, $v\coloneqq \Im f$ son medibles.
\end{fdefinition}

\[\int f\,d\mu\coloneqq \int u\,d\mu + i\int v\,d\mu\]

\begin{fdefinition}
    $f\in \mathcal{L}_\C^1(\mu)$ si $|f|$ es integrable $\iff u,v\in \mathcal{L}_\R^1(\mu)$.
\end{fdefinition}

\begin{ftheorem}[Convergencia Dominada]
    $f_n\to f$ c.t.p. y $|f_n|\leq g$ c.t.p. donde $g\in \mathcal{L}_\R^1(\mu)$, entonces 

    \[\int f_n\,d\mu\to \int f\,d\mu\]

    De hecho, 

    \[\int |f_n-f|\,d\mu\xrightarrow{n\to\infty}0\]
\end{ftheorem}

\begin{ftheorem}[Tonelli]
    $(\Omega_1,\mathcal{M}_2,\mu_1)$, $(\Omega_2,\mathcal{M}_2,\mu_2)$ espacio de medida $\sigma-$finitos. $(\Omega_1\times\Omega_2,\mathcal{M}_1\times\mathcal{M}_2,\mu=\mu_1\times\mu_2)$. Sea $f(x,y)$ $\mathcal{M}_1\times\mathcal{M}_2-$medible y no negativa. Entonces, denotando $f^y(x)=f(x,y)$ para $y$ fijo es una función en $\Omega_1$, $f_x(y)=f(x,y)$ para $x$ fijo es una función en $\Omega_2$,

    \begin{align*}
        \int_{\Omega_1\times\Omega_2}f\,d\mu&=\int_{\Omega_2}\left(\int_{\Omega_1} f^y(x)\,d\mu_1(x)\right)d\mu(y)\\
        &=\int_{\Omega_1}\left(\int_{\Omega_2} f_x(y)\,d\mu_2(y)\right)d\mu_1(x)
    \end{align*}

    donde toda función integrada es medible en el espacio correspondiente.

\end{ftheorem}

\begin{ftheorem}[Fubini]
    Es posible cambiar el orden de integración cuando $f\in \mathcal{L}_\C^1(\mu)$:

    \begin{align*}
        \int_{\Omega_1\times\Omega_2}f\,d\mu&=\int_{\Omega_2}\left(\int_{\Omega_1} f^y(x)\,d\mu_1(x)\right)d\mu(y)\\
        &=\int_{\Omega_1}\left(\int_{\Omega_2} f_x(y)\,d\mu_2(y)\right)d\mu_1(x)
    \end{align*}

\end{ftheorem}

\section{Espacios de Lebesgue $L^p$}

\subsection{Espacios $L^p$}

Defina: 

\[||f||_p\coloneqq \left(\int |f|^p\,d\mu\right)^{1/p},\quad p\in[1,\infty)\]
\[||f||_\infty\coloneqq \inf\{M>0: |f|\leq M\quad \text{c.t.p.}\}\]

\[\mathcal{L}_\K^p(\mu)\coloneqq \{\text{funciones medibles}:\Omega\to\K:||f||_p<\infty\}\]

\begin{fproposition}\label{theo:3.7.1}
    $||\cdot||_p$ es una \textbf{semi-norma} en $L_\K^p(\mu)$. Además, 

    \[||f||_p=0\iff f=0\quad\text{c.t.p.}\]
\end{fproposition}

\begin{fcorollary}
    $\mathcal{M}_\K(\mu)=\{f=0\quad \text{c.t.p.}\}$. Entonces, 

    \[L_\K^p(\mu)\coloneqq \mathcal{L}_\K^p(\mu)/\mathcal{N}_\K(\mu)\]

    es un espacio \textbf{normado} con norma $||\cdot||_p$.
\end{fcorollary}

\begin{proof}[Demostración de la proposición \ref{theo:3.7.1}]
    $||\lambda f||_p=|\lambda|\cdot ||f||_p$.

    Desigualdad triangular = desigualdad de Minkowski

    \[||f+g||_p\leq ||f||_p+||g||_p\]

    ($p=1,\infty$ es obvio)
\end{proof}

\begin{ftheorem}[Desigualdad de Hölder]

    \[\int |fg|\,d\mu\leq ||f||_p ||g||_q\]

    donde

    \[\frac{1}{p}+\frac{1}{q}=1\]

    y $p,q\in [1,\infty]$
    
\end{ftheorem}

\begin{proof}
    Podemos asumir que

    \[0<||f||_p,||g||_q<\infty\]

    $p=1,q=\infty$. $0<||f||_1<\infty$, $0<||g||_\infty<\infty$.

    \begin{align*}
        \int |fg|\,d\mu&\leq \int(|f|\,d\mu)||g||_\infty\\
        &\leq |f|\cdot ||g||_\infty \quad \text{c.t.p.}=||f||_1\cdot ||g||_\infty
    \end{align*}

    Para los demás, $1<p,q<\infty$. Podemos asumir $||f||_p=1$, $||g||_q=1$ y será suficiente demostrar 

    \[\int |fg|\,d\mu\leq 1\]

    Aplicamos Young (lo que viene después) a $a=|f|$, $b=|g|$

    \[|fg|\leq \frac{|f|^p}{p}+\frac{|g|^q}{q}\]

    \[\int |fg|\,d\mu\leq \frac{1}{p}\underbrace{\int |f|^p}_{=1}+\frac{1}{q}\underbrace{\int |g|^q}_{=1}\]
\end{proof}

\begin{ftheorem}[Desigualdad de Young]
    $0\leq a,b\leq \infty$

    \[ab\leq \frac{a^p}{p}+\frac{b^q}{q}\quad 1<p,q<\infty\]
\end{ftheorem}

\begin{proof}
    Podemos asumir $a,b>0$.

    \[ab=e^{\log (ab)}=e^{\log a+\log b}=e^{\frac{1}{p}\log (a^p)+\frac{1}{q}\log (b^q)}\]

    $e^{sx+(1-s)y}\leq se^{x}+(1-s)e^y$ (convexidad de $e^x$)

    por lo que 

    \[\leq \frac{1}{p}e^{\log (a^p)}+\frac{1}{q}e^{\log(b^q)}\]
\end{proof}

\begin{proof}[Desigualdad de Minkowski]
    en $1<p<\infty$ 

    \begin{align*}
        |f+g|^p&=|f+g|\cdot |f+g|^{p-1}\\
        &\leq |f|\cdot |f+g|^{p-1}+|g|\cdot |f+g|^{p-1}\\
        \int|f+g|^p\,d\mu&\leq \int |f|\cdot |f+g|^{p-1}+\int |g|\cdot |f+g|^{p-1}\\
    \end{align*}
\end{proof}

\begin{ftheorem}[Riesz-Fischer]
    $L^p(\mu)$ es un espacio de Banach.
\end{ftheorem}

\begin{proof}
    $f_k\in L^p(\mu)$, $k\in\N$. Queremos demostrar que si 

    \[\sum_{k=1}^\infty ||f_k||_p\eqqcolon M<\infty\]
    \[\implies \sum_{k=1}^n f_k\xrightarrow{L^p}F\in L^p\]

    $p=\infty$ (ejercicio). Sea $p\in[1,\infty)$

    \[G_n(x)\coloneqq \sum_{k=1}^n |f_k(x)|\quad \text{medible}, \geq 0\nearrow_{n\to\infty} G(x)\coloneqq \sum_{k=1}^\infty |f_k(x)|\]

    Por teorema de convergencia monotona,

    \[\int G(x)^p\,d\mu=\lim_{n\to\infty}\int G_n(x)^p\,d\mu\]

    \[\left(\int G_n(x)^p\,d\mu\right)=||G_n||_p\leq \sum_{k=1}^n ||f_K||_p\leq M\]

    por Minkowski.

    \[\implies \int G_n(x)^p\,d\mu\leq M^p\]
    \[\implies G^p\in L^1\quad (G\in L^p)\]

    En particular, $0\leq G^p(x)<\infty\quad \mu-\text{c.t.p.}$

    \[\implies G(x)<\infty\quad \mu-\text{c.t.p.}\]

    es decir, $\mu-\text{c.t.p.}$, $\sum |f_k(x)|$ converge. Defina
    \[F(x)=\begin{cases}
        \sum_{k=1}^\infty f_k(x) &x \text{ tal que } G(x)<\infty\\
        0&\text{e.o.c.}
    \end{cases}\]

    $F$ es medible y $F\in L^p(\mu)$ pues

    \[|F(x)|\leq \sum_{k=1}^\infty |f_k(x)|=G(x)\quad \mu-\text{c.t.p.}\]

    \[\implies |F(x)|^p\leq G(x)^p\quad \mu-\text{c.t.p.}\]
    \[\implies \int |F(x)|^p\leq \int G(x)^p<\infty\]

    Falta establecer la convergencia en $L^p$:

    \[\left|\left|F-\sum_{k=1}^N f_k\right|\right|_p\xrightarrow{N\to\infty}0\]

    \[\left|F-\sum_{k=1}^n f_k\right|(x)\leq \sum_{k=N+1}^\infty |f_k|(x)\leq G(x)\quad\mu-\text{c.t.p.}\]
    \[\implies \left|F-\sum_{k=1}^N f_k\right|^p\leq G^p\quad \mu-\text{c.t.p.}\]

    Por definición de $F$, 
    
    \[|F-\sum_{k=1}^N f_k|\xrightarrow{N\to\infty}0\quad\mu-\text{c.t.p.}\]
    
    Por el teorema de convergencia dominada 

    \[\int \left|F-\sum_{k=1}^N f_k\right|^p\,d\mu\xrightarrow{N\to\infty}\int 0\,d\mu=0\]

\end{proof}

\subsection{Los espacios $L^p$ y dualidad}

$1\leq p\leq\infty$, $q\rightarrow$ exponente dual: $\frac{1}{p}+\frac{1}{q}=1$
    
$p=1\to q=\infty$\\
$p=2\to q=2$\\
$p=\infty\to q=1$

Se puede definir un \textbf{emparejamiento} entre $L^p$ y $L^q$.

\begin{align*}
    \inn{\cdot,\cdot}:L^p(\mu)\times L^q(\mu)&\to\K\\
    f,g&\to \inn{f,g}\coloneqq \int fg\,d\mu
\end{align*}

es bien definido:

\[f\in L^p,g\in L^q\implies |fg|\in L^1\]

\[\left(\int |fg|\,d\mu\right)\leq ||f||_p||g||_q<\infty\]
\[\implies fg\in L^1\]
\[\implies |\inn{f,g}|=\left|\int fg\right|\leq \int |fg|\leq ||f||_p||g||_q\]

Debido a esto podemos definir $\ell_g\in (L^p)^*$

\begin{align*}
   \ell_g:L^p(\mu)&\to\K\\
   f&\to \inn{f,g}
\end{align*}

es lineal y acotado con 

\[||\ell_g||_{(L^p)^*}\leq ||g||_q\]

De esta manera tenemos una aplicación

\begin{align*}
    \phi:L^q(\mu)&\to [L^p(\mu)]^*\\
    g&\to\ell_g
\end{align*}

$\phi$ es lineal, acotada e inyectiva $(\ell_g=0\implies g=0)$.

\subsection{Teorema de Representación de Riesz}

\begin{ftheorem}
    Sea $(\Omega,\mathcal{M},\mu)$ $\sigma-$finito. Sea $1\leq p<\infty$

Entonces $\phi$ es un \textbf{isomorfismo isométrico}:

\[\forall \ell\in (L^p(\mu))^*,\exists !g\in L^q(\mu)\text{ tal que } \ell(f)=\inn{f,g}\quad\forall f\in L^p\]

con $||\ell||_{(L^p)^*}=||g||_q$.

\end{ftheorem}

\begin{fnote}
    \begin{enumerate}
        \item Incluye el caso $p=2$ (Espacio de Hilbert).
        \item $\Omega:\N,\mu=\text{medida de contar}$
        
        \[L^p(\mu)=\{f:\N\to\K:\left(\sum |f_k|^p\right)^{1/p}<\infty\}=\ell^p\]

        \item El teorema dice que
        
        \[(\ell^p)^*\simeq \ell^q\quad p\in [1,\infty)\]

        \item $p=\infty$: $(L^\infty)^*\not\simeq L^1$
        
        \[\phi:L^1 \not \hookrightarrow (L^\infty)^*\quad\text{ no es sobreyectiva}\]
    \end{enumerate}
\end{fnote}

La demostración requiere la herramienta del Teorema de Radon-Nikodym.

\begin{fdefinition}
    $(\Omega,\mathcal{M})$ y medidas $\mu,\nu:\mathcal{M}\to[0,\infty]$. Decimos que $\nu$ es \textbf{absolutamente continua} respecto a $\mu$ si

    \[\mu(E)=0\implies \nu(E)=0\]

    y escribimos $\nu\ll\mu$.
\end{fdefinition}

\begin{fexample}
    Si $h\geq 0,h\in L^1(\mu)$ podemos definir $\nu:\mathcal{M}\to[0,\infty]$

    \[\nu(E)\coloneqq \int h\chi_E\,d\mu\eqqcolon \int_E h\,d\mu\]

    $\nu$ es una medida.

    \[``d\nu=h\,d\mu''\quad\text{$h$ es densidad}\]

    $\nu \ll \mu$ pues $\mu(E)=0$
    \[\implies \nu(E)=\int h\chi_E\,d\mu=0\]
\end{fexample}

\begin{ftheorem}[Radon-Nikodym]
    Sean $\mu$ y $\nu$ medidas en $(\Omega,\mathcal{M})$ $\sigma$-finitas. Si $\nu\ll\mu$, entonces $\exists!h\geq 0$ medible tal que $\nu(E)=\displaystyle\int_E h\,d\mu$. 

    ($h$ es única $\mu$-c.t.p.)
\end{ftheorem}

$(d\nu=h\,d\nu)$, $h=[\frac{d\nu}{d\mu}]$ derivada de Radon-Nimkodym.

\begin{proof}
    \textbf{Unicidad:}

    \[\int h_1\chi_E\,d\mu=\int h_2\chi_E\,d\mu=\nu(E)\quad\forall E\in\mathcal{M}\]

    \[\int (h_1-h_2)\chi_E\,d\mu=0\quad\forall E\in\mathcal{M}\]

    \[E=\{h_1>h_2\}\]

    \[0=\int_{\{h_1-h_2>0\}} (h_1-h_2)\,d\mu\implies \mu(\{h_1>h_2\})=0\implies \mu(\{h_1\neq h_2\})=0\implies h_1=h_2\quad\mu-\text{c.t.p.}\]

    \textbf{Existencia:} (argumento de Von Neumann) que utiliza el Teorema de Representación de Riesz en $L^2$. 

    idea: $\lambda=\mu+\nu$. Suponga que $\mu(\Omega),\nu(\Omega)<\infty$

    \[\mu(E)=0\iff \lambda(E)=0\]

    Vamos a definir un funcional lineal acotado 

    \begin{align*}
        \ell:L^2_\R(\lambda)&\to\R\\
        f&\to \int f\,d\mu
    \end{align*}

    $\ell$ es obviamente lineal y acotado:

    \begin{align*}
        \left|\int f\,d\mu\right|\leq \int |f|\,d\mu\\
        &=\int |f|\cdot 1\,d\mu\\
        &\leq \int |f|\cdot 1\,d\lambda\\
        &\leq ||f||_{L^2(\lambda)}\underbrace{||1||_{L^2(\lambda)}}_{[\lambda(\Omega)]^{1/2}}
    \end{align*}

    Es decir, $|\ell(f)|\leq (\lambda(\Omega))^{1/2}||f||_{L^2(\lambda)}$

    Por Teorema de Representación de Riesz:

    \[\ell(f)=\int fg\,d\lambda,\quad g\in L^2(\lambda)\]

    \begin{align}
        \int_{\Omega} f\,d\mu&=\int fg\,d\lambda\nonumber\\
        &=\int fg\,d\mu+\int fg\,d\nu\nonumber\\
        \int f(1-g)\,d\mu&=\int fg\,d\nu\quad\forall f\in L^2(\lambda)
    \end{align}

    Formalmente ``$(1-g)\,d\mu=g\,d\nu$'' $\implies$ ``$h=\frac{1-g}{g}$''.

    Primero vamos a demostrar que 

    \[0<g\leq 1\quad \mu-\text{c.t.p.}\]

    \begin{enumerate}[label=(\alph*)]
        \item $F\coloneqq \{g\leq 0\}$
        
        \[\mu(F)=\int \chi_F\,d\mu\leq \int \chi_F (1-g)\,d\mu\]

        por (3.1), 

        \[=\int\chi_F g\,d\nu\leq 0\]
        \[\implies \mu(F)=0\]

        \item $G\coloneqq \{g>1\}$. Suponga que $\mu(G)>0$
        
        \begin{align*}
            0>\int_G (1-g)\,d\mu&=\int (1-g)\chi_G\,d\mu\\
            &=\int (1-g)\chi_G\,d\nu\\
            &=\int_G g\,d\nu\geq 0
        \end{align*}

        lo que es una contradicción.

    \end{enumerate}

    $g\in L^2(\lambda)$. Podemos elegir representante $g$, tal que $0<g\leq 1$ en $\Omega$. Definimos 

    \[h\coloneqq \frac{1-g}{g}\geq 0\quad \text{en }\Omega\]

    Tome $A\in\mathcal{M}$, $f_n=\chi_{\{A\cap g\geq \frac{1}{n}\}}/g\in L^2(\lambda)$. Ponemos $f_n$ en $(3.1)$:

    \[\int f_n(1-g)\,d\mu=\int f_ng\,d\nu\]

    $x\in \{g<\frac{1}{n}\},f_n=0$. $x\in \{g\geq \frac{1}{n}\}$, $f_n\leq \frac{1}{\frac{1}{n}}=n$. $\implies f_n$ es acotada. $\implies f\in L^2(\lambda)$.

    \[\int \frac{1-g}{g}\chi_{A\cap\{g\geq \frac{1}{n}\}}\,d\mu=\ nt \chi_{A\cap \{g\geq \frac{1}{n}\}}\,d\nu\]

    \[A\cap\{g\geq\frac{1}{n}\}\nearrow A\]

    Tomando $\lim_{n\to\infty}$, por Teorema de Convergencia Monótona obtenemos 

    \[\int h\chi_A\,d\mu=\int\chi_A\,d\nu\]

    Ahora suponga que $\mu,\nu$ son $\sigma-$finitas: existe $\Omega_n\nearrow \Omega$, tales que 

    \[\mu(\Omega_n),\nu(\Omega_n)<\infty\]

    Aplicaremos el resultado a $(\Omega_n,\mathcal{M},\mu\text{ y }\nu|_{\Omega_n})$. $\mathcal{M}_n=\{E\cap \Omega_n:E\in\mathcal{M}\}$.

    \[\nu(A)=\int h_n\chi_A\,d\mu\quad\forall a\in \mathcal{M}_n\]

    para alguna $h_n\geq 0$ y $\mathcal{M}_n-$medible.

    \[=\int h_{n+1}\chi_A\,d\mu\]

    por unicidad 

    \[h_{n+1}|_{\Omega_n}=h_n\quad\mu-\text{c.t.p.}\]

    Extienda cada $h_n$ por $0$ fuera de $\Omega_n$. De esta manera $h_n$ es $\mathcal{M}-$medible. Defina $h\coloneqq \displaystyle\lim_{n\to\infty}h_n$

    \[h_n\nearrow h\]

    Para todo $E\in\mathcal{M}$

    \begin{align*}
        \nu(E)&=\lim_{n\to\infty} \nu(\Omega_n\cap E)\\
        &=\lim_{n\to\infty} \int_{E\cap\Omega_n} h_n\,d\mu\\
        &=\lim_{n\to\infty} \int_{E\cap \Omega_n} h\,d\mu\\
        &=\int_E h\,d\mu
    \end{align*}
\end{proof}

Necesitaremos también el siguiente resultado:

\begin{fdefinition}
    $\ell\in(L^p_\R)^*$ es \textbf{positivo} si $\ell(f)\geq 0\quad\forall f\in L^p_\R,f\geq 0$ 
\end{fdefinition}

\begin{ftheorem}
    Sea $\ell\in (L_\R^p)^*$, $1\leq p<\infty$. Entonces 

    \[\ell=\ell_+-\ell_-\]

    $\ell_{\pm}\in (L_\R^p)^*$ son positivos.
\end{ftheorem}

\begin{proof}
    Sea $\ell\in(L^p)^*$.

    \begin{enumerate}
        \item Definiremos $\ell_+$ para $f\geq 0$.
        
        \[\ell_+(f)\coloneqq \sup_{0\leq g\leq f}\ell(g)\]

        Obviamente, $\ell_{+}(cf)=c\ell_+(f)$, $c\geq 0$. Probemos la aditividad:

        \[\ell_+(f_1+f_2)=\ell_+(f_1)+\ell_+(f_2)\quad\forall f_1,f_2\geq 0\text{ en }L^p\]

        \[\ell\underbrace{(g_1+g_2)}_{g}=\ell(g_1)+\ell(g_2)\]

        Si $0\leq g_1\leq f_1$, $0\leq g_2\leq f_2$

        \[\implies g=g_1+g_2\leq f_1+f_2\]

        \[\sup_{0\leq g\leq f_1+f_2}\ell(g)\geq \ell(g_1)+\ell(g_2)\]

        Tomando $\sup$ sobre $0\leq g_1\leq f_1$, $0\leq g_2\leq f_2$

        \[\ell_+(f_1+f_2)\geq \ell(f_1)+\ell_(f_2)\]

        Para demostrar la otra, notamos que cada $0\leq g\leq f_1+f_2$ se puede escribir

        \[g=g_1+g_2\]

        donde $g_1\coloneqq \min(g,f_1)\leq f_1$, $g_2\coloneqq g-g_1\leq f_2$.

        \[\ell(g)\leq \ell_+(f_1)+\ell_+(f_2)\]
        \[\implies \ell_+(f)\leq \ell_+(f_1)+\ell_*(f_2)\]

        \item Extendemos $\ell_+$ a toda $f\in L^p$.
        
        \[f=f_+-f_-\]

        Definimos $\ell_+(f)=\ell_+(f_+)-\ell_+(f_-)$.

        Esta definicón no depende de como descomponemos $f$ como diferencia de 2 funciones no negativas.

        \[f=f_+-f_-=f_1-f_2\implies f_+f_2=f_1+f_-\]
        \[\implies \ell_+(f_1+f_2)=\ell_+(f_1+f_-)\]
        \[\implies \ell_+(f_+)+\ell_+(f_2)=\ell_+(f_1)+\ell_+(f_-)\]
        \[\implies \ell_+(f_+)-\ell_+(f_-)=\ell_+(f_1)-\ell_+(f_2)\]

        \item Por lo tanto, $\ell_+$ es \textbf{lineal}
        
        \[\ell_+(cf)=c\ell_+(f)\quad\forall c\geq 0\]
        \[\ell_+(-cf)=\ell_+(c(-f))=c\ell_+(-f)=-c\ell_+(f)\]
        \[\implies \ell_+(-f)=\ell_+(f_-)-\ell_+(f_+)=-\ell_+(f)\]

        $\ell_+$ es acotado.

        \begin{align*}
            |\ell_+(f)|&=|\ell_+(f_+)-\ell_+(f_-)|\\
            &\leq |\ell_+(f_+)|+|\ell_+(f_-)|\\
            &\leq ||\ell||\cdot||f_+||_{L^p}+||\ell||\cdot||f_-||_{L^p}\\
            &\leq 2||\ell||\cdot||f||_{L^p}
        \end{align*}

        ya que

        \begin{align*}
            \ell_+(f)&\leq \sup_{0\leq g\leq f}||\ell||\cdot||g||_{L^p}\\
            &\leq ||\ell||\cdot ||f||_{L^p}
        \end{align*}

        Definimos 

        \[\ell_-\coloneqq \ell_+-\ell\]
        \[\implies \ell \in (L^p_\R)^*\]

        $\ell_-$ es positiva pues $\forall f\geq 0,f\in L^p$.

        \begin{align*}
            \ell_-(f)&=\ell_+(f)-\ell(f)\geq 0\\
            &=\sup\{\ell(g):0\leq g\leq f\}-\ell(f)\geq 0
        \end{align*}
    \end{enumerate}
\end{proof}

\begin{proof}[Demostración del Teorema de Riesz (\ref{theo:3.7.5})]

    % \begin{fexercise}
    %     Demostrar la unicidad
    % \end{fexercise}

    $\K=\R, \ell\in (L^p_\R)^*$ y positivo. Supondremos que $\mu(\Omega)<\infty$. Definimos 

    \begin{align*}
        \nu:\mathcal{M}&\to [0,\infty)\\
        A&\to \ell(\chi_A)
    \end{align*}

    Afirmamos que $\nu$ es una medida finita.

    \begin{enumerate}[label=(\alph*)]
        \item $\nu\geq0$ 
        \[\nu(\Omega)\leq ||\ell||\cdot ||\chi_\Omega||_{L^p}=||\ell||\left(\int_\Omega 1^p\,d\mu\right)^{1/p}=||\ell||(\mu(\Omega)^{1/p})<\infty\]

        \item $\nu(\varnothing)=0$
        \item Si $E=\biguplus E_k$, $\chi_{\bigcup_{k=1}^N E_k}\nearrow \chi_E$
        
        \[0\leq |\chi_E-\chi_{\bigcup_{k=1}^N E_k}|^p\leq \chi_E^p\in L^1\]

        Por Teorema de Convergencia Dominada 

        \[\implies \chi_{\bigcup_{k=1}^N E_k}\xrightarrow{L^p}\chi_E\]
        \[\implies \ell(\chi_{\bigcup_{k=1}^n E_k})\to \ell(\chi_E)\iff \sum_{k=1}^N \ell(\chi_{E_k})\xrightarrow{N\to\infty} \ell(\chi_E)\]

        \begin{align*}
            \nu(E)&=\ell(\chi_E)\\
            &=\lim_{N\to\infty} \sum_{k=1}^N \ell(\chi_{E_k})\\
            &=\lim_{N\to\infty} \sum_{k=1}^N \nu(E_k)\\
            &=\sum_{k=1}^\infty \nu(E_k)
        \end{align*}

        Además, $\nu\ll \mu$

        \begin{align*}
            \nu(A)&\leq ||\ell||\mu(A)^{1/p}
        \end{align*}

        Si $\mu(A)=0\implies \nu(A)=0$ Por el teorema de Radon-Nikodym, 

        \[\ell(\chi_A)=\nu(A)=\int \chi_A h\,d\mu\]

        para una $h\geq 0$. Tomando combinaciones lineales finitas de $\chi_A$'s:

        \[\ell(s)=\int sh\,d\mu\]

        para toda función simple $s:\Omega\to\R$. Ahora, cada función $f\geq 0$ no negativa 

        \[0\leq s_n\leq f,\quad s_n\nearrow f\]

        Por Teorema de Convergencia Dominada,

        \[\implies s_n\xrightarrow{L^p}f\]

        Entonces, 

        \[\ell(f)=\lim_{n\to\infty}\ell(s_n)=\lim_{n\to\infty} \int s_nh\,d\mu\]
        \[\int fh\,d\mu\]

    \end{enumerate}

        Cuando $\mu$ es $\sigma$-finita,
        
        \[\Omega=\biguplus_n \Omega_n,\quad \mu(\Omega_n)<\infty\]

        En cada $\Omega_n$, tenemos $h_n\geq 0$, tal que 

        \[\ell(f\chi_{\Omega_n})=\int f\chi_{\Omega_n} h_n\,d\mu\quad \forall f\geq 0,f\in L^p\]

        Extienda $h_n$ por 0 fuera de $\Omega_n$. Tenemos que

        \[\sum_{n=1}^N f\chi_{\Omega_n}\xrightarrow{L^p}f\]

        por lo que 

        \[\ell(f)=\lim_{N\to\infty}\ell(\sum_{n=1}^N f\chi_{\Omega_n})=\lim_{N\to\infty} \sum_{n=1}^N \int f\chi_{\Omega_n}h_n\,d\mu\]
        \[=\lim_{N\to\infty} \sum_{n=1}^N \int \Omega fh_n\,d\mu=\lim_{N\to\infty}\int f\sum_{n=1}^N h_n\,d\mu\]
        \[=\int_{\Omega} fh\,d\mu\]
        
        donde $h\coloneqq \displaystyle\sum_{n=1}^\infty h_n$. En particular, 

        \[fh\in L^1(\mu)\]

        Tomaremos ahora un $f\in L^p_\R(\mu)$ con signo arbitrario. $\ell\in (L_\R^p)^*$ y positivo

        \begin{align*}
            \ell(f)&=\ell(f_+-f_-)\\
            &=\ell(f_+)-\ell(f_-)\\
            &=\int f_+h-\int f_-h\int fh\,d\mu
        \end{align*}

        $f\in L^p\implies |f|\in L^p,|f|\geq 0$

        \[\implies \ell(|f|)=\int |f|h\,d\mu\implies |f|h\in L^1\]

        Por lo tanto, $f_{\pm}h\in L^1$

        \[\int (f_+-f_-)g=\int f_+h-f_-h=\int f_+h-\int f_-h\]

        Si $\ell\in (L^p)^*$, lo expresamos como diferencia de 2 funcionales positivos:

        \[\ell=\ell_+-\ell_-\]

        cada una con su $h_\pm$ correspondiente, $h\coloneqq h_+-h_-$

        \[\forall f\in L^p_\R(\mu), fh_\pm \in L^1\]
        \[\implies fh=fh_+-fh_-\in L^1\]
        \[\implies \ell(f)=\ell_+(f)-\ell_-(f)=\int fh_+-\int fh_-=\int fh\]

        En esta etapa hemos demostrado que $\forall \ell\in (L_\R^p)^*$ se puede escribir como 

        \[\ell(f)=\int fh\,d\mu\]

        para alguna $h$ medible donde $fh\in L^1$.

        Para extender al caso complejo, noten que si $\ell\in (L_\C^p)^*$ y $f\in L^p_\R$.

        \[\implies \ell(f)=\Re\ell(f)+i\Im\ell(f)\]

        donde $\Re \ell\in (L_\R^p)^*$ y $\Im \ell\in (L^p_\R)^*$.

        \[|\Re \ell(f)|\leq |\ell(f)|\leq ||\ell||_{L_\C^p}^*||f||_{L_\C^p}=||\ell||_{(L_\C^p)^*}||f||_{L^p_\R}\]

        \[\Re \ell=\inn{\cdot,h_1}\]
        \[\Im \ell=\inn{\cdot,h_2}\]

        donde $fh_i\in L^1$. Por lo tanto, si 

        \[h\coloneqq h_1+ih_2\]

        \[|fh|\leq |fh_1|+|fh_2|\quad \forall f\in L_\R^p\]

        y 

        \[\ell(f)=\inn{f,h_1}+i\inn{f,h_2}=\inn{f,h_1+ih_2}\quad \forall f\in L^p_\R\]

        Por linealidad 

        \[\ell(f)=\inn{f,h}\quad\forall f\in L_\C^p\]

        donde $|fh|\in L^1$.

        $\ell\in (L^p)^*$. Hemos demostrado que existe $h$  medible tal que 

        \[fh\in L^1\quad\forall f\in L^p\]

        y 

        \[\ell(f)=\int fh\,d\mu\]

        Afirmamos que $h\in L^q$ y $||\ell||=||h||_q$. Por Hölder,

        \[|\ell(f)|\leq \int |fh|\,d\mu\leq ||f||_p||h||_q\]
        \[\implies ||\ell||\leq ||h||_q\]

        Mostraremos ahora la otra desigualdad

        \begin{enumerate}[label=(\alph*)]
            \item $p\in (1,\infty)$. Defina 
            
            \[B_n\coloneqq\Omega_n\cap \{|h|\leq n\},\quad\Omega_n\nearrow \Omega\]
            \[f_n\coloneqq |h|^{q-1}\sgn h\chi_{B_n}\]

            $f_n\in L^p$

            \begin{align*}
                ||f_n||_p^p&=\int_{B_n}(|h|^{q-1})^p=\int_{B_n}|h|^q\\
                \implies ||f_n||_p&=\left(\int |h|^q\right)^{1/p}
            \end{align*}
            \begin{align*}
                \ell(f_n)&=\int f_nh\,d\mu=\int |h|^{q-1}\sgn h h\,d\mu\\
                &=\int_{B_n}|h|^q\,d\mu
            \end{align*}

            \[\int_{B_n}|h|^q \,d\mu=\ell(f_n)\leq ||\ell||\cdot ||f_n||_p=||\ell||\left(\int |h|^q\,d\mu\right)^{1/p}\]
            \[\left(\int_{B_n}|h|^q \,d\mu\right)^{1-1/p}\leq ||\ell||\]
            \[\left(\int_{B_n}|h|^q\right)^{1/q}\leq ||\ell||\]

            Por Teorema de Convergencia Monótona,

            \[\left(\int_{B_n}|h|^q\right)^{1/q}\xrightarrow{n\to\infty} ||h||_q\]

            \item $p=1, q=\infty$. Suponga que $||\ell||+2\varepsilon\leq ||h||_\infty$, para algún $\varepsilon>0$.
            
            \[||h||_\infty=\inf\{M>0:|h|\leq M\quad \text{c.t.p.}\}\]

            $\exists A\in\mathcal{M},\mu(A)>0$, tal que 

            \[|h|\geq ||h||_\infty-\varepsilon\quad\forall x\in A\]

            Ya que $\mu$ es $\sigma$-finita, $\Omega_n\nearrow \Omega$

            \[A_n\coloneqq A\cap \Omega_n\nearrow A\]

            Tenemos $|h|\geq ||h||_\infty-\varepsilon$ en $A_n$ donde $0<\mu(A_n)<\infty$. Tome $f_n\coloneqq \sgn h\chi_{A_n}\in L^1$

            \begin{align*}
                \ell(f)\int fh\,d\mu&=\int_{A_n} |h|\\
                &\geq (||h||_\infty-\varepsilon)\mu(A_n)\\
                &\geq (||\ell||_\infty+\frac2\varepsilon-\varepsilon)||f||_1\\
                &=(||\ell||_\infty +\varepsilon)||f||_1
            \end{align*}

            $\implies f$ viola la norma de $||\ell||$. Contradicción
        \end{enumerate}
\end{proof}

\section{Teorema de Hahn-Banach}

Sea $X$ un espacio normado, y sea $X^*$ su dual = espacio de funcionales lineales acotados. No hemos visto si \textbf{existen} aún funcionales lineales acotados en $X$ \textbf{no triviales}. Resulta ser el caso que hay una \textbf{abundancia} de funcionales lineales acotados en $X$.

$X\rightarrow$ espacio vectorial. $X'\rightarrow$ espacio de funcionales lineales ($X\neq X'$). Diremos que $f\in X'$ ($f:X\to\K$) extiende, $g\in Y'$ ($Y\subseteq X$ subespacio, $g:Y\to\K$). Si 

\[f(y)=g(y)\quad\forall y\in Y\]

$(X,f)\succ (Y,g)$.

\begin{fdefinition}
    Sea $X$ un espacio vectorial \textbf{real}. Decimos que 

    \[p:X\to \R\]

    es un \textbf{funcional} convexo si satisface 

    \begin{enumerate}
        \item (Homogeneidad positiva) $p(\lambda x)=\lambda p(x)$, $\forall \lambda\geq 0$
        \item (subaditividad) $p(x+y)\leq p(x)+p(y)$
    \end{enumerate}

    Se dice convexo porque

    \begin{align*}
        p(\lambda x+(1-\lambda)y)&\leq p(\lambda x)+p((1-\lambda)y)\\
        &=\lambda p(x)+(1-\lambda)p(y)
    \end{align*}
\end{fdefinition}

\begin{fexample}
    Una seminorma/norma es un funcional convexo.
\end{fexample}

\begin{fexample}
    Un funcional lineal (sobre $\R$) es un funcional convexo.
\end{fexample}

\begin{fdefinition}
    Decimos que el funcional convexo $p$ domina el funcional lineal $f$ si 

\[f(x)\leq p(x)\quad\forall x\in X\]
\end{fdefinition}

\begin{fproposition}
    Sea $X$ normado. $f\in X'$ es \textbf{acotado} si y solo si $f$ es dominado por $p(x)\coloneqq M||x||$ para alguna $M>0$.
\end{fproposition}

\begin{proof}
    $(\implies)$: $f\in X*$ 

    \begin{align*}
        |f(x)|&\leq M||x||\\
        \implies f(x)\leq M||x||
    \end{align*}

    $(\impliedby)$: 

    \begin{align*}
        f(x)&\leq M||x||\quad\forall x\in X\\
        -f(x)=f(-x)&\leq M||-x||=M||x||\\
        \implies -M||x||\leq f(x)&\leq M||x||\\
    \end{align*}
\end{proof}

\begin{ftheorem}[Hahn-Banach]
    Sean $X,Y$ espacios vectoriales \textbf{reales}, $Y\subseteq X$ y sea $p:X\to\R$ un funcional lineal \textbf{convexo}. Si $f\in Y'$ es dominado por $p$,

    \[f(y)\leq p(y)\quad\forall y\in Y\]

    entonces existe una \textbf{extensión} $F\in X'$ dominado por $p$:

    \[F(x)\leq p(x)\quad\forall x\in X\]
\end{ftheorem}

\begin{fcorollary}
    $X$ es un espacio normado \textbf{real}. $Y\subseteq X$ subespacio $Y\neq \{0\}$. $f\in Y^*$. Entonces existe una extensión $F\in X^*$ con 

    \[||F||_{X^*}=||f||_{Y^*}\]
\end{fcorollary}

$(Y=\gen(v))$, $f(\lambda v)=\lambda$, $||f||_{Y^*}=\frac{1}{||v||}$. Por Hahn-Banach, $F:X\to\R$, $||F||_{X^*}=||f||_{Y^*}$.

\begin{proof}
    Defina $p(x)\coloneqq ||f||_{Y^*}||x||$. $f$ es dominado por $p$., por lo que por el Teorema de Hahn-Banach nos da una extensión 

    \[F:X\to\R\]
    \[F(x)\leq p(x)=||f||_{Y^*}||x||\]

    $\implies F\in X^*$ y $||F||_{X^*}\leq ||f||_{Y^*}$.

    \[||F||_{X^*}=||f||_{Y^*}\]

    pues es una extensión.
\end{proof}

\begin{proof}[Demostración del Teorema de Hahn-Banach (\ref{theo:3.8.2})]
    Asumimos que $Y\subsetneq X\implies $ existe $z\in X\setminus Y$. Vamos a extender $f$ a $F:Y+\gen(z)\to \R$ de la manera que $F$ sea \textbf{dominado} por $p$.

    \[F(y+tz)\coloneqq f(y)+t s,\quad F(z)=s\]

    define un funcional lineal en $Y+\gen(z)$. La meta es es elegir $s$ de tal manera que $F(y+tz)=f(y)+ts\leq p(y+tz)$. Noten que se satisface cuando $t=0$. Afirmamos que para demostrar su validez $\forall t\neq 0$ basta ver que se satisface oara $t=\pm 1$.

    \begin{align*}
        F(y+tz)&=|t|F\left(\frac{y}{|t|}+\frac{t}{|t|}z\right)\\
        &=|t|f\left(\frac{y}{|t|}+\sgn ts\right)\\
        &\leq |t| p\left(\frac{y}{|t|}+\sgn ts\right)=p(y+tz)
    \end{align*}

    Meta: elija $s$ de modo que 

    \[f(y)+s\leq p(y+z)\quad t=1\]
    \[f(y')-s\leq p(y'-z)\quad t=-1\]

    $\forall y,y'\in Y$. Tal $s$ existe si 

    \[\sup_{y'\in Y} f(y')-p(y'-z)\leq \inf_{y\in Y} p(y+z)-f(y)\]

    Esto es válido cuando 

    \[f(y')-p(y'-z)\leq p(y+z)-f(y)\quad\forall y,y'\in Y\]
    \begin{align*}
        \iff f(y')+f(y)&\leq p(y+z)+p(y'-z)\\
        \iff f(y'+y)\leq p(y+z)+p(y'-z)\quad\forall y,y'\in Y
    \end{align*}

    Lo que es verdadero por la convexidad de $p$ y porque $f$ está dominado por $p$.

    \begin{align*}
        f(y'+y)\leq p(y+y')&=p(y+z+y'-z)\\
        &\leq p(y+z)+p(y'-z)\quad\forall y,y'\in Y
    \end{align*}

    De esta manera obtuvimos 

    \[(\tilde Y,F)\succ (Y,f)\]

    Para extender a todo $X$ utilizaremos un argumento estándar por el Lema de Zorn. 

    \begin{fdefinition}
        Orden parcial $\prec$ en un conjunto $E$ es una relación entre algunos de los elementos de $E$ que satisface

        \begin{enumerate}
            \item $e\prec e$
            \item $e\prec f$ y $f\prec e\implies e=f$
            \item $e\prec f$ y $f\prec g\implies e\prec g$
        \end{enumerate}
    \end{fdefinition}

    \begin{fdefinition}
        Un subconjunto $C\subseteq E$ se llama \textbf{cadena} si es totalmente ordenado. Es decir, todo par de elementos de $C$ son relacionados.
    \end{fdefinition}

    \begin{fdefinition}
        Una cota superior de $D\subseteq E$ es un elemento $e\in E$ tal que 

        \[d\prec e\quad\forall d\in D\]
    \end{fdefinition}

    \begin{fdefinition}
        Un elemento \textbf{maximal} $m\in E$ es un elemento de $E$ que no puede ser dominado: si 

        \[m\prec g\implies m=e\]
    \end{fdefinition}

    \begin{flemma}[Lema de Zorn]
        Si toda cadena de un conjunto $E$ parcialmente ordenado tiene una cota superior, entonces $E$ tiene un elemento maximal.
    \end{flemma}

    Lema de Zorn $\iff$ Axioma de Elección

    En nuestro contexto, 

    \[E=\{(L, \ell):(L,\ell)\succ (y,f),\ell:L\to\R \text{ es dominado por }p\}\]

    Orden es: $(L_1,\ell_1)\succ (L_2,\ell_2)$. Sea $C\subseteq E$ una cadena.

    \[C=\{(L_\alpha,\ell_\alpha)\}_\alpha\]

    \[L\coloneqq\bigcup_{\alpha}L_\alpha\text{ es un subespacio vectorial de }X\]

    \begin{align*}
        x,y\in L\implies x\in L_{\alpha_1}, y\in L_{\alpha_2}\supseteq L_{\alpha_1}\\
        \implies x+y\in L_{\alpha_2}\implies \lambda x+\mu y\in L_{\alpha_2}\subseteq L
    \end{align*}

    Defina 

    \begin{align*}
        \ell:L&\to\R\\
        x&\to \ell_\alpha(x)\quad \text{si } x\in L_\alpha
    \end{align*}

    Por la rezón anterior, no hay ambiguedad en esta definición: si $x\in L_\beta$, podemos asumir que $(L_\beta,\ell_\beta)\succ (L_\alpha,\ell_\alpha)$
    \[\implies \ell_\beta(x)=\ell_\alpha(x)\quad \forall x\in L_\alpha\]

    Concluimos que $(L,\ell)$ es una cota superior de $C$. Por el Lema de Zorn, existe un elemento maximal $(\tilde L, \tilde F)$. $\tilde L=X$. Si $\tilde L\subsetneq X$, podemos extender $\tilde F$ a $\tilde L+\gen(z)$, $z\in X\setminus \tilde L$ siendo dominado por $p$. Esto contradice la maximalidad de $(\tilde L,\tilde F)$.
\end{proof}

En el caso de espacios normados \textbf{complejos}, utilizaremos la siguiente observación: Todo $X_\C$ se puede ver como un espacio vectorial \textbf{real} $X_\R$.

Tenemos la siguiente correspondencia \textbf{biyectiva}:

\begin{align*}
    X_\R'&\stackrel{\sim}{\smash{\longrightarrow}\rule{0pt}{0.4ex}} X_\C'\\
    u&\longrightarrow x\to u(x)+\frac{1}{i}u(ix)\eqqcolon F(x)\\
    \Re F&\longleftarrow F
\end{align*}

$F(x)=\underbrace{\Re F(x)}_{u(x)}+i\Im F(x)$

\begin{align*}
    \Im F(x)&=\Re\left[\frac{1}{i} F(x)\right]\\
    &=-\Re[i F(x)]\\
    &=-\Re[F(ix)]\\
    &=-u(ix)
\end{align*}

Además, tenemos $\alpha\in \C,|\alpha|=1$

\begin{align*}
    |F(x)|&=\underbrace{\sgn F(z)}_\alpha F(x)=\alpha F(x)=F(\alpha x)\\
    &=\Re F(\alpha x)=u(\alpha x)=|u(\alpha x)|
\end{align*}

Cuando $X_\C$ es normado, $X_\R$ hereda la norma de $X_\C$. Si 

\[F\in X_\C^*\implies u=\Re F\in X_\R^*\]

y 

\[||F||_{X_\C^*}=||u||_{X_\R^*}\]

\begin{ftheorem}
    Suponga que $Y_\C\subseteq X_\C$, $X_\C$ un espacio normado complejo, y $f\in Y_\C^*$. Entonces $f$ se puede extender a $F\in X_\C^*$ preservando la norma: $||F||_{X^*}=||f||_{Y^*}$
\end{ftheorem}

\begin{proof}
    $f=\underbrace{\Re f}_{u\in Y_\R^*}+\underbrace{I\Im f}_{\frac{1}{i}u(i\cdot)}$

    Extendemos $U\in X_\R^*$ donde $||U||_{X_\R^*}=||u||_{Y_\R^*}$. Definimos 

    \[F(x)\coloneqq U(x)+\frac{1}{i}U(ix)\]

    Tenemos $||F||_{X_\C^*}=||U||_{X_\R^*}=||u||_{Y_\R^*}=||f||_{Y_\C^*}$
\end{proof}

\begin{fcorollary}\label{theo:3.8.4.1}
    Para todo $x_0\in X$, $X$ normado, existe $f_0\in X^*$ tal que $||f_0||=1$ y tal que $f_0(x_0)=||x_0||$.
\end{fcorollary}

\begin{proof}
    Aplicamos el teorema anterior a $Y=\gen(x_0)$ y

    \begin{align*}
        \bar f_0:Y&\to \K\\
        tx_0&\to t||x_0||
    \end{align*}

    \[\bar f_0(x_0)=||x_0||,||\bar f_0||_{Y^*}=1\]

    Lo extendemos $f_0\in X^*$.
\end{proof}

\begin{fcorollary}
    \[||x||=\sup\{|f(x)|:||f||=1\}\]
\end{fcorollary}

\begin{proof}
    \[|f(x)|\leq ||f||\cdot ||x||=||x||\]

    Por \ref{theo:3.8.4.1}, $||x||=|f(x)|$ para algún $f\in X^*$ con $||f||=1$.

    \[\implies ||x||\leq \sup\{|f(x)|:||f||=1\}\]
\end{proof}

\begin{ftheorem}
    $\forall x\in X, X$ espacio normado, define un funcional lineal acotado en $X^*$

    \begin{align*}
        \hat x:X^*&\to\K\\
        f&\to f(x)
    \end{align*}

    \[||\hat x||=\sup_{\substack{f\in X^*\\ ||f||=1}}||f(x)||=||x||\]

    Entonces, el mapeo 

    \begin{align*}
        \mathcal{J}:X&\to (X^*)^*\\
        x&\to \hat x
    \end{align*}

    es una isometría lineal.
\end{ftheorem}

 \begin{fnote}\hfill
    \begin{itemize}
        \item $\mathcal{J}$ es inyectivo.

        \item \[\overline{\mathcal{J}(X)}\subseteq (X^*)^*\implies \overline{\mathcal{J}(X)}\simeq \text{completación de }X\]

        \item Cuando $\mathcal{J}$ es sobreyectivo, $X\simeq X^{**}$ es un espacio de Banach. que se llama \textbf{reflexivo}.
    \end{itemize}
 \end{fnote}

 \begin{fexample}
    (Espacio de dimensión finita) $L^p(\mu)$, $p\in (1,\infty]$.
 \end{fexample}

 \section{Relaciones de Ortogonalidad}

 \paragraph*{Notación:} $x\in X,f\in X^*$, $X$ normado. $f(x)\coloneqq \inn{f,x}$. $Y\subseteq X$, definimos el \textbf{aniquilador} de $Y$
 
 \[Y^\perp\coloneqq \{f\in X^*:\inn{f,y}=0\quad\forall y\in Y\}\subseteq X^*\]

 Similarmente, $Z\subseteq X^*$,

 \[Z^\perp\coloneqq \{x\in X:\inn{f,x}=0\quad\forall f\in Z\}\subseteq X\]

 Obviamente $Y^\perp$ es un subespacio cerrado de $X^*$ y $Z^\perp$ es un subespacio cerrado de $X$.

 \begin{fexample}
    Cuando $X$ es un espacio de Hilbert, $X^*\simeq X$ por Riesz. $Y\subseteq X$, el complemento ortogonal 

    \[Y^\perp=\{x\in X:\inn{x,y}=0\quad\forall y\in Y\}\]

    $\simeq$ aniquilador de $Y$.
 \end{fexample}

 \begin{fproposition}
    Sea $Y\subseteq X$ subespacio del espacio normado $X$. Entonces, $(Y^\perp)^\perp=\overline{Y}$
 \end{fproposition}

 \begin{proof}
    Es fácil ver que $Y\subseteq (Y^\perp)^\perp$.\\
    Para demostrar la otra inclusión, suponga que $\overline{Y}\subsetneq (Y^\perp)^\perp$. Entonces existe $x\neq 0$, $x\in (Y^\perp)^\perp\setminus \overline{Y}$. Defina

    \begin{align*}
        f:\overline{Y}+\gen(x)&\to\K\\
        y+\lambda x&\to \lambda
    \end{align*}

    Obviamente $f$ es un funcional lineal en $\overline{Y}+\gen(x)$ que satisface:

    \[f(x)=1\]
    \[f(y)=0\]

    Además $f$ es acotado en $Z\coloneqq \overline{Y}+\gen(x)$:

    \[f(y)=0\quad\forall y\in Y\]

    Sea $z=y+\lambda x,\lambda\neq 0$. $\implies z\neq 0$.

    \begin{align*}
        |f(z)|&=|\lambda|=\frac{|\lambda|}{||z||}||z||\\
        &=\frac{|\lambda|}{||y+\lambda x||}||z||=\frac{1}{||\frac{y}{\lambda}+x||}||z||\leq \frac{1}{dist(x,\overline{Y})}||z||
    \end{align*}

    Por Teorema de Hahn-Banach, podemos extender $f$ a todo $X$, y asumir que $f\in X^*$. Además,

    \[\inn{f,y}=0\quad\forall y\in \overline{Y}\implies f\in \overline{Y}^\perp\supseteq Y^\perp\]

    pero $f(x)\neq 0$. Por otro lado, 

    \[x\in (Y^\perp)^\perp\implies \inn{g,x}=0\quad\forall g\in Y^\perp\]

    En particular, $\inn{f,x}=0$, lo que es una contradicción.
 \end{proof}

 Suponga que $T\in \mathcal{B}(X,Y)$, $X,Y$ normados. Definimos el \textbf{operador adjunto/transpuesto}

 \begin{align*}
    T^*:Y^*&\to X^*\\
    f&\to f\circ T\eqqcolon T^*(f)\\
    \inn{T^*f,x}&=\inn{f,T(x)}\quad\forall x\in X
 \end{align*}

 Obviamente $T^*$ es lineal y es acotado:

 \begin{align*}
    |T^*f(x)|=|f(Tx)|\leq ||f||_{Y^*}||Tx||_{Y}&\leq ||f||_{Y^*}||T||_{\mathcal{B}(X,Y)}||x||_X\\
    \implies ||T^*f||_{X^*}&\leq ||f||_{Y^*}||T||_{\mathcal{B}(X,Y)}\\
    \implies ||T^*||_{\mathcal{B}(Y^*,X^*)}&\leq ||T||_{\mathcal{B}(X,Y)}\\
    \implies ||T^*||&\in\mathcal{B}(Y^*,X^*)
 \end{align*}

 \begin{ftheorem}
    La asignación 

    \begin{align*}
        \mathcal{B}(X,Y)&\to \mathcal{B}(Y^*,X^*)\\
        T&\to T^*
    \end{align*}

    es una isometría lineal. Además,

    \begin{enumerate}[label=(\alph*)]
        \item $(\operatorname{Im} T)^\perp=\ker T^* (\subseteq Y^*)$
        \item $(\ker T^*)^\perp=\overline{\operatorname{Im}T} (\subseteq Y)$
        \item $(\operatorname{Im}T^*)^\perp=\ker T (\subseteq X)$
    \end{enumerate}
 \end{ftheorem}

 \begin{proof}
    Obviamente, $(\lambda T_1+T_2)^*=\lambda T_1^*+T_2^*$

    \begin{align*}
        ||T||&=\sup_{||x||_X=1}||Tx||_Y\\
        &=\sup_{||x||_X=1}\left(\sup_{||f||_{Y^*}=1}|\inn{f,Tx}|\right)\\
        &=\sup_{\substack{||x||_X=1\\||f||_{Y^*}=1}}|\inn{f,Tx}|=\sup|\inn{T^*f,x}|\\
        &=\sup_{||f||_{Y^*}=1}\sup_{||x||_X=L}|T^*f(x)|\\
        &=\sup_{||f||_{Y^*}=1}||T^*f||=||T^*||
    \end{align*}

    \begin{enumerate}[label=(\alph*)]
        \item \begin{align*}
            f\in (\operatorname{Im}T)^\perp&\iff \inn{T^*f,x}=\inn{f,Tx}=0\quad\forall x\in X\\
            &\iff T^*f=0\iff f\in \ker T^*
        \end{align*}

        \item \begin{align*}
            (\ker T^*)^\perp=((\operatorname{Im}T)^\perp)^\perp=\overline{\operatorname{Im}T}
        \end{align*}

        \item \begin{align*}
            x\in (\operatorname{Im}T^*)&\iff \inn{T^*f,x}=0\quad\forall f\in Y^*\\
            &\iff \inn{f,Tx}=0\quad\forall f\in Y^*\\
            &\iff Tx=0\iff x\in \ker T 
        \end{align*}
    \end{enumerate}
 \end{proof}

 \section{Operadores Compactos}

 \begin{fdefinition}
    Sean $X,Y$ espacios de Banach.

    \[B^X\coloneqq \{x\in X:||x||_X\leq 1\}\]

    Decimos que un operador lineal $T:X\to Y$ es \textbf{compacto} si $\overline{T(B^X)}$ es compacto en $Y$.\\
    $\iff$ toda sucesión en ${T(B^X)}$ tiene una subsucesión convergente en $Y$.\\
    $\iff$ toda sucesión en $T(B^X)$ tiene una subsucesión de Cauchy.

    Denotamos la clase de opradores \textbf{compactos} con $\mathcal{B}_c(X,Y)$
 \end{fdefinition}

 \begin{ftheorem}
    $\mathcal{B}_c(X,Y)\subseteq \mathcal{B}(X,Y)$ es un subespacio cerrado. 

    \[\{T_n\}\subseteq \mathcal{B}_c(X,Y)\text{ y }||T_n-T||\xrightarrow{n\to\infty} 0\implies T\in \mathcal{B}_c(X,Y)\]
 \end{ftheorem}

 \begin{proof}
    $T$ compacto $\implies T$ es acotado. $\overline{T(B^X)}$ es compacto en $Y$.

    \begin{align*}
        \overline{T(B^X)}&\subseteq \bigcup_{n=1}^\infty B_n^Y\\
        &\subseteq B_N^Y\quad\text{para algún $N\in \N$}\\
        \implies ||Tx||\leq N\quad&\forall ||x||=1\implies ||T||\leq N
    \end{align*}

    $\mathcal{B}_c(X,Y)$ es un subespacio de $\mathcal{B}(X,Y)$:

    \[T\text{ compacto}\implies \lambda T\text{ compacto}\]

    $T_1,T_2$ compactos. Tome $\{x_n\}\in B^X$, entonces existe una subsucesión $\{x_{n_k}\}_k$: $T_1 x_{n_k}\to y_1$, $T_2x_{n_k}\to y_2$

    \[\implies (T_1+T_2)x_{n_k}\to y_1+y_2\implies T_1+T_2\text{ es compacto}\]
 \end{proof}