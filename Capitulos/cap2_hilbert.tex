\section{Espacios de Hilbert}

\begin{fdefinition}
    Sea $H$ un espacio vectorial sobre $\K=\R$ o $\C$. Un producto interno $\langle \cdot,\cdot\rangle$ es una función $H\times H\to \K$ que satisface

    \begin{enumerate}
        \item Linealidad en $\langle\cdot,y\rangle,\quad \forall y\in H$:
        
        \[\langle x_1+x_2,y\rangle=\langle x_1,y\rangle+\langle x_2,y\rangle\]
        \[\langle \lambda x,y\rangle = \lambda \langle x,y\rangle\]

        \item (Hermiticidad)
        
        \[\langle y,x\rangle=\overline{\langle x,y\rangle}\]

        (En $\K=\R$, esto es simetría)

        \item (Definidad) $\langle x,x\rangle\geq 0$ y $\langle x,x\rangle=\implies x=0$
    \end{enumerate}
\end{fdefinition}

\begin{fnote}
    1. y 2., implican que $\langle x,\cdot\rangle$ es lineal conjugada en la segunda entrada.

    \[\langle x,\lambda y+z\rangle=\overline{\lambda}\langle x,y\rangle + \langle x,z\rangle\]
\end{fnote}

\paragraph*{Terminología} Tal función se llama \textbf{forma sesquilineal}

\begin{fnote}
    $\K=\R$, $\inn{\cdot,\cdot}$ es una \textbf{forma simétrica definida positiva}
\end{fnote}

Decimos que $(H,\inn{\cdot,\cdot})$ es un \textbf{espacio pre-Hilbertiano}

De 1. y 2., $\inn{0,y}=0$, $\inn{x,0}=0$

Definimos $||x||:=\inn{x,x}^{1/2}$

\begin{fproposition}[Desigualdad de Cauchy-Schwarz]
    Sea $H$ un espacio pre-Hilbertiano

    \[|\inn{x,y}|\leq ||x||\cdot ||y||\quad \forall x,y\in H\]
\end{fproposition}

\begin{proof}
    Si $y=0$, la desigualdad es verdadera. Podemos asumir que $y\neq 0$.

    \begin{align*}
        0&\leq \inn{x+\lambda y,x+\lambda y}\\
        &=\inn{x,x}+\lambda\inn{y,x}+\overline{\lambda}\inn{x,y}+\lambda \overline{\lambda}\inn{y,y}\\
        &=||x||^2 +\underbrace{\lambda \overline{\inn{x,y}}+\overline{\lambda}\inn{x,y}}_{2\Re (\inn{x,y}\overline{\lambda})}+|\lambda|^2|\cdot |y||^2
    \end{align*}

    Evaluando en $\lambda=-\dfrac{\inn{x,y}}{||y||^2}$

    \begin{align*}
        0&\leq ||x||^2+2\Re (\inn{x,y}\frac{-\overline{\inn{x,y}}}{||y||^2})\\
        0&\leq ||x||^2-2\frac{|\inn{x,y}|^2}{||y||^2}+\frac{|\inn{x,y}|^2}{||y||^2}\\
        &\implies ||x||^2\geq \frac{|\inn{x,y}|^2}{||y||^2}
    \end{align*}
\end{proof}

\begin{fproposition}
    $||\cdot||$ define una norma $H$.
\end{fproposition}
\begin{proof}
    \begin{enumerate}
        \item Definidad $\checkmark$
        \item $||\lambda x||=\inn{\lambda x,\lambda x}^{1/2}=(\lambda \overline{\lambda}||x||^2)^{1/2}=|\lambda|\cdot ||x||$
        \item (Desigualdad triangular)
        
        \begin{align*}
            ||x+y||^2=||x||^2+2\Re (\inn{x,y})+||y||^2&\leq ||x||^2+2||x||\cdot ||y||+||y||^2\\
            &=(||x||+||y||)^2
        \end{align*}
    \end{enumerate}
\end{proof}

\begin{fproposition}
    $\inn{\cdot,\cdot}$ es continuo en $H\times H$
\end{fproposition}

\begin{proof}
    $x_n\to x$ en $||\cdot$ e $y_n\to y$ en $||\cdot||$

    \begin{align*}
        |\inn{x_n,y_n}-\inn{x,y}|&=|\inn{x_n-x,y_n}+\inn{x,y_n-y}|\\
        &\leq |\inn{x_n-x,y_n}|+|\inn{x,y_n-y}|\\
        &\leq ||x_n-x||\cdot ||y_n||+||x||\cdot ||y_n-y||\\
        &\xrightarrow[n\to\infty]{} 0
    \end{align*}
\end{proof}

\begin{fdefinition}
    Decimos que $X\perp Y$ en el espacio pre-Hilbertiano $H$ si $\inn{x,y}=0$. Si $E\subseteq H$ subconjunto, definimos el \textbf{espacio ortogonal}

    \[E^\perp :=\{x\in H:x\perp y\quad \forall y\in H\}\]
\end{fdefinition}

$E^\perp$ es un \textbf{subespacio} de $H$ y es cerrado:

$x_n\in E^\perp$ y $x_n\to x$ en $H$ entonces 

\[\inn{x,y}=\lim_{n\to\infty} \inn{x_n,y}=0\quad \forall y\in E\]

\begin{ftheorem}[Pitagoras]
    Si $x_1,\ldots,x_n\in H$ (pre-Hilbertiano) son mutuamente ortogonales, entonces 

    \[||x_1+\cdots+x_n||^2=\sum_{k=1}^n ||x_k||^2\]
\end{ftheorem}

\begin{fproposition}[Ley del paralelogramo]
    \[||x+y||^2+||x-y||^2=2||x||^2+2||y||^2\]
\end{fproposition}

\begin{proof}
    \begin{align*}
        ||x\pm y||^2=||x||^2\pm 2\Re \inn{x,y}+||y||^2
    \end{align*}

    Sumando los 2 términos (diagonales), estamos listos.
\end{proof}

\begin{fdefinition}
    Decimos que un espacio $(H,\inn{\cdot,\cdot})$ pre-Hilbertiano es un espacio de \textbf{Hilbert} si es \textbf{completo} respecto $||\cdot||$ inducida por $\inn{\cdot,\cdot}$
\end{fdefinition}

\begin{fexample}
    $(\C^n,\inn{\cdot,\cdot})$. $\inn{x,y}=\sum_{k=1}^n x_k\overline{y_k}$ es un espacio de Hilbert.
\end{fexample}

\begin{fexample}
    $(\ell^2,\inn{\cdot,\cdot})$. $\inn{\{x_k\},\{y_k\}}=\sum_{k=1}^\infty x_k\overline{y_k}$
\end{fexample}

¿$\ell^p$ tiene una estructura de espacio de Hilbert? $\iff p=2$