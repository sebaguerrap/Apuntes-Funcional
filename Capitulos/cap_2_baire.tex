\section{El teorema de Baire}

$(X,d)$ espacio métrico.

\[B_r(x)=\{y\in X:d(x,y)<r\}\]
\[\overline{B_r}(x)=\{y\in X:d(x,y)\leq r\}\]

$O\subseteq X$ es abierto si $\forall x\in O,\exists B_r(x)\in O$. $\bigcup_\alpha O_\alpha$ es abierto.

$F\subseteq X$ es cerrado si $F^c$ es abierto. $\bigcap_\alpha F_\alpha$ es cerrado.

\[\overline{E}=\bigcap_{F\supseteq E} F\]

\[\mathring{E}=\bigcup_{O\subseteq E}O\]

\[E\overset{denso}{\subseteq} X \text{ si }\overline{E}=X\]


\begin{fdefinition}
    a
    % $E\subseteq X$ es \textbf{denso en ninguna parte} si $\mathaccent{\overline{E}}=\varnothing$.
    \seba{arreglar}
\end{fdefinition}

esencialmente, denso en ninguna parte $E$ significa que $E$ no contiene bolas abiertas.

\begin{fexample}
    $E=\{x\}$ es denso en niguna parte.
\end{fexample}

\begin{fproposition}
    F es cerrado y denso en ninguna parte $\iff F^c$ es abierto y denso.
\end{fproposition}

\subsubsection*{La noción de categoria de Baire}

\begin{fdefinition}
    $E\subseteq X$ cat I si $E=\bigcup_{k}E_k$ donde $E_k$ es denso en ninguna parte.
\end{fdefinition}

\begin{fexample}
    $\mathbb{Q}$ es cat I.
\end{fexample}

\begin{fdefinition}
    Si $G$ tiene $G^c$ que es cat I, decimos que $G$ es \textbf{genérico}.
\end{fdefinition}

\begin{fdefinition}
    $E$ es de cat II si no es de primera categoría.
\end{fdefinition}

\paragraph*{Observaciones}

\begin{enumerate}
    \item Si $E$ es cat I, y $F\subseteq E$ es cat I
    
    \begin{align*}&F\subseteq E\subseteq \bigcup_k E_k\\
    &\implies F=\bigcup_k E_k\cap F,\quad \overline{E_k\cap F}\subseteq \overline{E_k}\\
    &\implies E_k\cap F \text{ son densos en niguna parte.}
    \end{align*}

    \item Si $\{E_k\}_{k\in\N}$ de cat I, $\bigcup_{k} E_k=\bigcup_{k}\bigcup_l \underbrace{E_{kl}}_{\text{denso en NP}}$ es una unión contable.
    
    \item No hay conexión entre conjuntos de cat I y conjuntos despreciables del punto de vista de teoría de la medida.
    \begin{fexample}
        $G_j=\bigcup_{n} (q_n-2^{-(n+j+1)},q_n+2^{-(n+j+1)})$

        $\{q_j\}$ enumeración de $\Q$.

        $G_j$ es abierto y denso en $\R$.

        \begin{align*}
            &\implies E_j=G_j^c \text{ es cerrado y denso en NP}\\
            &\implies E:=\bigcup_j E_j \text{ es cat I}
        \end{align*}

        y de plena medida en $\R$.

        $\iff E^c$ es de medida $0$ de Lebesgue.

        \begin{align*}
            |E^c|&=|\bigcap E_j^c|\\
            &=|\bigcap G_j|\leq |G_j|\\
            |G_j|&\leq \sum_{n=1}^\infty 2\cdot 2^{-(n+j+1)}\\
            &=2^{-j}\xrightarrow{j\to\infty} 0
        \end{align*}
    \end{fexample}
    
\end{enumerate}


\begin{ftheorem}[Teorema de Baire]
    Sea $(X,d)$ \textbf{completo}. Entonces, $X$ es de la cat II en sí mismo.
\end{ftheorem}

\begin{proof}
    Supongamos que $X$ es de cat I en sí:

    \[X=\bigcup_{k} \underbrace{E_k}_{\text{densos en NP}}=\bigcup_{k} \underbrace{\overline{E_k}}_{=F_k \text{ denso en NP y cerrado}}\]

    Llegaremos a una contradicción si demostramos que hay un $x\not\in F_k,\quad \forall k$.

    $F_1\neq X$. $\overline{B_{r_1}}(x_1)\subseteq F^c$, $\overline{B_{r_2}}(x_2)\subseteq F_2^c$.

    De esta manera obtenemos bolas cerradas $\overline{B_{r_k}}(x_k)$ tales que

    \begin{enumerate}
        \item \[\overline{B_{r_{k+1}}}(x_{k+1})\subseteq \overline{B_{r_k}}(x_k)\]
        \item \[\overline{B_{r_k}}(x_k)\subseteq F_k^c\]
        \item \[r_{k+1}\leq \frac{r_k}{2}\implies r_k\to 0\]
    \end{enumerate}

    $\{x_k\}$ es Cauchy pues:

    \[\forall k,l\geq n, x_k,x_l\in \overline{B_{r_n}}(x_n)\]

    \[\implies |x_k-x_l|\leq 2 r_n\xrightarrow{n\to\infty} 0\]

    \[\implies x_k\to x\in X\]

    Como $x_k\in \overline{B_{r_k}}\quad \forall k\geq n$,

    \[\implies x=\lim x_k\in \overline{B_{r_n}}(x_n)\subseteq F_n^c\]

    Por lo que $x\not\in F_n\quad \forall n$.
    
\end{proof}

\begin{fcorollary}\label{theo:2.3.2.1}
    $G\subseteq X$ es \textbf{genérico} $\implies$ denso en $X$, con $X$ completo.
\end{fcorollary}

\begin{proof}
    Asumimos que $G$ genérico no es denso, entonces hay una bola $B$

    \[\implies \overline{B}\subseteq G^c=\bigcup_k {E_k}\subseteq \bigcup \overline{E_k}\]

    \[\implies \overline{B}=\bigcup_{k} \underbrace{\overline{E_k}\cap \overline{B}}_{\text{cerrados y densos en NP}}\]

    Pero $\overline{B}$ es un espacio métrico completo, contradicción con el teorema de Baire.
\end{proof}

\begin{fcorollary}
    $X$ completo, $X=\bigcup_k F_k\leftarrow$ cerrado.

    Entonces, por lo menos uno $F_k$ contiene una bola.
\end{fcorollary}

\subsection{Aplicaciones}

\begin{ftheorem}
    El conjunto de funciones continuas en $[0,1]$ que no son derivables en nigún punto es \textbf{denso} en $C([0,1])$
\end{ftheorem}

\begin{proof}
    Sea $\mathcal{D}=\{f\in C([0,1]):f'(x_*) \text{ existe en un punto } x_*\in [0,1]\}$

    Queremos demostrar que $\mathcal{D}$ es cat I en $C([0,1])$.

    Por \ref{theo:2.3.2.1}, $\mathcal{D}^c$ es genérico $\implies$ denso en $C([0,1])$.

    Si $f\in \mathcal{D}\implies f'(x_*)$ existe

    \[\implies \lim_{x\to x_*} \frac{f(x)-f(x_*)}{x-x_*}\]

    existe.

    \[\implies |f(x)-f(x_*)|\leq M|x-x_*|\quad \forall x\in [0,1]\]

    para algún $M>0$.

    \[\implies \mathcal{D}\subseteq \bigcup_{N=1}^\infty E_N\]

    $E_N:=\{f\in C([0,1]):|f(x)-f(x_*)|\leq N|x-x_*|\text{ para algún $x_*\in[0,1]$}\}$

    Estaremos listos si probamos que:

    \begin{enumerate}
        \item $E_N$ es cerrado en $C([0,1])$
        \item $E_N$ es denso en ninguna parte.
    \end{enumerate}

    \begin{enumerate}
        \item $f_n\in E_N$ y $f_n\to f$, en $||\cdot||_\infty$.
        
        $[0,1]\ni x_n^*\to x^*$ (podemos extraer una subsucesión que converge)

        \[|f_n(x)-f_n(x_n^*)|\leq N|x-x_n^*|\quad \forall x\in[0,1]\]

        Queremos demostrar que 

        \[|f(x)-f(x^*)|\leq N|x-x^*|\]

        \begin{align*}
            |f(x)-f(x^*)|\leq \underbrace{|f(x)-f_n(x)|}_{\leq ||f-f_n||_\infty\leq \varepsilon/2}+|f_n(x)-f_n(x^*)|+\underbrace{|f_n(x^*)-f(x^*)|}_{\leq \varepsilon/3}
        \end{align*}

        \begin{align*}
            |f_n(x)-f_n(x^*)|&\leq |f_n(x)-f_n(x^*)|+|f_n(x_n^*)-f_n(x^*)|\\
            &\leq N|x-x_n^*|+N|x_n^*-x^*|\\
            &\leq N(|x-x^*|+|x^*-x_n^*|)+N|x_n^*-x^*|\\
            &\leq N|x-x^*|+\underbrace{2N|x_n^*-x^*|}_{\varepsilon/3}
        \end{align*}

        \item ¿Por qué $E_N$ es denso en NP de $X$?
        \[P_M=\{\text{funciones continuas en $[0,1]$ derivables a trozos, }|f'|=M\}\]

        son funciones zig-zag. Cuando $M>N$, $P_M\cap E_N=\varnothing$. Además, $P_M$ es denso en $C([0,1])$.
        Como consecuencia, $E_N$ no puede tener interior no trivial ya que $E_N$ no puede tener una bola abierta (hay funciones de $P_M$ en $E_N$ y $P_M$ es denso).

        Mostraremos que  $P_M$ es denso.

        \[P=\{\text{las funciones continuas lineales a tozos}\}\overset{denso}{\subseteq}C([0,1])\]

        Podemos aproximar cada $f\in P$ con una función $g\in P_M$ arbitrariamente bien.
    \end{enumerate}
\end{proof}

\subsubsection*{Teorema de la Aplicación Abierta y Teorema del grafo cerrado}

Sean $(X,||\cdot||_X),(Y,||\cdot||_Y)$ espacios de Banach.

\[T\in \mathcal{B}(X,Y)\implies T^{-1}(O)\overset{ab}{\subseteq}X\quad \forall O\overset{ab}{\subseteq}Y\]

Si $T$ es biyectiva adicionalmente, entonces $S:=T^{-1}$ es lineal (no necesariamente acotada).
Sin embargo, si $S$ es continua, entonces $S^{-1}(U)\overset{ab}{\subseteq},\forall U\overset{ab}{\subseteq}X$

\[\iff T(U)\overset{ab}{\subseteq}Y\quad \forall U\overset{ab}{\subseteq}X\]

\begin{fdefinition}
    Sea $T:X\to Y$ una aplicación. Decimos que $T$ es abierta si 
    
    \[T(U)\overset{ab}{\subseteq}Y\quad \forall U\overset{ab}{\subseteq}X\]
\end{fdefinition}

Si $T:X\to Y$ es lineal, continua y biyectiva, entonces $T^{-1}:Y\to X$ es lineal. ¿Es $T^{-1}$ continua?

Lo será cuando $T$ es abierta.

\begin{ftheorem}[Aplicación Abierta]
    Si $X,Y$ son espacios de Banach, $T\in \mathcal{B}(X,Y)$ y sobreyectiva, entonces $T$ es abierta.
\end{ftheorem}

\begin{fcorollary}
    Si $X,Y$ son espacios de Banach, $T\in\mathcal{B}(X,Y)$ es biyectiva, entonces $T^{-1}\in\mathcal{B}(Y,X)$. Existen $c,C>0$ tales que

    \[c||x||_X\leq ||\underbrace{Tx}_{y}||_Y\leq C||x||_X\quad \forall x\in X\]

    \[c||T^{-1}y||_X\leq ||y||_Y\]
\end{fcorollary}

\begin{proof}[Demostración del teorema \ref{theo:2.3.4}]
    \begin{enumerate}
        \item Será suficiente demostrar que $T(B_2^X\supseteq B_\delta^Y)$. ($B_r^X=B_r^X(0)$)
        
        Por linealidad

        \begin{align*}
            T(B_r^X(x))&=T(x+B_r^X)\\
            &=Tx+T(B_r^X)={y+\frac{r}{2}T(B_2^X)}\\
            &\supseteq y+\frac{r}{2}B_\delta^Y=B_{\frac{\delta r}{2}}^Y(y)\\
        \end{align*}

        \item Vamos a demostrar que $\overline{T(B_1^X)}\supseteq B_\delta^X$ para algún $\delta>0$
        
        Por la sobreyectividad:

        \[cat II\rightarrow Y=\bigcup_{n=1}^\infty \overline{T(B_n^X)}\]

        Entonces, $T(B_n^X)\supseteq B_r^Y(y)$ para algún $n\in\N,r>0,y\in Y$. Tomamos $\tilde y$ tal que $|\tilde y-y|\leq \frac{r}{2}$ e $\tilde y=Tx$ para algún $x\in B_n^X$.

        \begin{align*}
            T(B_{2n}^x(\tilde x))\supseteq \overline{T(B_n^X)}\supseteq B_r^Y(y)\supseteq B_{\frac{r}{2}}^Y(y)
        \end{align*}

        Restando $Tx$

        \[T(B_{2n}^X)\supseteq B_{\frac{r}{2}}^X\]

        Reescalando

        \[\overline{T(B_1^X)}\supseteq B_{\frac{r}{4n}^Y}\quad \delta=\frac{r}{4n}\]

        \item Tenemos $T(B_1^X)\supseteq B_\delta^Y$. Reescalando
        
        \[\overline{T(B_{2^{-k}}^X)}\supseteq B_{\delta 2^{-k}}^Y\]

        ¿Por qué $T(B_2^X)\supseteq B_\delta^Y$?

        Fije $y_0\in B_\delta^Y$. Podemos encontrar $x_0\in B_1^X$ tal que 

        \[||y_0-Tx_0||_Y<\frac{\delta}{2}\]

        \[\implies y_1:=y_0-Tx_0\in B_{\delta/2}^Y\]

        $\implies$ existe $x_1\in B_{\frac{1}{2}}^X$ tal que 

        \[||y_1-Tx_1||<\frac{\delta}{4}\]

        De esta manera construimos sucesiones $\{x_n\},\{y_n\}$, tales que

        \begin{enumerate}
            \item $x_n\in B_{2^{-n}}^X, y_n\in B_{\delta 2^{-n}}^Y$
            \item $y_{n+1}=y_n-Tx_n$
        \end{enumerate}

        $x:=\displaystyle\sum_{n=0}^\infty x_n\in X$ porque $X$ es Banach. Veremos que $Tx=y$ y $x\in B_2^X$.
    \end{enumerate}
\end{proof}