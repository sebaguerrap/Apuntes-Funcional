\chapter{skere}

We are interested in solving the Dirichlet boundary value problem.

\[\begin{cases}
    -u''(x)+V(x)u(x)=f(x) &x\in [0,1]\\
    u(0)=u(1)=0
\end{cases}\]

\[\begin{cases}
    -\Delta u+V(x)u=f(x) &\text{in }\Omega\subseteq \R^n\\
    u|_{\partial\Omega}=0
\end{cases}\]

\underline{Assumptions}:\begin{enumerate}
    \item $V(x)$ is a real-values continuous function
    \item $f(x)$ is a given continuous function on $[0,1]$.
\end{enumerate}

\begin{ftheorem}
    Under the assumptions and $V\geq 0$, the ODE (primera ecuación) has a \textbf{unique} solution $u\in C^2([0,1])$.
\end{ftheorem}

\begin{fproposition}
    Under the assumptions of Theorem, if $u_1,u_2\in C^2([0,1])$ solve the ODE $\implies u_1=u_2$.
\end{fproposition}

\begin{proof}
    $u=u_1u_2\implies u$ satisfies the ODE with $f\equiv 0$
    
    \begin{align*}
        0=\int (-u''(x)+Vu)\overline{u}(x)\,dx&=\int -u''\overline{u}\,dx+\int V|u|^2\,dx\geq 0\\
        &=\cancelto{0}{-\overline{u}u'\big\rvert_{x=0}^{x=1}}+\int u'\overline{u}'\,dx+\int V|u|^2\,dx\\
        =\int |u'|^2\,dx+\int V|u|^2\,dx
    \end{align*}

    \[\implies u'\equiv 0\implies u=const\overset{BC}{\implies} u\equiv 0\]
\end{proof}

Let us first treat $V\equiv 0$.

\[\begin{cases}
    -u''=f(x)\\
    u(0)=u(1)=0
\end{cases}\]

The solution is given by 

\begin{align*}
    u(x)&=A f(x)\coloneqq (1-x)\int_0^x y f(y)\,dy+x\int_x^1 (1-y) f(y)\,dy\\
    &=\int_0^1 K(x,y)f(y)
\end{align*}

where 

\[K(x,y)=\begin{cases}
    (1-x)y, &0\leq y\leq x\leq 1\\
    (1-y)x, &0\leq x\leq y\leq 1
\end{cases}\]

is the Green's function for the ODE.

Since $K(x,y)\in C([0,1]^2)\subseteq L^2([0,1]^2)$ we can think of $A=T_k$ as a Hilbert-Schmidt operator.

\[T_k=A:L^2([0,1])\to L^2([0,1])\]

is \textbf{compact}. Furthermore $A$ is self-adjoint:

\begin{align*}\inn{Af,g}&=\int\left(\int K(x,y)f(y)\,dy\right)\overline{g}(x)\,dx=\int f(y)\int K(x,y)\overline{g}(x)\,dxdy\\
&=\int f(y)\overline{\int K(x,y)g(x)\,dx}dy\\
&=\int f(y)\overline{A g(y)}\,dy\\
&=\inn{f,Ag}\end{align*}

\begin{fproposition}
    $A(L^2([0,1]))\subseteq C([0,1])$
\end{fproposition}

\begin{proof}
    \begin{align*}
        |A f(x)|&\leq \int_0^1 |K(x,y)||f(y)|\,dy\\
        &\leq M\int_0^1 |f(y)|\,dy\\
        &\overset{CS}{\leq} M\left(\int_0^1 |f(y)|^2\right)^{1/2}\\
        &=M||f||_2
    \end{align*}

    \begin{align*}
        |Af(x)-Af(z)|&=\left|\int K(x,y)f\,dy-\int K(z,y)f(y)\,dy\right|\\
        &\leq \int |K(x,y)-K(z,y)||f(y)|\,dy\\
        &\leq \sup_{y\in[0,1]} |K(x,y)-K(z,y)|||f||_{2}
    \end{align*}
\end{proof}

How do we now solve 

\[\begin{straightcases}
    -u''+V(x)u=f(x)\\
    u(0)=u(1)=0
\end{straightcases}\]

\[-u''=f(x)-V(x)u\]
\[\implies u(x)=A(f-Vu)\]
\[\implies u(x)+A(Vu)=A(f)\]

$m_V u= Vu$. When $V$ is continuous

\[m_V:L^2\to L^2\]

\[(I+\underbrace{Am_V}_{\tilde T \text{compact}})u=Af\]

\seba{Última clase skere}

$V,f\in C([0,1]),V\geq 0$
\begin{equation}\begin{straightcases}
    -u''+Vu=f\\
    u(0)=u(1)=0
\end{straightcases}\end{equation}

\[\begin{straightcases}
    -u''=f\\
    u(0)=u(1)=0
\end{straightcases}\]

con solución $u=Af$. $A$ is compact and self-adjoint. When $f\in L^2\implies Af\in C$. $f\in C\implies Af\in C^2$

Podemos escribir (5.1) como 

$$-u''=A(f-Vu)$$
\begin{equation}(I+Am_V)u=Af\end{equation}

\begin{fproposition}
    If $u\in L^2$ solves (5.2) then $u\in C^2([0,1])$ is a solution of (5.1)
\end{fproposition}

\begin{proof}
    \[u=Af-Am_V u\]

    $Af\in C, Am_V u\in C$

    \[\implies u\in C([0,1])\]
    \[\implies m_V u\in C([0,1])\]
    \[\implies A(m_V u)\in C^2\text{ and }Af\in C^2\]
    \[\implies u\in C^2\]

    (Bootstrap argument)
\end{proof}

Q. Can we invert $I+\tilde T$?\\
A. $I+\tilde T$ is invertible $\iff \ker (I+\tilde T)=\{0\}$ (Fredholm Alternative)

\[(I+\tilde T)v=0\]
\[AVv=v\]

\underline{One idea}: $\inn{AVv,v}=-\inn{v,v}=-\|v\|^2$

\underline{Issue} $Am_V$ is not self-adjoint!

\[(Am_V)^*=m_V^*A^*=m_VA\neq Am_V\]

\underline{Idea} $(I+Am_V)u=Af$. Perhaps, $A=A^{1/2}A^{1/2}$ and if $u=A^{1/2}V$, then 

\[(I-Am_V)A^{1/2}v=Af\]
\[A^{1/2}v+Am_VA^{1/2}v=Af\]
\[A^{1/2}(I+A^{1/2}m_VA^{1/2})v=Af\]

So if we manage to solve 

\begin{equation}
    (I+\underbrace{A^{1/2}m_VA^{1/2}}_{T})v=A^{1/2}f
\end{equation}

we would be done!

$u=A^{1/2}v$ would then solve (5.3) and yay!

$T$ formally is self-adjoint:

\[(A^{1/2}m_V A^{1/2})^*=(A^{1/2})^*(m_V)^*(A^{1/2})^*=a^{1/2}m_V A^{1/2}\]

Why can we take the $\sqrt{\cdot}$ of $A$?

\begin{fproposition}
    $\ker(A)=\{0\}$ and the orthonormal eigenvectors of $A$ are $B=\{v_n=\sqrt{2}\sin(n\pi x)\}_{n=1}^\infty$ associated with $\lambda_n=\frac{1}{n^2\pi^2}$
\end{fproposition}

\begin{fremark}
    $B$ is an orthonormal basis (by the spectral theorem) for $L^2([0,1])$. ($\to$ reproduces result Fourier basis on $[-\pi,\pi]$)
\end{fremark}

\begin{proof}
    $\ker A=(\im A^*)^\perp=(\im A)^\perp$

    $\ker A=\{0\}\iff \im A$ dense in $L^2$. $f\in C\implies u=Af$ is the unique solution of (5.1). $u\in C^\infty_c([0,1])\implies u=Af''=A(-u'')$

    \seba{fue cabros me dio lata anotar esta demostración}
\end{proof}

\begin{fproposition}
    $A^{1/2}$ is a self-adjoint compact operator.
\end{fproposition}

\begin{proof}
    \[S_Nx\coloneqq\sum_{n=1}^N \lambda_n^{1/2}\inn{x,v_n}v_n\]

    finite-rank$\implies$ compact

    \seba{nuevamente me dio lata terminar la demostración}
\end{proof}

\begin{fproposition}
    \[A^{1/2}A^{1/2}=A\]
    \[(I+T)v=A^{1/2}f\]
    \[T\coloneqq A^{1/2}m_V A^{1/2}\]
\end{fproposition}

\begin{fproposition}
    $I+T$ has a continuous inverse
\end{fproposition}

\begin{proof}
    By Fredholm Alternative, it suffices to show 

    \[(I+T)v=0\]

    only has $v=0$ as a solution.

    \seba{El resto se deja como ejercicio para el lector}
\end{proof}