\chapter{Espacios de Banach}

\section{Nociones básicas}

\begin{fdefinition}[Espacios métricos]
    Un espacio métrico $(X,d)$ y $d:X\times X\to [0,\infty)$ la métrica que satisface:

    \begin{enumerate}
        \item $d(x,y)=0\iff x=y$
        \item (simetría) $d(x,y)=d(y,x)$
        \item (Desigualdad triangular) $d(x,y)\leq d(x,z)+d(z,y)$
    \end{enumerate}
\end{fdefinition}

\begin{fdefinition}[]
    Sea $V$ un espacio vectorial (sobre $\R$ o $\C$). Una norma en $V$ es una función $||\cdot||:V\to [0,\infty)$ que satsiface:

    \begin{enumerate}
        \item $||v||=0\iff v=0$
        \item $||\lambda v||=|\lambda|\cdot ||v||$
        \item (Desigualdad triangular) $||v+w||\leq ||v||+||w||$
    \end{enumerate}
\end{fdefinition}

Una función $||\cdot||:V\to[0,\infty)$ que satisface solo $2.$ y $3.$ se llama \color{red} semi-norma \color{black}.

Una espacio vectorial $V$ con una norma se llama \color{red} Espacio normado \color{black} $(V,||\cdot||)$.

\begin{fproposition}
    $(V,||\cdot||)$ define un espacio métrico con métrica $d(v,w):=||v-w||$.
\end{fproposition}

\begin{fexample}
    \begin{itemize}
        \item $V=\R^n,\C^n$ tiene la estructura de espacio normado:

    \[|x|_2:=\left(\sum_{k=1}^n |x_k|^2\right)^{1/2},\quad x=(x_1,\ldots,x_n)\]

    \item En $\R^2$, $|(x_1,x_2)|:=|x_1|$ define una semi-norma:


    \[|(x_1,x_2)|=0\iff x_1=0,x_2\in\R\]

    \item $|x|_\infty=\displaystyle\max_{k=1,\ldots,n}\{x_k\}$ es una norma.

    \item \[|x|_p:=\left(\sum_{k=1}^n |x_k|^p\right)^{1/p},\quad p\in [1,\infty)\]
    \end{itemize}

\seba{Añadir dibujos de norma infinito y norma 1}


\end{fexample}

\begin{fproposition}

En $\R^n$ y $\C^n$ todas normas son equivalentes: si $||\cdot||_1,||\cdot||_2$ son 2 normas, existe $c>0$ tal que

\[\frac{1}{c}||v||_2\leq ||v||_1\leq c||v||_2,\quad \forall v\in V\]

\end{fproposition}

\begin{fdefinition}
    Sea $X$ un espacio métrico. Definimos 

    \[C_\infty(X):=\{f:X\to\C\text{ continuas y acotadas}\}\]

\end{fdefinition}

\begin{fexample}
    $C_\infty([0,1]) = C([0,1])$ (funciones continuas)
\end{fexample}

\begin{fproposition}
    $||f||_\infty:=\displaystyle\sup_{x\in X}|f(x)|$ define una norma en $C_\infty(X)$.
\end{fproposition}

\begin{proof}
    \begin{enumerate}
        \item $||f||_\infty=0\iff f(x)=0\, \forall x\in X$.

        \item \begin{align*}||\lambda f||_\infty&=\sup_x |\lambda f(x)|\\
            &=\sup_x |\lambda|\cdot |f(x)|\\
            &=|\lambda|\cdot ||f||_\infty
        \end{align*}

        \item \begin{align*}
            |f_1(x)+f_2(x)|&\leq |f_1(x)|+|f_2(x)|\\
            &\leq ||f_1||_\infty+||f_2||_\infty
        \end{align*}
    \end{enumerate}
\end{proof}

Convergencia en $||\cdot||_\infty$

\[f_n\to f,\quad \text{en }C_\infty(X)\]

si 

\[||f_n-f||_\infty\xrightarrow{n\to\infty} 0\]

\[\iff \forall \varepsilon>0 \exists N\in\N\text{ tal que}\]

\[||f_n-f||_\infty<\varepsilon,\quad \forall n\geq N\]

\[\iff |f_n(x)-f(x)|<\varepsilon,\quad \forall x\in X\]

\begin{fexample}
    $\K=\R$ o $\C$.

    \[\ell^p(\K):=\{\{a_k\}_k \subseteq \K: ||a||_p<\infty\}\]

    donde

    \[||a||_p:=\begin{cases}\left(\displaystyle\sum_{k=1}^\infty |a_k|^p\right)^{1/p}&p\in [1,\infty)\\
    \displaystyle\sup_{k\in\N} |a_k|& p=\infty\end{cases}\]
\end{fexample}

Sea $(X,\mathcal{B},\sigma)$ un espacio de medida.

\[L^p(x,\sigma):=\{f:X\to\K \, \sigma\text{-medibles, tales que} ||f||_{L^p}<\infty\}\]

donde 

\[||f||_{L^p}:= \left(\int |f|^p\,d\sigma\right)^{1/p}\]

\[||f||_{L^\infty}:=\esssup_x |f|\]

\begin{fexample}
    $X=[0,1]$, $\sigma=$ medida de Lebesgue. En $C([0,1])$ definimos

    \[||f||_\infty =\sup |f(x)|\]
    \[||f||_{L^1}=\int |f(x)|\,dx\]

    Estas 2 normas \color{red} no son equivalentes \color{black}
\end{fexample}

\begin{fdefinition}
    Un espacio normado $(V,||\cdot||)$ es un espacio de Banach si es \color{red} completo \color{black} con respecto a la métrica inducida.
\end{fdefinition}

\begin{fexample}
    $\R^n,\C^n$ son espacios de Banach (con respecto a cualquier norma)
    
    $L^p(X,\mathcal{B},\sigma)$ es un espacio de Banach (cuando $(X,\mathcal{B},\sigma)$ es completo).
\end{fexample}

\begin{fproposition}
    $C_\infty(X)$ es un espacio de Banach.
\end{fproposition}

\begin{proof}
    $\{f_n\}\subseteq V=C_\infty(X)$ de Cauchy.

    \begin{enumerate}
        \item Adivinar el límite $f$.
        \item Probar la convergencia:
        \[||f_n-f||\to 0\]
        \item $f$ está en el espacio.
    \end{enumerate}

    $\forall \varepsilon>0\exists N=N(\varepsilon)$ tal que

    \[||f_n-f_m||_\infty\leq \varepsilon,\quad \forall n,m\geq N\]

    Para todo $x\in X$ fijo, tenemos entonces 

    \[|f_n(x)-f_m(x)|\leq ||f_n-f_m||_\infty\leq \varepsilon\]

    Esto es $\{f_n(x)\}_n$ es Cauchy en $\C$.

    \[\implies f(x):=\lim_{n\to\infty} f_n(x) \text{ existe}\]

    \begin{align*}|f_n(x)-f(x)|&=\lim_{m\to\infty} |f_n(x)-f_m(x)|\\
    &\leq \varepsilon \quad \forall n\geq N(\varepsilon) \text{ independiente de }x\in X
\end{align*}

    \[\implies ||f_n-f||_\infty<\varepsilon,\quad \forall n\geq N(\varepsilon)\]

    Esto es $f_n\to f$ uniformemente sobre $X$.

    $\implies f$ es continua sobre $X$. 

    ¿Por qué $f$ es acotada?

    Considere $\varepsilon=1$

    \[\implies ||f_n-f_{\bar N}||_\infty \leq 1\]

    cuando $n\geq \bar N:=N(1)$.

    \begin{align*}||f_n||_\infty &\leq ||f_{\bar N}||_\infty + ||f_n-f_{\bar N}||_\infty\\
    &\leq ||f_{\bar N}||_\infty +1\end{align*}

    \[\implies f(x)=\lim_{n\to\infty} f_n(x)\text{ es acotada}\]

    \begin{fdefinition}
        Sea $(V,||\cdot||)$ un espacio normado. $v_n\in V,n\in\N$. $\displaystyle\sum_{n=1}^\infty v_n$ es \color{red} sumable \color{black} si

        \[S_m=\sum_{n=1}^m v_n\]

        converge.

        $\displaystyle\sum_{n} v_n$ es \color{red} absolutamente sumable \color{black} si 

        \[\sum_{n=1}^\infty ||v_n||\]

        converge.
    \end{fdefinition}
\end{proof}

\begin{fproposition}
    Si $\displaystyle\sum_{n=1}^\infty v_n$ es absolutamente sumable, entonces, $\{S_m\}$ es Cauchy
\end{fproposition}

\begin{ftheorem}
    Un espacio normado $(V,||\cdot||)$ es un espacio de Banach si y solo si toda serie absolutamente sumable es sumable.
\end{ftheorem}

\begin{proof}
    $\impliedby:$ \begin{enumerate}
        \item Tome una sucesión $\{v_n\}$ de Cauchy. Es suficiente demostrar que una subsucesión converge. $v_{n_k}\to v$ en $V$. Fije $\varepsilon>0$.
        $\implies ||v_m-v||\leq \underbrace{||v_m-v_{n_k}||}_{\leq \varepsilon/2}+\underbrace{||v_{n_k}-v||}_{\leq \varepsilon/2}\leq \varepsilon$, tomando $k,m$ suficientemente grandes.

        \item Dos trucos: Podemos ``acelerar'' la convergencia. Existe una subsucesión $\{v_{n_k}\}$ tal que 
        
        \begin{equation}||v_{n_{k+1}}-v_{n_k}||\leq 2^{-k}\end{equation}

        \[||v_n-v_m||\leq 2^{-k}\quad \forall n,m\geq N(2^{-k}):=N_k\]

        $n_k:=N_1+\ldots+N_k$

        Afirmamos que $\{v_{n_k}\}$ converge.

        Truco de la suma telescopica.

        \[\sum_{k=1}^\infty (v_{n_{k+1}}-v_{n_k})\]

        es absolutamente sumable debido a (1.1) entonces es sumable:

        \[\sum_{k=1}^m (v_{n_{k+1}}-v_{n_k})\xrightarrow{m\to\infty} S\in V\]

        Sumas parciales convergen 

        \[v_{n_{m+1}}-v_{n_1}\xrightarrow{m\to\infty}S\in V\]

        \[\implies v_{n_{m+1}}\xrightarrow{m\to\infty}S+v_{n_1}\in V\]

    \end{enumerate}
\end{proof}

\section{Operadores y funcionales}

Nos interesan las aplicaciones lineales entre espacios normados.

\begin{fexample}
    \begin{align*}T:C([0,1],\C)&\to C([0,1],\C)\\
    f&\to F(x)=\int_0^x f(y)\, dy\end{align*}

    $T$ es lineal.

    \[F(x)=\int_0^1 \mathds{1}_{\{y<x\}} f(y)\,dy\]
\end{fexample}

\begin{fdefinition}
    $V,W$ son 2 espacios vectoriales.

    $T:V\to W$ es lineal si 

    \[T(\lambda_1 v_1+\lambda_2 v_2)=\lambda_1 T(v_1)+\lambda_2 T(v_2)\quad \forall v_1,v_2\in V\text{ y } \lambda_1,\lambda_2\in \K\]

\end{fdefinition}

\begin{align*}T:C([0,1])&\to C([0,1])\\
f&\to \int_0^1 \underbrace{K(x,y)}_{\text{Kernel}} f(y)\,dy:=Tf(x)\end{align*}

operador integral. Cuando $K\in C([0,1]^2), T$ está bien definida.

En $\dimn \infty$ vamos a exigir que los operadores lineales sean \color{red} continuos \color{black}.

\begin{fdefinition}
    $T:V\to W$, $V,W$ son espacios métricos. Decimos que $T$ es continuo si
    
    \[T^{-1}(O)\overset{ab}{\subseteq} V,\, \forall O\overset{ab}{\subseteq} V\]

    \[\iff T^{-1}(C)\overset{cerr}{\subseteq} V\quad \forall C\overset{cerr}{\subseteq} W\]
    $\iff v_n\to v$ en $V$ entonces $Tv_n\to Tv$ en $W$.
\end{fdefinition}

\begin{ftheorem}
    Sean $V,W$ espacios normados. Entonces $T:V\to W$ operador lineal es continuo si y solo si

    \begin{equation}||Tv||_W\leq C||v||\quad \forall v\in V\end{equation}

    para alguna constante $C$.
\end{ftheorem}

\begin{fdefinition}
    Operador lineal que satisface $1.2$ se llama \color{red} acotado \color{black}.
\end{fdefinition}

\begin{proof}
    $\implies:$ Sea $T$ continuo. $B:=\{||w||_W<1\}$

    $0\in T^{-1}(B)=B_r^v$

    \[T^{-1}(B)\supseteq B_r^v:=\{v\in V:||v||_V<r\}\]

    pues $T^{-1}(B)$ es abierto 

    \[\implies T^{-1}(B)\supseteq \{v\in V:||v||_V=\frac{r}{2}\}\]

    esfera de radio $\frac{r}{2}$.

    \[||T\bar v||_W<1\]

    Todo $v\in V,v\neq 0$ se puede escribir como $v=\frac{\bar v}{r/2}||v||_V$

    Para algún $\bar v\in S_{r/2}^v$

    Por lo tanto

    \[||Tv||_W=||T(\frac{\bar v }{r/2}||v||_V)||_W\]

    \[=||\frac{2}{r}||v||_V T(\bar v)||_W\]
    \[=\frac{2}{r}||v||_V ||T\bar v||_W<1\]
    \[\leq \frac{2}{r}||v||_V\quad \forall v\neq 0\]
\end{proof}

\begin{fexample}
    \[Tf(x):=\int_0^1 K(x,y)f(y)\,dy\]

    es acotado en $(C([0,1]),||||_\infty )$

    \begin{align*}|Tf(x)|&\leq \int_0^1 \underbrace{|K(x,y)|}_{\leq M}|f(y)|\,dy\\
    &\leq M\int_0^1 |f(y)|\,dy\leq M ||f||_\infty\quad \forall x\implies ||Tf||_\infty \leq M||f||_\infty\end{align*}
\end{fexample}

\begin{fdefinition}
    Sean $V,W$ espacios normados. Defina $\mathcal{B}(V,W)$ como el conjunto de operadores lineales continuos acotados de $V$ a $W$.

    Obviamente $\mathcal{B}(V,W)$ es un espacio vectorial.
\end{fdefinition}

Norma operador $T:V\to W$:

\[||T||:=\sup_{||v||=1} ||Tv||\]

Obviamente, $T\in \mathcal{B}(V,W), ||T||<\infty$

\[||Tv||\leq C \underbrace{||v||}_1=C\]

\[\implies ||T||\leq C\]

De hecho, para $T\in \mathcal{B}(V,W)$

\begin{align*}
    ||T||&=\sup_{v\neq 0} \frac{||Tv||}{||v||}=\sup_{||v||\leq 1} ||Tv||\\
    &=\inf \{C>0:||Tv||\leq C||v||\quad \forall v\in V\}
\end{align*}

Tenemos $||Tv||\leq ||T||||v||$

\begin{ftheorem}
    $\mathcal{B}(V,W)$ es un espacio normado bajo la norma operador.
\end{ftheorem}

\begin{proof}
    \begin{enumerate}
        \item $||T||=0\implies ||Tv||=0\forall v\in V$
         
        $\implies Tv=0\implies T=0$.

        \item $||\lambda T||=|\lambda| ||T||$

        \item Sea $v\in V,||v||=1$. $\forall T,S\in\mathcal{B}(V,W),$
        
        \begin{align*}||(T+S)v||&=||Tv+Sv||\\
        &\leq ||Tv||+||Sv||\\
        &\leq ||T|| ||v||+||S|| ||v||=(||T||+||S||)||v||\end{align*}

    \begin{align*}
        &\implies ||(T+S)v||\leq ||T||+||S||\\
        &\implies ||T+S||\leq ||T||+||S||
    \end{align*}
    \end{enumerate}
\end{proof}

    ¿Cuándo es $\mathcal{B}(V,W)$ completo?

    \begin{ftheorem}
        $\mathcal{B}(V,W)$ es Banach cuando $W$ es Banach.
    \end{ftheorem}

    \subsection{Aplicaciones}

    \begin{fdefinition}
        Sea $V$ un espacio normado sobre $\K$.

        \[V^*=\mathcal{B}(V,\K)\]

        se llama el espacio \color{red} dual \color{black} de $V$.
    \end{fdefinition}

    \begin{ftheorem}
        Cuando $\K=\R,\C$ (completos) $V^*$ es un espacio de Banach
    \end{ftheorem}

    Elementos de $V^*$ se llaman \color{red} funcionales \color{black} en $V$.


    \begin{fexample}
        $[\ell^p(\C)]^* =?,p\in[1,\infty)$

        Resulta que $?=l^q(\C)$ donde $\frac{1}{p}+\frac{1}{q}=1$.

        Si $v\in \ell^p,w\in\ell^q$

        podemos definir un funcional en $\ell^p$

        \begin{align*}\ell_w:\ell^p(\C)&\to\C\\
        v=\{v_k\}&\to \sum_{k=1}^\infty v_k\bar w_k\end{align*}

        \[|\ell_w|\leq ||w||_{\ell^q}||v||_{\ell^p}\]

        Es la desigualdad de Hölder discreta.
    \end{fexample}

    $(\ell^1)^*\simeq \ell^\infty$
    $(\ell^2)^*\simeq \ell^2$

    \begin{fnote}
        $(\ell^\infty)^*\not\simeq \ell^1$
    \end{fnote}
