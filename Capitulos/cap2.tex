\chapter{Espacios de Banach}

\section{Nociones básicas}

\begin{fdefinition}[Espacios métricos]
    Un espacio métrico $(X,d)$ y $d:X\times X\to [0,\infty)$ la métrica que satisface:

    \begin{enumerate}
        \item $d(x,y)=0\iff x=y$
        \item (simetría) $d(x,y)=d(y,x)$
        \item (Desigualdad triangular) $d(x,y)\leq d(x,z)+d(z,y)$
    \end{enumerate}
\end{fdefinition}

\begin{fdefinition}[]
    Sea $V$ un espacio vectorial (sobre $\R$ o $\C$). Una norma en $V$ es una función $||\cdot||:V\to [0,\infty)$ que satsiface:

    \begin{enumerate}
        \item $||v||=0\iff v=0$
        \item $||\lambda v||=|\lambda|\cdot ||v||$
        \item (Desigualdad triangular) $||v+w||\leq ||v||+||w||$
    \end{enumerate}
\end{fdefinition}

Una función $||\cdot||:V\to[0,\infty)$ que satisface solo $2.$ y $3.$ se llama \color{red} semi-norma \color{black}.

Una espacio vectorial $V$ con una norma se llama \color{red} Espacio normado \color{black} $(V,||\cdot||)$.

\begin{fproposition}
    $(V,||\cdot||)$ define un espacio métrico con métrica $d(v,w):=||v-w||$.
\end{fproposition}

\begin{fexample}
    \begin{itemize}
        \item $V=\R^n,\C^n$ tiene la estructura de espacio normado:

    \[|x|_2:=\left(\sum_{k=1}^n |x_k|^2\right)^{1/2},\quad x=(x_1,\ldots,x_n)\]

    \item En $\R^2$, $|(x_1,x_2)|:=|x_1|$ define una semi-norma:


    \[|(x_1,x_2)|=0\iff x_1=0,x_2\in\R\]

    \item $|x|_\infty=\displaystyle\max_{k=1,\ldots,n}\{x_k\}$ es una norma.

    \item \[|x|_p:=\left(\sum_{k=1}^n |x_k|^p\right)^{1/p},\quad p\in [1,\infty)\]
    \end{itemize}

\seba{Añadir dibujos de norma infinito y norma 1}


\end{fexample}

\begin{fproposition}

En $\R^n$ y $\C^n$ todas normas son equivalentes: si $||\cdot||_1,||\cdot||_2$ son 2 normas, existe $c>0$ tal que

\[\frac{1}{c}||v||_2\leq ||v||_1\leq c||v||_2,\quad \forall v\in V\]

\end{fproposition}

\begin{fdefinition}
    Sea $X$ un espacio métrico. Definimos 

    \[C_\infty(X):=\{f:X\to\C\text{ continuas y acotadas}\}\]

\end{fdefinition}

\begin{fexample}
    $C_\infty([0,1]) = C([0,1])$ (funciones continuas)
\end{fexample}

\begin{fproposition}
    $||f||_\infty:=\displaystyle\sup_{x\in X}|f(x)|$ define una norma en $C_\infty(X)$.
\end{fproposition}

\begin{proof}
    \begin{enumerate}
        \item $||f||_\infty=0\iff f(x)=0\, \forall x\in X$.

        \item \begin{align*}||\lambda f||_\infty&=\sup_x |\lambda f(x)|\\
            &=\sup_x |\lambda|\cdot |f(x)|\\
            &=|\lambda|\cdot ||f||_\infty
        \end{align*}

        \item \begin{align*}
            |f_1(x)+f_2(x)|&\leq |f_1(x)|+|f_2(x)|\\
            &\leq ||f_1||_\infty+||f_2||_\infty
        \end{align*}
    \end{enumerate}
\end{proof}

Convergencia en $||\cdot||_\infty$

\[f_n\to f,\quad \text{en }C_\infty(X)\]

si 

\[||f_n-f||_\infty\xrightarrow{n\to\infty} 0\]

\[\iff \forall \varepsilon>0 \exists N\in\N\text{ tal que}\]

\[||f_n-f||_\infty<\varepsilon,\quad \forall n\geq N\]

\[\iff |f_n(x)-f(x)|<\varepsilon,\quad \forall x\in X\]

\begin{fexample}
    $\K=\R$ o $\C$.

    \[\ell^p(\K):=\{\{a_k\}_k \subseteq \K: ||a||_p<\infty\}\]

    donde

    \[||a||_p:=\begin{cases}\left(\displaystyle\sum_{k=1}^\infty |a_k|^p\right)^{1/p}&p\in [1,\infty)\\
    \displaystyle\sup_{k\in\N} |a_k|& p=\infty\end{cases}\]
\end{fexample}

Sea $(X,\mathcal{B},\sigma)$ un espacio de medida.

\[L^p(x,\sigma):=\{f:X\to\K \, \sigma\text{-medibles, tales que} ||f||_{L^p}<\infty\}\]

donde 

\[||f||_{L^p}:= \left(\int |f|^p\,d\sigma\right)^{1/p}\]

\[||f||_{L^\infty}:=\esssup_x |f|\]

\begin{fexample}
    $X=[0,1]$, $\sigma=$ medida de Lebesgue. En $C([0,1])$ definimos

    \[||f||_\infty =\sup |f(x)|\]
    \[||f||_{L^1}=\int |f(x)|\,dx\]

    Estas 2 normas \color{red} no son equivalentes \color{black}
\end{fexample}

\begin{fdefinition}
    Un espacio normado $(V,||\cdot||)$ es un espacio de Banach si es \color{red} completo \color{black} con respecto a la métrica inducida.
\end{fdefinition}

\begin{fexample}
    $\R^n,\C^n$ son espacios de Banach (con respecto a cualquier norma)
    
    $L^p(X,\mathcal{B},\sigma)$ es un espacio de Banach (cuando $(X,\mathcal{B},\sigma)$ es completo).
\end{fexample}

\begin{fproposition}
    $C_\infty(X)$ es un espacio de Banach.
\end{fproposition}

\begin{proof}
    $\{f_n\}\subseteq V=C_\infty(X)$ de Cauchy.

    \begin{enumerate}
        \item Adivinar el límite $f$.
        \item Probar la convergencia:
        \[||f_n-f||\to 0\]
        \item $f$ está en el espacio.
    \end{enumerate}

    $\forall \varepsilon>0\exists N=N(\varepsilon)$ tal que

    \[||f_n-f_m||_\infty\leq \varepsilon,\quad \forall n,m\geq N\]

    Para todo $x\in X$ fijo, tenemos entonces 

    \[|f_n(x)-f_m(x)|\leq ||f_n-f_m||_\infty\leq \varepsilon\]

    Esto es $\{f_n(x)\}_n$ es Cauchy en $\C$.

    \[\implies f(x):=\lim_{n\to\infty} f_n(x) \text{ existe}\]

    \begin{align*}|f_n(x)-f(x)|&=\lim_{m\to\infty} |f_n(x)-f_m(x)|\\
    &\leq \varepsilon \quad \forall n\geq N(\varepsilon) \text{ independiente de }x\in X
\end{align*}

    \[\implies ||f_n-f||_\infty<\varepsilon,\quad \forall n\geq N(\varepsilon)\]

    Esto es $f_n\to f$ uniformemente sobre $X$.

    $\implies f$ es continua sobre $X$. 

    ¿Por qué $f$ es acotada?

    Considere $\varepsilon=1$

    \[\implies ||f_n-f_{\bar N}||_\infty \leq 1\]

    cuando $n\geq \bar N:=N(1)$.

    \begin{align*}||f_n||_\infty &\leq ||f_{\bar N}||_\infty + ||f_n-f_{\bar N}||_\infty\\
    &\leq ||f_{\bar N}||_\infty +1\end{align*}

    \[\implies f(x)=\lim_{n\to\infty} f_n(x)\text{ es acotada}\]

    \begin{fdefinition}
        Sea $(V,||\cdot||)$ un espacio normado. $v_n\in V,n\in\N$. $\displaystyle\sum_{n=1}^\infty v_n$ es \color{red} sumable \color{black} si

        \[S_m=\sum_{n=1}^m v_n\]

        converge.

        $\displaystyle\sum_{n} v_n$ es \color{red} absolutamente sumable \color{black} si 

        \[\sum_{n=1}^\infty ||v_n||\]

        converge.
    \end{fdefinition}
\end{proof}

\begin{fproposition}
    Si $\displaystyle\sum_{n=1}^\infty v_n$ es absolutamente sumable, entonces, $\{S_m\}$ es Cauchy
\end{fproposition}

\begin{ftheorem}
    Un espacio normado $(V,||\cdot||)$ es un espacio de Banach si y solo si toda serie absolutamente sumable es sumable.
\end{ftheorem}

\begin{proof}
    $\impliedby:$ \begin{enumerate}
        \item Tome una sucesión $\{v_n\}$ de Cauchy. Es suficiente demostrar que una subsucesión converge. $v_{n_k}\to v$ en $V$. Fije $\varepsilon>0$.
        $\implies ||v_m-v||\leq \underbrace{||v_m-v_{n_k}||}_{\leq \varepsilon/2}+\underbrace{||v_{n_k}-v||}_{\leq \varepsilon/2}\leq \varepsilon$, tomando $k,m$ suficientemente grandes.

        \item Dos trucos: Podemos ``acelerar'' la convergencia. Existe una subsucesión $\{v_{n_k}\}$ tal que 
        
        \begin{equation}||v_{n_{k+1}}-v_{n_k}||\leq 2^{-k}\end{equation}

        \[||v_n-v_m||\leq 2^{-k}\quad \forall n,m\geq N(2^{-k}):=N_k\]

        $n_k:=N_1+\ldots+N_k$

        Afirmamos que $\{v_{n_k}\}$ converge.

        Truco de la suma telescopica.

        \[\sum_{k=1}^\infty (v_{n_{k+1}}-v_{n_k})\]

        es absolutamente sumable debido a (1.1) entonces es sumable:

        \[\sum_{k=1}^m (v_{n_{k+1}}-v_{n_k})\xrightarrow{m\to\infty} S\in V\]

        Sumas parciales convergen 

        \[v_{n_{m+1}}-v_{n_1}\xrightarrow{m\to\infty}S\in V\]

        \[\implies v_{n_{m+1}}\xrightarrow{m\to\infty}S+v_{n_1}\in V\]

    \end{enumerate}
\end{proof}

\section{Operadores y funcionales}

Nos interesan las aplicaciones lineales entre espacios normados.

\begin{fexample}
    \begin{align*}T:C([0,1],\C)&\to C([0,1],\C)\\
    f&\to F(x)=\int_0^x f(y)\, dy\end{align*}

    $T$ es lineal.

    \[F(x)=\int_0^1 \mathds{1}_{\{y<x\}} f(y)\,dy\]
\end{fexample}

\begin{fdefinition}
    $V,W$ son 2 espacios vectoriales.

    $T:V\to W$ es lineal si 

    \[T(\lambda_1 v_1+\lambda_2 v_2)=\lambda_1 T(v_1)+\lambda_2 T(v_2)\quad \forall v_1,v_2\in V\text{ y } \lambda_1,\lambda_2\in \K\]

\end{fdefinition}

\begin{align*}T:C([0,1])&\to C([0,1])\\
f&\to \int_0^1 \underbrace{K(x,y)}_{\text{Kernel}} f(y)\,dy:=Tf(x)\end{align*}

operador integral. Cuando $K\in C([0,1]^2), T$ está bien definida.

En $\dimn \infty$ vamos a exigir que los operadores lineales sean \color{red} continuos \color{black}.

\begin{fdefinition}
    $T:V\to W$, $V,W$ son espacios métricos. Decimos que $T$ es continuo si
    
    \[T^{-1}(O)\overset{ab}{\subseteq} V,\, \forall O\overset{ab}{\subseteq} V\]

    \[\iff T^{-1}(C)\overset{cerr}{\subseteq} V\quad \forall C\overset{cerr}{\subseteq} W\]
    $\iff v_n\to v$ en $V$ entonces $Tv_n\to Tv$ en $W$.
\end{fdefinition}

\begin{ftheorem}
    Sean $V,W$ espacios normados. Entonces $T:V\to W$ operador lineal es continuo si y solo si

    \begin{equation}||Tv||_W\leq C||v||\quad \forall v\in V\end{equation}

    para alguna constante $C$.
\end{ftheorem}

\begin{fdefinition}
    Operador lineal que satisface $1.2$ se llama \color{red} acotado \color{black}.
\end{fdefinition}

\begin{proof}
    $\implies:$ Sea $T$ continuo. $B:=\{||w||_W<1\}$

    $0\in T^{-1}(B)=B_r^v$

    \[T^{-1}(B)\supseteq B_r^v:=\{v\in V:||v||_V<r\}\]

    pues $T^{-1}(B)$ es abierto 

    \[\implies T^{-1}(B)\supseteq \{v\in V:||v||_V=\frac{r}{2}\}\]

    esfera de radio $\frac{r}{2}$.

    \[||T\bar v||_W<1\]

    Todo $v\in V,v\neq 0$ se puede escribir como $v=\frac{\bar v}{r/2}||v||_V$

    Para algún $\bar v\in S_{r/2}^v$

    Por lo tanto

    \[||Tv||_W=||T(\frac{\bar v }{r/2}||v||_V)||_W\]

    \[=||\frac{2}{r}||v||_V T(\bar v)||_W\]
    \[=\frac{2}{r}||v||_V ||T\bar v||_W<1\]
    \[\leq \frac{2}{r}||v||_V\quad \forall v\neq 0\]
\end{proof}

\begin{fexample}
    \[Tf(x):=\int_0^1 K(x,y)f(y)\,dy\]

    es acotado en $(C([0,1]),||||_\infty )$

    \begin{align*}|Tf(x)|&\leq \int_0^1 \underbrace{|K(x,y)|}_{\leq M}|f(y)|\,dy\\
    &\leq M\int_0^1 |f(y)|\,dy\leq M ||f||_\infty\quad \forall x\implies ||Tf||_\infty \leq M||f||_\infty\end{align*}
\end{fexample}

\begin{fdefinition}
    Sean $V,W$ espacios normados. Defina $\mathcal{B}(V,W)$ como el conjunto de operadores lineales continuos acotados de $V$ a $W$.

    Obviamente $\mathcal{B}(V,W)$ es un espacio vectorial.
\end{fdefinition}

Norma operador $T:V\to W$:

\[||T||:=\sup_{||v||=1} ||Tv||\]

Obviamente, $T\in \mathcal{B}(V,W), ||T||<\infty$

\[||Tv||\leq C \underbrace{||v||}_1=C\]

\[\implies ||T||\leq C\]

De hecho, para $T\in \mathcal{B}(V,W)$

\begin{align*}
    ||T||&=\sup_{v\neq 0} \frac{||Tv||}{||v||}=\sup_{||v||\leq 1} ||Tv||\\
    &=\inf \{C>0:||Tv||\leq C||v||\quad \forall v\in V\}
\end{align*}

Tenemos $||Tv||\leq ||T||||v||$

\begin{ftheorem}
    $\mathcal{B}(V,W)$ es un espacio normado bajo la norma operador.
\end{ftheorem}

\begin{proof}
    \begin{enumerate}
        \item $||T||=0\implies ||Tv||=0\forall v\in V$
         
        $\implies Tv=0\implies T=0$.

        \item $||\lambda T||=|\lambda| ||T||$

        \item Sea $v\in V,||v||=1$. $\forall T,S\in\mathcal{B}(V,W),$
        
        \begin{align*}||(T+S)v||&=||Tv+Sv||\\
        &\leq ||Tv||+||Sv||\\
        &\leq ||T|| ||v||+||S|| ||v||=(||T||+||S||)||v||\end{align*}

    \begin{align*}
        &\implies ||(T+S)v||\leq ||T||+||S||\\
        &\implies ||T+S||\leq ||T||+||S||
    \end{align*}
    \end{enumerate}
\end{proof}

    ¿Cuándo es $\mathcal{B}(V,W)$ completo?

    \begin{ftheorem}
        $\mathcal{B}(V,W)$ es Banach cuando $W$ es Banach.
    \end{ftheorem}

    \begin{proof}
        $T_n\in \mathcal{B}(V,W)$ Cauchy. Queremos demostrar que converge en $||\cdot||_{\mathcal{B}(V,W)}$.

        \begin{enumerate}
            \item $\forall v\in V,\{T_n v\}$ es Cauchy en $W$ pues 
            \[||T_n v-T_n v||\leq ||T_n-T_w||\cdot ||v||\]

            $\implies \{T_nv\}$ converge. Definimos 

            \[Tv:=\lim_{n\to\infty} T_nv\]

            \item ¿Por qué $T\in \mathcal{B}(V,W)$?
            $\rightarrow$ lineal:

            \[T(\lambda v)=\lim_{n\to\infty} T_n(\lambda v)=\lambda \lim_{n\to\infty} T_n v=\lambda T(v)\]

            \[T(v_1+v_2)=T(v_1)+T(v_2)\]

            $\rightarrow$ acotado:

            $\{T_n\}$ es Cauchy.

            $\{||T_n||\}$ es Cauchy en $[0,\infty)$

            \[|||T_n||-||T_m|||\leq ||T_n-T_w||\]
            \[\implies ||T_n||\leq C\quad \forall n\in \N\]

            Sea $v\in V, ||v||=1$.

            \begin{align*}||Tv||&=||\lim_{n\to\infty} T_n v||\\
            &=\lim_{n\to\infty} \underbrace{||T_n v||}_{\leq C||v||=C}\leq C\end{align*}

            \[\implies ||T||\leq C\]

            \item Convergencia: $T_n\to T$ en norma operador. Sea $v\in V, ||v||=1$.
            \[||(T_n-T)v||\]

            $T_m v\to Tv$

            \begin{align*}&=\lim_{m\to\infty}||(T_n-T_m)v||\\
            &\leq \underbrace{||T_n-T_m||}_{\leq\varepsilon}\cdot ||v||\quad \forall n,m\geq N(\varepsilon)\\
            &\implies ||T_n-T||\leq \varepsilon\quad \forall n\geq N(\varepsilon)\end{align*}
        \end{enumerate}
    \end{proof}

    \subsection{Aplicaciones}

    \begin{fdefinition}
        Sea $V$ un espacio normado sobre $\K$.

        \[V^*=\mathcal{B}(V,\K)\]

        se llama el espacio \color{red} dual \color{black} de $V$.
    \end{fdefinition}

    \begin{ftheorem}
        Cuando $\K=\R,\C$ (completos) $V^*$ es un espacio de Banach
    \end{ftheorem}

    Elementos de $V^*$ se llaman \color{red} funcionales \color{black} en $V$.


    \begin{fexample}
        $[\ell^p(\C)]^* =?,p\in[1,\infty)$

        Resulta que $?=l^q(\C)$ donde $\frac{1}{p}+\frac{1}{q}=1$.

        Si $v\in \ell^p,w\in\ell^q$

        podemos definir un funcional en $\ell^p$

        \begin{align*}\ell_w:\ell^p(\C)&\to\C\\
        v=\{v_k\}&\to \sum_{k=1}^\infty v_k\bar w_k\end{align*}

        \[|\ell_w|\leq ||w||_{\ell^q}||v||_{\ell^p}\]

        Es la desigualdad de Hölder discreta.
    \end{fexample}

    $(\ell^1)^*\simeq \ell^\infty$
    $(\ell^2)^*\simeq \ell^2$

    \begin{fnote}
        $(\ell^\infty)^*\not\simeq \ell^1$
    \end{fnote}

Cuando $V=W$ espacio de Banach, entonces $B(V,V)$ es un espacio de Banach. Es también \color{red} álgebra \color{black}.

\[T,S\in B(V,V)\implies TS\in B(V,V)\]

\begin{align*}
    ||TS||=\sup_{||v||=1} ||T(Sv)||&\leq ||T||\cdot ||Sv||\\
    &\leq ||T||\cdot ||S||\cdot ||v||\leq ||T||\cdot ||S||
\end{align*}

Cómo resolver ecuaciones del tipo

\[(T-\lambda I)u=v\]

donde $v\in V\leftarrow$ un espacio de Banach, $T\in B(V,V)$, $\lambda\neq 0$.

Queremos construir el operador \color{red} inverso \color{black}

\[S:=(T-\lambda I)^{-1}\]

Cuando $|\lambda|>||T||$, $S$ se puede construir a través de la \color{red} serie de Neumann \color{black}

\[-\lambda (I-\underbrace{\frac{T}{\lambda}}_{||T/\lambda||<1})u=v\]

Sabemos que 

\[(1-x)^{-1}=\sum_{n=0}^\infty x^n\quad |x|<1\]

Definimos 

\begin{equation}S:=-\frac{1}{\lambda}\sum_{n=0}^\infty \left(\frac{T}{\lambda}\right)^n\label{eq:2.3}\end{equation}

\ref{eq:2.3} define $S\in B(V,V)$ ya que

\[\sum_{n=0}^\infty \left(\frac{T}{\lambda}\right)^n\]

es sumable pues es absolutamente sumable en el espacio de Banach $B(V,V)$.

$\rightarrow$ ¿por qué $(T-\lambda I)S=S(T-\lambda I)=I$?

Para verificar que $S(T-\lambda I)=I$, 

\[S_N=\sum_{n=0}^N -\frac{1}{\lambda} \left(\frac{T}{\lambda}\right)^n\]

\begin{align*}S_N(T-\lambda I)&=S_N T-S_N \lambda=\sum_{n=0}^N -\left(\frac{T}{\lambda}\right)^{n+1}-\sum_{n=0}^N -\left(\frac{T}{\lambda}\right)^{n}\\
&=\underbrace{-\left(\frac{T}{\lambda}\right)^{N+1}}_{\to 0 \text{ en }B(V,V)}+I\end{align*}

¿Cómo obtener espacios normados/Banach de otros espacios?

\begin{fdefinition}[Espacio cociente]
    Sea $W$ un subespacio del espacio vectorial $V$.

    \[V/W:=\{[v],v\in V\}\]

    $[\cdot]$ se define a través $v_1\sim v_2$ si $v_1-v_2\in W$.

    Se nota también $V\mod W$ y se llama el espacio cociente.
\end{fdefinition}

Es útil denotar $[v]=v+W$

Una construcción de subespacio $W\subseteq V$ tal que $V/W$ es normado es a través de una \color{red} semi-norma \color{black} definida en $V$.

\begin{fexample}
    $V=C^1([0,1])=$ espacio de funciones en $[0,1]$ con derivadas continuas en $[0,1]$.

    \[||f||:=\max_{t\in [0,1]}|f'(t)|\]

    \[||f||=0\iff f=\text{const}\]
\end{fexample}

\begin{ftheorem}
    Sea $(V,||\cdot||)$ un espacio vectorial semi-normado. Entonces $Z:=\{v\in V:||v||=0\}$ es un subespacio de $V$ y
    
    \begin{equation}||v+Z||_{V/Z}:=||v||\label{eq:2.4}\end{equation}
    
    define una norma en $V/Z$.
\end{ftheorem}

\begin{proof}
    \begin{enumerate}
        \item $Z$ es un subespacio vectorial.
        
        \[z_1,z_2\in Z\implies z_1+z_2\in Z\]

        \[||z_1+z_2||\leq ||z_1||+||z_2||=0\]

        \[z\in Z\implies \lambda z\in Z\]

        Así, $V/Z$ tiene la estructura de un espacio vectorial.

        \item Tenemos que comprobar que \ref{eq:2.4} es una buena definición:

        Si $v_1,v_2$ son 2 representantes de $[v]$:

        \[v_1=v_2+z,\quad z\in Z\]

        \begin{align*}
            ||v_1||\leq ||v_2||+||z||\implies ||v_1||\leq ||v_2||\\
            ||v_2||\leq ||v_1||\implies ||v_1||=||v_2||
        \end{align*}

        \[||v+z||_{V/Z}=0\]
        \[\implies v+Z=Z\implies v\in Z\]

        Las otras 2 proposiciones se heredan de manera obvia
    \end{enumerate}
\end{proof}

$C^1([0,1])/const$ es un espacio normado con la norma inducida.

Otra construcción similar:
\begin{fproposition}
    Si $W\overset{cerr}{\subseteq}V$ subespacio cerrado de un espacio normado $(V,||\cdot||)$, entonces $V/W$ tiene una norma:

    \[||[v]||_{V/W}:=\inf_{w\in W}||v-w||\]
\end{fproposition}

\begin{proof}
    En ayudantía
\end{proof}

\subsubsection*{Completación de espacios normados}

\begin{fdefinition}
    Sea $(V,||\cdot||)$ un espacio normado. La \color{red} completación \color{black} de $V$ es un espacio de Banach $(\tilde V,||\cdot||_{\tilde V})$ con una aplicación lineal 

    \[\mathcal{J}_{\tilde V}:V\to\tilde V\]

    que satisface las siguientes propiedades:

    \begin{enumerate}
        \item $\mathcal{J}_{\tilde V}$ es uno a uno
        \item $\mathcal{J}_{\tilde V}(V)$ es denso en $\tilde V$
        \item $\mathcal{J}_{\tilde V}(V)$ es una isometría:
        
        \[||\mathcal{J}_{\tilde V}(v)||_{\tilde V}=||v||_{V}\quad \forall v\in V\]
    \end{enumerate}
\end{fdefinition}

\begin{ftheorem}
    Todo espacio normado $V$ tiene una completación. Esta es única en el siguiente sentido:

    \seba{hacer dibujo}

    $\tilde V=\{\text{sucesiones de Cauchy en $V$ que convergen}\}$

    $\{v_n\}\sim\{w_n\}$ si $||v_n-w_n||\to 0$

    Sea $\tilde v\in \tilde V$ 

    \seba{ESTOY HASTA EL PICO}
\end{ftheorem}

\section{El teorema de Baire}

\subsection{Categorias de Baire}

$(X,d)$ espacio métrico.

\[B_r(x)=\{y\in X:d(x,y)<r\}\]
\[\overline{B_r}(x)=\{y\in X:d(x,y)\leq r\}\]

$O\subseteq X$ es abierto si $\forall x\in O,\exists B_r(x)\in O$. $\bigcup_\alpha O_\alpha$ es abierto.

$F\subseteq X$ es cerrado si $F^c$ es abierto. $\bigcap_\alpha F_\alpha$ es cerrado.

\[\overline{E}=\bigcap_{F\supseteq E} F\]

\[\mathring{E}=\bigcup_{O\subseteq E}O\]

\[E\overset{denso}{\subseteq} X \text{ si }\overline{E}=X\]


\begin{fdefinition}
    $E\subseteq X$ es \textbf{denso en ninguna parte} si $\mathring{\overline{E}}=\varnothing$.
\end{fdefinition}

esencialmente, denso en ninguna parte $E$ significa que $E$ no contiene bolas abiertas.

\begin{fexample}
    $E=\{x\}$ es denso en niguna parte.
\end{fexample}

\begin{fproposition}
    F es cerrado y denso en ninguna parte $\iff F^c$ es abierto y denso.
\end{fproposition}

\subsubsection*{La noción de categoria de Baire}

\begin{fdefinition}
    $E\subseteq X$ cat I si $E=\bigcup_{k}E_k$ donde $E_k$ es denso en ninguna parte.
\end{fdefinition}

\begin{fexample}
    $\mathbb{Q}$ es cat I.
\end{fexample}

\begin{fdefinition}
    Si $G$ tiene $G^c$ que es cat I, decimos que $G$ es \textbf{genérico}.
\end{fdefinition}

\begin{fdefinition}
    $E$ es de cat II si no es de primera categoría.
\end{fdefinition}

\paragraph*{Observaciones}

\begin{enumerate}
    \item Si $E$ es cat I, y $F\subseteq E$ es cat I
    
    \begin{align*}&F\subseteq E\subseteq \bigcup_k E_k\\
    &\implies F=\bigcup_k E_k\cap F,\quad \overline{E_k\cap F}\subseteq \overline{E_k}\\
    &\implies E_k\cap F \text{ son densos en niguna parte.}
    \end{align*}

    \item Si $\{E_k\}_{k\in\N}$ de cat I, $\bigcup_{k} E_k=\bigcup_{k}\bigcup_l \underbrace{E_{kl}}_{\text{denso en NP}}$ es una unión contable.
    
    \item No hay conexión entre conjuntos de cat I y conjuntos despreciables del punto de vista de teoría de la medida.
    \begin{fexample}
        $G_j=\bigcup_{n} (q_n-2^{-(n+j+1)},q_n+2^{-(n+j+1)})$

        $\{q_j\}$ enumeración de $\Q$.

        $G_j$ es abierto y denso en $\R$.

        \begin{align*}
            &\implies E_j=G_j^c \text{ es cerrado y denso en NP}\\
            &\implies E:=\bigcup_j E_j \text{ es cat I}
        \end{align*}

        y de plena medida en $\R$.

        $\iff E^c$ es de medida $0$ de Lebesgue.

        \begin{align*}
            |E^c|&=|\bigcap E_j^c|\\
            &=|\bigcap G_j|\leq |G_j|\\
            |G_j|&\leq \sum_{n=1}^\infty 2\cdot 2^{-(n+j+1)}\\
            &=2^{-j}\xrightarrow{j\to\infty} 0
        \end{align*}
    \end{fexample}
    
\end{enumerate}


\begin{ftheorem}[Teorema de Baire]
    Sea $(X,d)$ \textbf{completo}. Entonces, $X$ es de la cat II en sí mismo.
\end{ftheorem}

\begin{proof}
    Supongamos que $X$ es de cat I en sí:

    \[X=\bigcup_{k} \underbrace{E_k}_{\text{densos en NP}}=\bigcup_{k} \underbrace{\overline{E_k}}_{=F_k \text{ denso en NP y cerrado}}\]

    Llegaremos a una contradicción si demostramos que hay un $x\not\in F_k,\quad \forall k$.

    $F_1\neq X$. $\overline{B_{r_1}}(x_1)\subseteq F^c$, $\overline{B_{r_2}}(x_2)\subseteq F_2^c$.

    De esta manera obtenemos bolas cerradas $\overline{B_{r_k}}(x_k)$ tales que

    \begin{enumerate}
        \item \[\overline{B_{r_{k+1}}}(x_{k+1})\subseteq \overline{B_{r_k}}(x_k)\]
        \item \[\overline{B_{r_k}}(x_k)\subseteq F_k^c\]
        \item \[r_{k+1}\leq \frac{r_k}{2}\implies r_k\to 0\]
    \end{enumerate}

    $\{x_k\}$ es Cauchy pues:

    \[\forall k,l\geq n, x_k,x_l\in \overline{B_{r_n}}(x_n)\]

    \[\implies |x_k-x_l|\leq 2 r_n\xrightarrow{n\to\infty} 0\]

    \[\implies x_k\to x\in X\]

    Como $x_k\in \overline{B_{r_k}}\quad \forall k\geq n$,

    \[\implies x=\lim x_k\in \overline{B_{r_n}}(x_n)\subseteq F_n^c\]

    Por lo que $x\not\in F_n\quad \forall n$.
    
\end{proof}

\begin{fcorollary}\label{theo:2.3.2.1}
    $G\subseteq X$ es \textbf{genérico} $\implies$ denso en $X$, con $X$ completo.
\end{fcorollary}

\begin{proof}
    Asumimos que $G$ genérico no es denso, entonces hay una bola $B$

    \[\implies \overline{B}\subseteq G^c=\bigcup_k {E_k}\subseteq \bigcup \overline{E_k}\]

    \[\implies \overline{B}=\bigcup_{k} \underbrace{\overline{E_k}\cap \overline{B}}_{\text{cerrados y densos en NP}}\]

    Pero $\overline{B}$ es un espacio métrico completo, contradicción con el teorema de Baire.
\end{proof}

\begin{fcorollary}
    $X$ completo, $X=\bigcup_k F_k\leftarrow$ cerrado.

    Entonces, por lo menos uno $F_k$ contiene una bola.
\end{fcorollary}

\subsection{Aplicación}

\begin{ftheorem}
    El conjunto de funciones continuas en $[0,1]$ que no son derivables en nigún punto es \textbf{denso} en $C([0,1])$
\end{ftheorem}

\begin{proof}
    Sea $\mathcal{D}=\{f\in C([0,1]):f'(x_*) \text{ existe en un punto } x_*\in [0,1]\}$

    Queremos demostrar que $\mathcal{D}$ es cat I en $C([0,1])$.

    Por \ref{theo:2.3.2.1}, $\mathcal{D}^c$ es genérico $\implies$ denso en $C([0,1])$.

    Si $f\in \mathcal{D}\implies f'(x_*)$ existe

    \[\implies \lim_{x\to x_*} \frac{f(x)-f(x_*)}{x-x_*}\]

    existe.

    \[\implies |f(x)-f(x_*)|\leq M|x-x_*|\quad \forall x\in [0,1]\]

    para algún $M>0$.

    \[\implies \mathcal{D}\subseteq \bigcup_{N=1}^\infty E_N\]

    $E_N:=\{f\in C([0,1]):|f(x)-f(x_*)|\leq N|x-x_*|\text{ para algún $x_*\in[0,1]$}\}$

    Estaremos listos si probamos que:

    \begin{enumerate}
        \item $E_N$ es cerrado en $C([0,1])$
        \item $E_N$ es denso en ninguna parte.
    \end{enumerate}

    \begin{enumerate}
        \item $f_n\in E_N$ y $f_n\to f$, en $||\cdot||_\infty$.
        
        $[0,1]\ni x_n^*\to x^*$ (podemos extraer una subsucesión que converge)

        \[|f_n(x)-f_n(x_n^*)|\leq N|x-x_n^*|\quad \forall x\in[0,1]\]

        Queremos demostrar que 

        \[|f(x)-f(x^*)|\leq N|x-x^*|\]

        \begin{align*}
            |f(x)-f(x^*)|\leq \underbrace{|f(x)-f_n(x)|}_{\leq ||f-f_n||_\infty\leq \varepsilon/2}+|f_n(x)-f_n(x^*)|+\underbrace{|f_n(x^*)-f(x^*)|}_{\leq \varepsilon/3}
        \end{align*}

        \begin{align*}
            |f_n(x)-f_n(x^*)|&\leq |f_n(x)-f_n(x^*)|+|f_n(x_n^*)-f_n(x^*)|\\
            &\leq N|x-x_n^*|+N|x_n^*-x^*|\\
            &\leq N(|x-x^*|+|x^*-x_n^*|)+N|x_n^*-x^*|\\
            &\leq N|x-x^*|+\underbrace{2N|x_n^*-x^*|}_{\varepsilon/3}
        \end{align*}

        \item ¿Por qué $E_N$ es denso en NP de $X$?
        \[P_M=\{\text{funciones continuas en $[0,1]$ derivables a trozos, }|f'|=M\}\]

        son funciones zig-zag. Cuando $M>N$, $P_M\cap E_N=\varnothing$. Además, $P_M$ es denso en $C([0,1])$.
        Como consecuencia, $E_N$ no puede tener interior no trivial ya que $E_N$ no puede tener una bola abierta (hay funciones de $P_M$ en $E_N$ y $P_M$ es denso).

        Mostraremos que  $P_M$ es denso.

        \[P=\{\text{las funciones continuas lineales a tozos}\}\overset{denso}{\subseteq}C([0,1])\]

        Podemos aproximar cada $f\in P$ con una función $g\in P_M$ arbitrariamente bien.
    \end{enumerate}
\end{proof}

\subsection{Teorema de la Aplicación Abierta}

Sean $(X,||\cdot||_X),(Y,||\cdot||_Y)$ espacios de Banach.

\[T\in \mathcal{B}(X,Y)\implies T^{-1}(O)\overset{ab}{\subseteq}X\quad \forall O\overset{ab}{\subseteq}Y\]

Si $T$ es biyectiva adicionalmente, entonces $S:=T^{-1}$ es lineal (no necesariamente acotada).
Sin embargo, si $S$ es continua, entonces $S^{-1}(U)\overset{ab}{\subseteq},\forall U\overset{ab}{\subseteq}X$

\[\iff T(U)\overset{ab}{\subseteq}Y\quad \forall U\overset{ab}{\subseteq}X\]

\begin{fdefinition}
    Sea $T:X\to Y$ una aplicación. Decimos que $T$ es abierta si 
    
    \[T(U)\overset{ab}{\subseteq}Y\quad \forall U\overset{ab}{\subseteq}X\]
\end{fdefinition}

Si $T:X\to Y$ es lineal, continua y biyectiva, entonces $T^{-1}:Y\to X$ es lineal. ¿Es $T^{-1}$ continua?

Lo será cuando $T$ es abierta.

\begin{ftheorem}[Aplicación Abierta]
    Si $X,Y$ son espacios de Banach, $T\in \mathcal{B}(X,Y)$ y sobreyectiva, entonces $T$ es abierta.
\end{ftheorem}

\begin{fcorollary}\label{theo:2.3.4.1}
    Si $X,Y$ son espacios de Banach, $T\in\mathcal{B}(X,Y)$ es biyectiva, entonces $T^{-1}\in\mathcal{B}(Y,X)$. Existen $c,C>0$ tales que

    \[c||x||_X\leq ||\underbrace{Tx}_{y}||_Y\leq C||x||_X\quad \forall x\in X\]

    \[c||T^{-1}y||_X\leq ||y||_Y\]
\end{fcorollary}

\begin{proof}[Demostración del teorema \ref{theo:2.3.4}]
    \begin{enumerate}
        \item Será suficiente demostrar que $T(B_2^X)\supseteq B_\delta^Y$. ($B_r^X=B_r^X(0)$)
        
        Por linealidad

        \begin{align*}
            T(B_r^X(x))&=T(x+B_r^X)\\
            &=Tx+T(B_r^X)={y+\frac{r}{2}T(B_2^X)}\\
            &\supseteq y+\frac{r}{2}B_\delta^Y=B_{\frac{\delta r}{2}}^Y(y)\\
        \end{align*}

        \item Vamos a demostrar que $\overline{T(B_1^X)}\supseteq B_\delta^X$ para algún $\delta>0$
        
        Por la sobreyectividad:

        \[cat II\rightarrow Y=\bigcup_{n=1}^\infty \overline{T(B_n^X)}\]

        Entonces, $T(B_n^X)\supseteq B_r^Y(y)$ para algún $n\in\N,r>0,y\in Y$. Tomamos $\tilde y$ tal que $|\tilde y-y|\leq \frac{r}{2}$ e $\tilde y=Tx$ para algún $x\in B_n^X$.

        \begin{align*}
            T(B_{2n}^x(\tilde x))\supseteq \overline{T(B_n^X)}\supseteq B_r^Y(y)\supseteq B_{\frac{r}{2}}^Y(y)
        \end{align*}

        Restando $Tx$

        \[T(B_{2n}^X)\supseteq B_{\frac{r}{2}}^X\]

        Reescalando

        \[\overline{T(B_1^X)}\supseteq B_{\frac{r}{4n}^Y}\quad \delta=\frac{r}{4n}\]

        \item Tenemos $T(B_1^X)\supseteq B_\delta^Y$. Reescalando
        
        \[\overline{T(B_{2^{-k}}^X)}\supseteq B_{\delta 2^{-k}}^Y\]

        ¿Por qué $T(B_2^X)\supseteq B_\delta^Y$?

        Fije $y_0\in B_\delta^Y$. Podemos encontrar $x_0\in B_1^X$ tal que 

        \[||y_0-Tx_0||_Y<\frac{\delta}{2}\]

        \[\implies y_1:=y_0-Tx_0\in B_{\delta/2}^Y\]

        $\implies$ existe $x_1\in B_{\frac{1}{2}}^X$ tal que 

        \[||y_1-Tx_1||<\frac{\delta}{4}\]

        De esta manera construimos sucesiones $\{x_n\},\{y_n\}$, tales que

        \begin{enumerate}
            \item $x_n\in B_{2^{-n}}^X, y_n\in B_{\delta 2^{-n}}^Y$
            \item $y_{n+1}=y_n-Tx_n$
        \end{enumerate}

        $x:=\displaystyle\sum_{n=0}^\infty x_n\in X$ porque $X$ es Banach. Veremos que $Tx=y$ y $x\in B_2^X$.

        $x$ es convergente puesto que es absolutamente convergente.

        \[||x||=\sum_{n=1}^\infty ||x_k||\leq 2\]

        Afirmamos que $Tx=y_0$. Por contradicción

        \begin{align*}Tx&=\lim_{N\to\infty} T(\sum_{n=0}^N x_k)\\
        &=\lim_{N\to\infty} \sum_{k=0}^N \underbrace{Tx_k}_{y_k-y_{k+1}}\\
        &=\lim_{N\to\infty} (y_0-y_{N+1})\\\
        &=y_0
    \end{align*}

    ya que $y_{N+1}\to 0$.
    \end{enumerate}
\end{proof}

\subsection{Teorema del Grafo Cerrado}

\begin{fdefinition}
    Sean $X,Y$ espacios métricos. Decimos que $T:X\to Y$ es \textbf{cerrada} si su grafo en $X\times Y$

    \[G_T=\{(x,Tx)\in X\times Y\}\]

    es cerrado en $X\times Y$.
\end{fdefinition}

En otras palabras,

\[(x_n,Tx_n)\to (x,y)\in X\times Y\implies (x,y)\in G_T\iff y=Tx\]

\begin{fnote}
    $T:X\to Y$ es continua $\implies T$ es cerrada.

    \[x_n\to x\implies Tx_n\to Tx\implies (x_n,Tx_n)\to (x,Tx)\]
\end{fnote}

\begin{ftheorem}
    Sean $X,Y$ Banach. Entonces, $T\in \mathcal{B}(X,Y)\iff T$ es lineal y cerrada.
\end{ftheorem}

\begin{proof}
    $\impliedby:$ Utilizaremos el hecho que si $X,Y$ son Banach, entonces $X\times Y$ es Banach.

    \[||(x,y)||_{X\times Y}:=||x||_X+||y||_Y\]

    \[G_T:=\{(x,Tx)\}\subseteq X\times Y\]

    \begin{enumerate}
        \item $G_T$ es un subespacio de $X\times Y$.
        \item $G_T\overset{cerr}{\subseteq}X\times Y$

        Entonces $G_T$ es un espacio de Banach en sí. Tenemos las proyecciones $\Pi_X:G_T\to X$ y $\Pi_Y:G_T\to Y$ continuas y lineales.

        \[T=\Pi_Y\circ (\Pi_X)^{-1}\]

        ya que $\Pi_x$ es biyectiva, continua y lineal (en un espacio de Banach a otro Banach). Por el teorema $\ref{theo:2.3.4.1}$, $\Pi_X^{-1}$ es continua. Por lo que $T=\Pi_Y\circ \Pi_X^{-1}$ es continua.
    \end{enumerate}
\end{proof}

\paragraph*{Significado} Si queremos demostrar que una aplicación lineal $T:X\to Y$ es continua, $x_n\to X\implies Tx_n\to T_x$

Podemos asumir adicionalmente que $TX_n\to Ty$, y demostrar que $y=Tx$