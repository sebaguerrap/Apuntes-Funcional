\chapter{Teoría de Operadores}

\section{Relaciones de Ortogonalidad}

\paragraph*{Notación:} $x\in X,f\in X^*$, $X$ normado. $f(x)\coloneqq \inn{f,x}$. $Y\subseteq X$, definimos el \textbf{aniquilador} de $Y$

\[Y^\perp\coloneqq \{f\in X^*:\inn{f,y}=0\quad\forall y\in Y\}\subseteq X^*\]

Similarmente, $Z\subseteq X^*$,

\[Z^\perp\coloneqq \{x\in X:\inn{f,x}=0\quad\forall f\in Z\}\subseteq X\]

Obviamente $Y^\perp$ es un subespacio cerrado de $X^*$ y $Z^\perp$ es un subespacio cerrado de $X$.

\begin{fexample}
   Cuando $X$ es un espacio de Hilbert, $X^*\simeq X$ por Riesz. $Y\subseteq X$, el complemento ortogonal 

   \[Y^\perp=\{x\in X:\inn{x,y}=0\quad\forall y\in Y\}\]

   $\simeq$ aniquilador de $Y$.
\end{fexample}

\begin{fproposition}
   Sea $Y\subseteq X$ subespacio del espacio normado $X$. Entonces, $(Y^\perp)^\perp=\overline{Y}$
\end{fproposition}

\begin{proof}
   Es fácil ver que $Y\subseteq (Y^\perp)^\perp$.\\
   Para demostrar la otra inclusión, suponga que $\overline{Y}\subsetneq (Y^\perp)^\perp$. Entonces existe $x\neq 0$, $x\in (Y^\perp)^\perp\setminus \overline{Y}$. Defina

   \begin{align*}
       f:\overline{Y}+\gen(x)&\to\K\\
       y+\lambda x&\to \lambda
   \end{align*}

   Obviamente $f$ es un funcional lineal en $\overline{Y}+\gen(x)$ que satisface:

   \[f(x)=1\]
   \[f(y)=0\]

   Además $f$ es acotado en $Z\coloneqq \overline{Y}+\gen(x)$:

   \[f(y)=0\quad\forall y\in Y\]

   Sea $z=y+\lambda x,\lambda\neq 0$. $\implies z\neq 0$.

   \begin{align*}
       |f(z)|&=|\lambda|=\frac{|\lambda|}{||z||}||z||\\
       &=\frac{|\lambda|}{||y+\lambda x||}||z||=\frac{1}{||\frac{y}{\lambda}+x||}||z||\leq \frac{1}{dist(x,\overline{Y})}||z||
   \end{align*}

   Por Teorema de Hahn-Banach, podemos extender $f$ a todo $X$, y asumir que $f\in X^*$. Además,

   \[\inn{f,y}=0\quad\forall y\in \overline{Y}\implies f\in \overline{Y}^\perp\supseteq Y^\perp\]

   pero $f(x)\neq 0$. Por otro lado, 

   \[x\in (Y^\perp)^\perp\implies \inn{g,x}=0\quad\forall g\in Y^\perp\]

   En particular, $\inn{f,x}=0$, lo que es una contradicción.
\end{proof}

Suponga que $T\in \mathcal{B}(X,Y)$, $X,Y$ normados. Definimos el \textbf{operador adjunto/transpuesto}

\begin{align*}
   T^*:Y^*&\to X^*\\
   f&\to f\circ T\eqqcolon T^*(f)\\
   \inn{T^*f,x}&=\inn{f,T(x)}\quad\forall x\in X
\end{align*}

Obviamente $T^*$ es lineal y es acotado:

\begin{align*}
   |T^*f(x)|=|f(Tx)|\leq ||f||_{Y^*}||Tx||_{Y}&\leq ||f||_{Y^*}||T||_{\mathcal{B}(X,Y)}||x||_X\\
   \implies ||T^*f||_{X^*}&\leq ||f||_{Y^*}||T||_{\mathcal{B}(X,Y)}\\
   \implies ||T^*||_{\mathcal{B}(Y^*,X^*)}&\leq ||T||_{\mathcal{B}(X,Y)}\\
   \implies ||T^*||&\in\mathcal{B}(Y^*,X^*)
\end{align*}

\begin{ftheorem}
   La asignación 

   \begin{align*}
       \mathcal{B}(X,Y)&\to \mathcal{B}(Y^*,X^*)\\
       T&\to T^*
   \end{align*}

   es una isometría lineal. Además,

   \begin{enumerate}[label=(\alph*)]
       \item $(\operatorname{Im} T)^\perp=\ker T^* (\subseteq Y^*)$
       \item $(\ker T^*)^\perp=\overline{\operatorname{Im}T} (\subseteq Y)$
       \item $(\operatorname{Im}T^*)^\perp=\ker T (\subseteq X)$
   \end{enumerate}
\end{ftheorem}

\begin{proof}
   Obviamente, $(\lambda T_1+T_2)^*=\lambda T_1^*+T_2^*$

   \begin{align*}
       ||T||&=\sup_{||x||_X=1}||Tx||_Y\\
       &=\sup_{||x||_X=1}\left(\sup_{||f||_{Y^*}=1}|\inn{f,Tx}|\right)\\
       &=\sup_{\substack{||x||_X=1\\||f||_{Y^*}=1}}|\inn{f,Tx}|=\sup|\inn{T^*f,x}|\\
       &=\sup_{||f||_{Y^*}=1}\sup_{||x||_X=L}|T^*f(x)|\\
       &=\sup_{||f||_{Y^*}=1}||T^*f||=||T^*||
   \end{align*}

   \begin{enumerate}[label=(\alph*)]
       \item \begin{align*}
           f\in (\operatorname{Im}T)^\perp&\iff \inn{T^*f,x}=\inn{f,Tx}=0\quad\forall x\in X\\
           &\iff T^*f=0\iff f\in \ker T^*
       \end{align*}

       \item \begin{align*}
           (\ker T^*)^\perp=((\operatorname{Im}T)^\perp)^\perp=\overline{\operatorname{Im}T}
       \end{align*}

       \item \begin{align*}
           x\in (\operatorname{Im}T^*)&\iff \inn{T^*f,x}=0\quad\forall f\in Y^*\\
           &\iff \inn{f,Tx}=0\quad\forall f\in Y^*\\
           &\iff Tx=0\iff x\in \ker T 
       \end{align*}
   \end{enumerate}
\end{proof}

\section{Operadores Compactos}

\begin{fdefinition}
   Sean $X,Y$ espacios de Banach.

   \[B^X\coloneqq \{x\in X:||x||_X\leq 1\}\]

   Decimos que un operador lineal $T:X\to Y$ es \textbf{compacto} si $\overline{T(B^X)}$ es compacto en $Y$.\\
   $\iff$ toda sucesión en ${T(B^X)}$ tiene una subsucesión convergente en $Y$.\\
   $\iff$ toda sucesión en $T(B^X)$ tiene una subsucesión de Cauchy.

   Denotamos la clase de opradores \textbf{compactos} con $\mathcal{B}_c(X,Y)$
\end{fdefinition}

\begin{ftheorem}
   $\mathcal{B}_c(X,Y)\subseteq \mathcal{B}(X,Y)$ es un subespacio cerrado. 

   \[\{T_n\}\subseteq \mathcal{B}_c(X,Y)\text{ y }||T_n-T||\xrightarrow{n\to\infty} 0\implies T\in \mathcal{B}_c(X,Y)\]
\end{ftheorem}

\begin{proof}
   % $T$ compacto $\implies T$ es acotado. $\overline{T(B^X)}$ es compacto en $Y$.

   % \begin{align*}
   %     \overline{T(B^X)}&\subseteq \bigcup_{n=1}^\infty B_n^Y\\
   %     &\subseteq B_N^Y\quad\text{para algún $N\in \N$}\\
   %     \implies ||Tx||\leq N\quad&\forall ||x||=1\implies ||T||\leq N
   % \end{align*}

   % $\mathcal{B}_c(X,Y)$ es un subespacio de $\mathcal{B}(X,Y)$:

   % \[T\text{ compacto}\implies \lambda T\text{ compacto}\]

   % $T_1,T_2$ compactos. Tome $\{x_n\}\in B^X$, entonces existe una subsucesión $\{x_{n_k}\}_k$: $T_1 x_{n_k}\to y_1$, $T_2x_{n_k}\to y_2$

   % \[\implies (T_1+T_2)x_{n_k}\to y_1+y_2\implies T_1+T_2\text{ es compacto}\]

   Fije $\varepsilon>0$. Elige $N$ grande tal que 

   \[||T-T_n||\leq \frac{\varepsilon}{2}\]

   \[\overline{T_N(B^X)}\subseteq \bigcup_{k=1}^M B_{\varepsilon/2}(y_k)\]

   \[T(B^X)\subseteq \bigcup_{k=1}^M B_{\varepsilon}(y_k)\]

   Tomaremos $\varepsilon_n=\frac{1}{m}\to 0$. Sea $\{x_n\}\subseteq B^X$.

   \begin{enumerate}
       \item Extraemos una subsucesión $\{x_{n,1}\}$ tal que $Tx_{n,1}\subseteq B_{\varepsilon_1}(\bar y_1)$
       
       \item $\{x_{n,1}\}\to\{x_{n,2}\}$ tal que 
       
       \[Tx_{n,2}\subseteq B_{\varepsilon_2}(\bar y_2)\]

       así hasta $\{x_{n,m}\}$ tal que

       \[Tx_{n,m}\subseteq B_{\varepsilon_m}(\bar y_m)\]
   \end{enumerate}

   Definimos $\tilde x_m\coloneqq x_{m,m}$. $\{T\tilde x_m\}$ es una sucesión de Cauchy.
\end{proof}

\begin{fdefinition}
   Un operador $T:X\to Y$ es de \textbf{rango finito} si $\im T$ tiene $\dimn$ finita.
\end{fdefinition}

\begin{fproposition}
   Un operador $T:X\to Y$ de rango finito es compacto. Como consecuencia si $T=\lim_{n\to\infty}T_n$, $T_n$ de rango finito, entonces $T\in\mathcal{B}_c(X,Y)$
\end{fproposition}

\begin{proof}
   $X\xrightarrow{T}\im T\simeq \K^m$, $m=\dimn \im T$.

   \begin{align*}
       \varphi:\K^m&\to\im T\\
       (c_1,\ldots,c_m)&\to \sum_{i=1}^m c_ie_i
   \end{align*}

   donde $\{e_i\}$ es una base de $\im T$. $\varphi$ es un isomorfismo continuo (con inversa continua). Por lo tanto, $\varphi^{-1}(\overline{T(B_x)})\subseteq K^m$ cerrado y acotado $\implies \varphi^{-1}(\overline{T(B_X)})$ es compacto en $K^m$. Por lo tanto, $\overline{T(B_X)}$ es compacto en $\im T\subseteq Y$.
\end{proof}

Q. Es un operador $T:\mathcal{B}_c(X,Y)$, límite de operadores de rango finito?.

A. En general, no. Sí, en el caso cuandso $Y$ es un espacio de Hilbert.

\begin{ftheorem}
   $T\in \mathcal{B}_c(X,Y)$, $Y$ espacio de Hilbert. Entonces existen $T_n$ de rango finito tal que 

   \[||T_n-T||\xrightarrow{n\to\infty}0\]
\end{ftheorem}

\begin{proof}
   Fije $\varepsilon>0$. $\overline{T(B^X)}\subseteq \bigcup_{k=1}^M B_\varepsilon(y_k)$

   \[F\coloneqq \gen(\{y_k\}_{k=1}^M)\overset{\text{cerr}}{\subseteq} Y\]

   Tenemos $P_F:Y\to Y$.

   \[T_\varepsilon\coloneqq P_F\circ T\quad\text{es de rango finito}\]

   Ahora, cada $x\in B^x$ tiene imagen $Tx\in B_\varepsilon(y_k)\implies ||Tx-y_k||<\varepsilon$

   \[||P_F(Tx)-P_F(y_k)||\leq ||Tx-y_k||\leq\varepsilon\]
   \[||T_\varepsilon x-y_k||<\varepsilon\]

   Concluimos que $\forall x\in B^X$,

   \[||Tx-T_\varepsilon x||\leq ||Tx-y_k||+||T_\varepsilon x-y_k||\leq 2\varepsilon\]
   \[\implies ||T-T_\varepsilon||\leq 2\varepsilon\]
\end{proof}

\begin{fexample}[Operadores de Hilbert-Schmidt]
   $X_i\coloneqq(\Omega_i,\mu_i)$, $i=1,2$.

   \[K(x_1,x_2)\in L^2_\R(X_1\times X_2)\]

   Sea $f\in L^2(X_2)$. Defina

   \[(T_k f)(x_1)\int_{\Omega_2}K(x_1,x_2)f(x_2)\,d\mu_2\]

   $T_k$ es un operador $\mathcal{B}(L^2(X_2),L^2(X_1))$.

   \begin{align*}
       |T_k f(x_1)|^2&\leq \left(\int_{\Omega_2}|K(x_1,x_2)||f(x_2)|\,d\mu_2\right)^2\\
       &\leq \underbrace{\int_{\Omega_2} |K(x_1,x_2)|^2\,d\mu_2}_{\text{finita } \mu_1-\text{c.t.p.}} ||f||_{L^2(X_2)}^2
   \end{align*}

   \[\int\left(\int K(x_1,x_2)^2\,d\mu_2\right)d\mu_1<\infty\]

   \begin{align*}
       \int_{\Omega_1}|T_k f(x_1)|^2\,d\mu_1(x_1)&\leq ||K||_{L^1(X_1\times X_2)}^2||f||_{L^2(X_2)}^2\\
       \implies T_k f&\in L^2(X_1)
   \end{align*}

   y 

   \[||T_k f||_{L^2(X_1)}\leq ||K||_{L^2(X_1\times X_2)}||f||_{L^2(X_2)}\]

   Además, $\forall g\in L^2(X_1)$, 

   \begin{align*}
       \inn{T_k f,g}_1&=\int_{X_1\times X_2} K(x_1,x_2)f(x_2)g(x_1)\,d(\mu_1\times \mu_2)\\
       &=\int_{X_2}\left(\underbrace{\int_{X_1}K(x_1,x_2)g(x_1)\,d\mu_1}_{T_{K^*}g, K^*(x_2,x_1)=K(x_1,x_2)}\right) f\,d\mu_2=\inn{f,T_{K^*}g}_2
   \end{align*}

   \[T_K^*=T_{K^*}\]

   Asumimos que $L^2(X_1), L^2(X_2)$ son separables. Sean $\{e_m\}_m$ una base o.n. de $L^2(X_1)$, $\{f_n\}_n$ una base o.n. de $L^2(X_2)$. 

   \[\{h_{mn}(x_1,x_2)\coloneqq e_m(x_1)f_n(x_2)\}_{mn}\text{ es una base o.n. de }L^2(X_1\times X_2)\]

   (Por Fubini $h_{mn}$ es maximal)

   \[K(x_1,x_2)=\sum_{m,n}a_{mn}e_m(x_1)f_n(x_2)\]
   \[K_N(x_1,x_2)=\sum_{\substack{n\leq N\\m\leq n}}a_{mn}h_{mn}(x_1,x_2)\]
   \[||K-K_N||_{L^2(X_1\times X_2)}^2=\sum_{m\text{ ó }n>N} |a_{mn}|^2\xrightarrow{n\to\infty} 0\]
   \[||T_K-T_{K_N}=||T_{K-K_n}||\leq ||K-K_N||_{L^2(X_1\times X_2)}\]

   $T_{K_N}$ es un operador de rango finito!

   \begin{align*}
       T_{K_N} f(x_1)&=\inn{K_N(x_1,\cdot),f}_{L^2(X_2)}\\
       &=\inn{\sum_{\substack{m\leq N\\n\leq N}}a_{mn}e_m(x_1)f_n(x_2),f(x_2)}\\
       &=\sum_{\substack{m\leq N\\n\leq N}}a_{mn}\inn{f_n,f}_{L^2(X_1)}e_m(x_1)
   \end{align*}

   \[\implies \im T_{K_N}\subseteq \gen(\{e_m\}_{m=1}^N)\]
\end{fexample}

\begin{fproposition}
   Composición de un operador compacto y un operador continuo es compacto.

   \[X\xrightarrow[\text{compacto}]{T}Y\xrightarrow[\text{continuo}]{S} Z\]

   $S\circ T$ es compacto.

   \[Z\xrightarrow[\text{continuo}]{S}X\xrightarrow[\text{compacto}]{T} Y\]
\end{fproposition}

\begin{proof}
   $\{x_n\}\subseteq B^X$. Por composición,

   \[Tx_{n_k}\to y\implies (S\circ T)(x_{n_k})\to Sy\text{ converge}\implies S\circ T\in\mathcal{B}_c(X,Z)\]

   Por otro lado,
   
   \[\{z_n\}\subseteq B^Z\implies x_n\coloneqq S z_n\in B_{||S||}^X\implies \frac{x_n}{||S||}\in B^X\implies T(\frac{x_{n_k}}{||s||})\to \frac{y}{||s||}\]

   \[\implies T\circ S(z_{n_k})\to y\]
   \[\implies T\circ S\in \mathcal{B}_c(Z,Y)\]
\end{proof}